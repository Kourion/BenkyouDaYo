\documentclass{article}
\usepackage{xeCJK}
\usepackage{enumitem}
\usepackage{geometry}
\usepackage{titlesec}
\usepackage{lipsum}
\usepackage[CJK,overlap]{ruby}
 \geometry{
 a4paper,
 %total={170mm,257mm},
 left=10mm,
 top=12mm,
 right=10mm,
 bottom=12mm,
 }
 
\renewcommand{\rubysep}{0.1ex}%3.9ex}
\makeatletter
\renewcommand\paragraph{\@startsection{paragraph}{4}{\z@}%
            {-0.5ex\@plus -.1ex \@minus -.25ex}%
            {-1em}%
            {\normalfont\normalsize\bfseries}}
\renewcommand\subparagraph{\@startsection{subparagraph}{5}{\z@}%
            {-0.1ex\@plus -.1ex \@minus -.25ex}%
            {-1em}%
            {\normalfont\normalsize\bfseries}}
\makeatother

\setlist{nolistsep}
\titlespacing*{\section}
{0pt}{5.5ex plus 1ex minus .2ex}{4.3ex plus .2ex}
\titlespacing*{\subsection}
{0pt}{0.5ex plus 1ex minus .2ex}{0.5ex plus .2ex}

\newcommand\tab[1][1cm]{\hspace*{#1}} 
\newcommand\tabni[1][0.2cm]{\hspace*{#1}}
\newcommand\tabyon[1][0.4cm]{\hspace*{#1}}
\newcommand\tabroku[1][0.6cm]{\hspace*{#1}}

\XeTeXlinebreaklocale "ja"
\XeTeXlinebreakskip=0em plus 0.1em minus 0.01em

\setlength{\parindent}{0pt}
\setCJKmainfont{ipaexm.ttf}
\setcounter{tocdepth}{5}

\begin{document}	%\large {1.5ex \@plus .2ex}%

\tableofcontents
\newpage

\part{\ruby{日本文化地理概論}{にほんぶんかちりがいろん}}
\section{\ruby{宗教}{しゅうきょう} \tabni religion | \ruby{社会思想}{しゃかいしそう} \tabni societal-thought }
%\subsection{社会 (しゃかい) society}
\begin{enumerate}[itemsep=-6pt]\setlength\itemsep{-0.4em}
\item Die großen Probleme der Bevölkerung: 
\begin{itemize}%\setlength\itemsep{-0.4em}
\item 失業 (しつぎょう)--> unemployment
\begin{itemize}
\item 失業りつ --> unemploment rate
\begin{itemize}%\setlength\itemsep{-0.4em}
\item りつ = rate
\end{itemize}
\item In Japan only qualified job-seekers are counted as unemployed.
\end{itemize}
\item 自殺 じさつ suicide
\begin{itemize}%\setlength\itemsep{-0.4em}
\item Korea is 1st in Suicides
\item Russian is 3rd in suicides
\item Japan is 4th in suicides \\
\end{itemize}
\end{itemize}
%\end{itemize}
%\begin{itemize}
\item \ruby{宗教}{しゅうきょう} \tabni religion
\begin{itemize}%\setlength\itemsep{-0.4em}
\item Shinto has 51\% buddhism 42.7\%, and christians 1.5\% of the population as followers.
\item People of Japan have the freedoms of religion and free speech. Guaranted by their constitution (けんぽう). \\
\end{itemize}
\item \ruby{恩}{おん} Moral indebtedness	
\begin{itemize}%\setlength\itemsep{-0.4em}
\item auf die frage warum man was macht(私は恩があります。ですからそうです。) \\
\end{itemize}
\item \ruby{義理}{ぎり} Moral and/or social obligation
\begin{itemize}%\setlength\itemsep{-0.4em}
\item 義理チョコ given on Feb.14(Valentines Day) to men by woman, social-obligation-chocolate. \\
\end{itemize}
\item \ruby{人情}{にんじょう} human kindness
\begin{itemize}%\setlength\itemsep{-0.4em}
\item ninjou ga atsui hito desu. --> a warm hearted person
\begin{itemize}%\setlength\itemsep{-0.4em}
\item japanese use thick and thin for describing a persons kindness.
\end{itemize}
\item \ruby{恥}{はじ} Gesichtslosigkeit (to loose face)\\ %TODO find better translation than gesichtlos
\end{itemize}
\item \ruby{根回し}{ねまわし} arrangement in advance, (originally root-wrapping)
\begin{itemize}%\setlength\itemsep{-0.4em}
\item ne-mawashi o shimasu = to prefere something the japanese people think preparing something properly allows it to be good. 
\item The japanese people do nemawashi especially before the discussion, they are so prepared that they have little to nothing left to discuss.
\end{itemize}
\begin{itemize}%\setlength\itemsep{-0.4em}
\item hon-ne literally: a real sound, meaning: a real feeling/opinion, private opinion
\item tatemae literally: a facade, meaning: public opinion \\
\end{itemize}
\item 年功序列 ねんこうじょれつ, system of age(seniority)	
\begin{itemize}%\setlength\itemsep{-0.4em}
\item The japanese people think this system allows them to avoid fights/discussions 
\item This system comes from growing rice as the seniors where very important in the rice farming society as the experience in rice growing was very important. \\
\end{itemize}
\item \ruby{村八分}{むらはちぶ} 
\begin{itemize}%\setlength\itemsep{-0.4em}
\item originally: no cooperation except for funerals and fires (かじ)--> traditional social punishment \\
\end{itemize}
\item \ruby{賄賂}{わいろ} bribe
\begin{itemize}%\setlength\itemsep{-0.4em}
\item \ruby{袖の下}{そでのした} under-the-sleeves --> meaning: money-under-the-sleeves \\
\end{itemize}
\item Gift-Giving
\begin{itemize}
\item \ruby{心行け}{こころずけ} kokoro: heart, zuke: attachement	--> token(money) of special thanks (in advance),	dozoyoroshiku-onegai-shimasu is said when given this present
\item \ruby{(ご)祝儀}{しゅぎ} --> token(money) as thanks afterwards, presented with the words: どもありがとうございました
\begin{itemize}%\setlength\itemsep{-0.4em}
\item ごしゅうぎ are used for weddings, there a envelope is used which is hard to open
\item ごしゅうぎ are also used for birthdays, there a easy to open reusable envelope is used \\
\end{itemize}
\end{itemize}
\item \ruby{お祝い}{おいわい} \tabni Oiwai --> Celebration
\begin{itemize}%\setlength\itemsep{-0.4em}
\item oiwai-no-okurimono, oiwai-no-shina --> celebration gifts
\begin{itemize}%\setlength\itemsep{-0.4em}
\item \ruby{中元}{ちゅうげん} \tabni Chuugen --> Bon Festival gifts
\begin{itemize}
\item Halb-Jahres-Geschenk am 15.Juli. (The present could be beer for example.)
\end{itemize}
\item \ruby{歳暮}{せいぼ} \tabni Seibo
%\begin{itemize}
%\item  
 --> Year-end gift, the present could be soya sauce for example.
%\end{itemize}
\end{itemize}
\item \ruby{お返し}{おかえし} O-kaeshi --> return --> important to give return gifts \\
\item \ruby{年賀状}{ねんがじょう} nenga-jo
\begin{itemize}%\setlength\itemsep{-0.4em}
\item new years greetings card --> very important for japanese --> should be written by hand
\begin{itemize}
\item as preparation japanese people should learn caligraphie
\item often hundreds of cards are sent by one person \\
\end{itemize} 
\end{itemize}
\end{itemize}
\item \ruby{名刺}{めいし} --> Business cards: 91mm x 55mm in size
\begin{itemize}%\setlength\itemsep{-0.4em}
\item exchanging business cards is an important part of japanese business
\item the younger person should initialize giving the card. 
\item The card is given with both hands and held so that the receiver can read the card immediately. 
\item Even students often have such cards. \\
\end{itemize}
\item \ruby{神道}{しんとう} \tabni Shinto
\begin{itemize}%\setlength\itemsep{-0.4em}
\item しんとう is based on nature, there are supposed to be 8 million gods(kami)
\item tenno was believed to be great great great grandson of amatersu was first tenno
\begin{itemize}%\setlength\itemsep{-0.4em}
\item all further tennos where decendants of him
\end{itemize}
\item jinja --> shrine, to properly worship gods offerings are very important
\item torii --> shrine gate, entrance to a sacred area \\
\end{itemize}
\item \ruby{神道}{しんとう} \tabni Bukkyou
\begin{itemize}
\item introduced through china-->korea--> to japan in the 6th centurary
\item buddhism is the path to enlightement
\item tera, o-tera, ji-in, bukkaku are interchangeable --> use o-tera, all mean buddhist-shrine
\item sanmon --> temple gate with 3 doorways \\
\end{itemize}

\item The Shinto-Buddhistic relationship
\begin{itemize}%\setlength\itemsep{-0.4em}
\item \ruby{神仏習合}{しんぶつしゅうごう} \tabni Shinbutsu-Shūgō, is the unity of shintoism und buddhism.
\begin{itemize}%\setlength\itemsep{-0.4em}
\item hotoke/butsu = buddha
\item shin = shinto
\item They believed that shinto gods where reincarations of buddhistic gods or the other way around.
\item There were shinto shrines in buddhistic temples and the other way around.
\item The difference between the two religions where unimportant for the average japanese.
\item This was kept until the meiji-restoration in the 19th century.
\item Until this point the effective ruler of the country had been the \ruby{将軍}{しょうぐん}(Shogun).
\item Shogun means general/military leader.
\item The Shoguns took over in the 12th century and ruled until the meiji restoration.
\end{itemize}
\item \ruby{神仏分離}{しんぶつぶんり} \tabni Shin-Butsu-Bunri, is the separation of shintoism and buddhism.
\begin{itemize}
\item Ordered in 1868 by the government to distinguish the shinto gods from the buddhistic ones. \\
\end{itemize}
\end{itemize}
%TODO main point
\item キリスト\ruby{教}{きょう} \tabni Kirisuto-Kyou --> christian religion
\begin{itemize}
\item From 1612 to 1873 anti-christian-edicts
\item The religion was introduced to japan around the end of 16th century.
\item Most important christian church(kyoukai) in Japan is the(Oura church) in Nagasaki build in 1864.
\item 1865 hidden christians appeared surprisingly in the church, these christians were called hidden christians, the christian faith had been hidden for over 250years from the beginning of the 17th century until then.
\item The church was destroyed through an atomic bomb in 1945, but rebuild after the war. \\
\end{itemize}
\item \ruby{右翼}{うよく} \tabni U-Yoku --> Right-Wing-People 
\begin{itemize}
\item Still see tenno as shinto-god, and think they sacrifice themselves for emperor and country.
\item They love fuji-san and sakana. \\
\end{itemize}
\item \ruby{右翼}{うよく} \tabni Kamikaze
\begin{itemize}
\item Kami: god/spirit,\tabni Kaze: wind --> thus: divine wind
\item Originally used to describe the storm who threw the mongols in the 13th century back into the sea. \item Also eventually used for suicide attacks in the 2nd worldwar. \\
\end{itemize}
\item \ruby{切腹}{せっぷく} \tabni Seppuku
\begin{itemize}
\item Act to forgive ones sin through suicide, seppuku is the cuting of the belly (disembowlement).
\item The japanese where not allowed to talk their own opinions to their superiors if they did anyways they would afterwards commit seppuku.
\end{itemize}
\end{enumerate}
\end{document}