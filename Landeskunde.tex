\documentclass{article}
\usepackage{xeCJK}
\usepackage{enumitem}
\usepackage{geometry}
\usepackage{titlesec}
\usepackage{lipsum}
\usepackage[CJK,overlap]{ruby}
 \geometry{
 a4paper,
 %total={170mm,257mm},
 left=10mm,
 top=12mm,
 right=10mm,
 bottom=12mm,
 }
 
\renewcommand{\rubysep}{0.1ex}%3.9ex}
\makeatletter
\renewcommand\paragraph{\@startsection{paragraph}{4}{\z@}%
            {-0.5ex\@plus -.1ex \@minus -.25ex}%
            {-1em}%
            {\normalfont\normalsize\bfseries}}
\renewcommand\subparagraph{\@startsection{subparagraph}{5}{\z@}%
            {-0.1ex\@plus -.1ex \@minus -.25ex}%
            {-1em}%
            {\normalfont\normalsize\bfseries}}
\makeatother

\setlist{nolistsep}
\titlespacing*{\section}
{0pt}{1.5ex plus 1ex minus .2ex}{1.3ex plus .2ex}
%{0pt}{5.5ex plus 1ex minus .2ex}{4.3ex plus .2ex}
\titlespacing*{\subsection}
{0pt}{0.5ex plus 1ex minus .2ex}{0.5ex plus .2ex}
\titlespacing*{\subsubsection}
{0pt}{0.1ex plus 1ex minus .2ex}{0.1ex plus .2ex}

\newcommand\tab[1][1cm]{\hspace*{#1}} 
\newcommand\tabi[1][0.05cm]{\hspace*{#1}}
\newcommand\tabni[1][0.2cm]{\hspace*{#1}}
\newcommand\tabyon[1][0.4cm]{\hspace*{#1}}
\newcommand\tabroku[1][0.6cm]{\hspace*{#1}}

\XeTeXlinebreaklocale "ja"
\XeTeXlinebreakskip=0em plus 0.1em minus 0.01em

\setlength{\parindent}{0pt}
\setCJKmainfont{ipaexm.ttf}
\setcounter{tocdepth}{5}

\begin{document}	%\large {1.5ex \@plus .2ex}%

\tableofcontents
\newpage

\part{\ruby{日本文化地理概論}{にほんぶんかちりがいろん}}
\section{\ruby{宗教}{しゅうきょう} \tabni religion | \ruby{社会思想}{しゃかいしそう} \tabni societal-thought }
\subsection{社会思想 \tabni societal-thought}
%\begin{enumerate}[itemsep=-6pt]\setlength\itemsep{-0.4em}
\subsubsection{Die großen Probleme der Bevölkerung:}
\begin{itemize}%\setlength\itemsep{-0.4em}
\item 失業 (しつぎょう)--> unemployment
\begin{itemize}
\item 失業りつ --> unemploment rate
\begin{itemize}%\setlength\itemsep{-0.4em}
\item りつ = rate
\end{itemize}
\item In Japan only qualified job-seekers are counted as unemployed.
\end{itemize}
\item 自殺 じさつ suicide
\begin{itemize}%\setlength\itemsep{-0.4em}
\item Korea is 1st in Suicides
\item Russian is 3rd in suicides
\item Japan is 4th in suicides \\
\end{itemize}
%\end{itemize}
\end{itemize}
%\begin{itemize}
%\subsubsection{\ruby{宗教}{しゅうきょう} \tabni religion} %IS ADDED BELOW!
%\begin{itemize}%\setlength\itemsep{-0.4em}
%\item Shinto has 51\% buddhism 42.7\%, and christians 1.5\% of the population as followers.
%\item People of Japan have the freedoms of religion and free speech. Guaranted by their constitution (けんぽう). \\
%\end{itemize}
\subsubsection{\ruby{恩}{おん} \tabni moral indebtedness || \ruby{義理}{ぎり} \tabni moral and/or social obligation}
\begin{itemize}%\setlength\itemsep{-0.4em}
\item auf die frage warum man was macht(私は恩があります。ですからそうです。) \\
\end{itemize}
%\subsubsection{}
\begin{itemize}%\setlength\itemsep{-0.4em}
\item 義理チョコ given on Feb.14(Valentines Day) to men by woman, social-obligation-chocolate. \\
\end{itemize}
\subsubsection{\ruby{人情}{にんじょう} \tabni human kindness}
\begin{itemize}%\setlength\itemsep{-0.4em}
\item ninjou ga atsui hito desu. --> a warm hearted person
\begin{itemize}%\setlength\itemsep{-0.4em}
\item japanese use thick and thin for describing a persons kindness.
\end{itemize}
\item \ruby{恥}{はじ} Gesichtslosigkeit (to loose face)\\ %TODO find better translation than gesichtlos
\end{itemize}
\subsubsection{\ruby{根回し}{ねまわし} \tabni arrangement in advance, (originally root-wrapping)}
\begin{itemize}%\setlength\itemsep{-0.4em}
\item ne-mawashi o shimasu = to prefere something the japanese people think preparing something properly allows it to be good. 
\item The japanese people do nemawashi especially before the discussion, they are so prepared that they have little to nothing left to discuss.
\end{itemize}
\begin{itemize}%\setlength\itemsep{-0.4em}
\item hon-ne literally: a real sound, meaning: a real feeling/opinion, private opinion
\item tatemae literally: a facade, meaning: public opinion \\
\end{itemize}
\subsubsection{\ruby{年功序}{ねんこうじょれつ} (ねんこうじょれつ) \tabni system of age(seniority)}
\begin{itemize}%\setlength\itemsep{-0.4em}
\item The japanese people think this system allows them to avoid fights/discussions 
\item This system comes from growing rice as the seniors where very important in the rice farming society as the experience in rice growing was very important. \\
\end{itemize}
\subsubsection{\ruby{村八分}{むらはちぶ}(むらはちぶ) \tabni societal ostrication}
\begin{itemize}%\setlength\itemsep{-0.4em}
\item originally: no cooperation except for funerals and fires (かじ)--> traditional social punishment \\
\end{itemize}
\subsubsection{\ruby{賄賂}{わいろ} \tabni bribe}
\begin{itemize}%\setlength\itemsep{-0.4em}
\item \ruby{袖の下}{そでのした} under-the-sleeves --> meaning: money-under-the-sleeves \\
\end{itemize}
\subsubsection{Gift-Giving}
\begin{itemize}
\item \ruby{心行け}{こころずけ} kokoro: heart, zuke: attachement	--> token(money) of special thanks (in advance),	dozoyoroshiku-onegai-shimasu is said when given this present
\item \ruby{(ご)祝儀}{しゅぎ} --> token(money) as thanks afterwards, presented with the words: どもありがとうございました
\begin{itemize}%\setlength\itemsep{-0.4em}
\item ごしゅうぎ are used for weddings, there a envelope is used which is hard to open
\item ごしゅうぎ are also used for birthdays, there a easy to open reusable envelope is used \\
\end{itemize}
\end{itemize}
\subsubsection{\ruby{お祝い}{おいわい} \tabni Oiwai --> Celebration}
\begin{itemize}%\setlength\itemsep{-0.4em}
\item oiwai-no-okurimono, oiwai-no-shina --> celebration gifts
\begin{itemize}%\setlength\itemsep{-0.4em}
\item \ruby{中元}{ちゅうげん} \tabni Chuugen --> Bon Festival gifts
\begin{itemize}
\item Halb-Jahres-Geschenk am 15.Juli. (The present could be beer for example.)
\end{itemize}
\item \ruby{歳暮}{せいぼ} \tabni Seibo
%\begin{itemize}
%\item  
 --> Year-end gift, the present could be soya sauce for example.
%\end{itemize}
\end{itemize}
\item \ruby{お返し}{おかえし} O-kaeshi --> return --> important to give return gifts \\
\item \ruby{年賀状}{ねんがじょう} nenga-jo
\begin{itemize}%\setlength\itemsep{-0.4em}
\item new years greetings card --> very important for japanese --> should be written by hand
\begin{itemize}
\item as preparation japanese people should learn caligraphie
\item often hundreds of cards are sent by one person \\
\end{itemize} 
\end{itemize}
\end{itemize}
\subsubsection{\ruby{名刺}{めいし} --> Business cards: 91mm x 55mm in size}
\begin{itemize}%\setlength\itemsep{-0.4em}
\item exchanging business cards is an important part of japanese business
\item the younger person should initialize giving the card. 
\item The card is given with both hands and held so that the receiver can read the card immediately. 
\item Even students often have such cards. \\
\end{itemize}
\subsubsection{\ruby{右翼}{うよく} \tabni U-Yoku --> Right-Wing-People}
\begin{itemize}
\item Still see tenno as shinto-god, and think they sacrifice themselves for emperor and country.
\item They love fuji-san and sakana. \\
\end{itemize}
\subsubsection{\ruby{右翼}{うよく} \tabni Kamikaze}
\begin{itemize}
\item Kami: god/spirit,\tabni Kaze: wind --> thus: divine wind
\item Originally used to describe the storm who threw the mongols in the 13th century back into the sea. \item Also eventually used for suicide attacks in the 2nd worldwar. \\
\end{itemize}
\subsubsection{\ruby{切腹}{せっぷく} \tabni Seppuku}
\begin{itemize}
\item Act to forgive ones sin through suicide, seppuku is the cuting of the belly (disembowlement).
\item The japanese where not allowed to talk their own opinions to their superiors if they did anyways they would afterwards commit seppuku.
\end{itemize}
\subsection{宗教 \tabni religion}
\begin{itemize}%\setlength\itemsep{-0.4em}
\item Shinto has 51\% buddhism 42.7\%, and christians 1.5\% of the population as followers.
\item People of Japan have the freedoms of religion and free speech. Guaranted by their constitution (けんぽう). \\
\end{itemize}
\begin{enumerate}
%\subsubsection{\ruby{神道}{しんとう} \tabni Shinto}
\item \ruby{神道}{しんとう} \tabni Shinto
\begin{itemize}%\setlength\itemsep{-0.4em}
\item しんとう is based on nature, there are supposed to be 8 million gods(kami)
\item tenno was believed to be great great great grandson of amatersu was first tenno
\begin{itemize}%\setlength\itemsep{-0.4em}
\item all further tennos where decendants of him
\end{itemize}
\item jinja --> shrine, to properly worship gods offerings are very important
\item torii --> shrine gate, entrance to a sacred area \\
\end{itemize}
%\subsubsection{\ruby{神道}{しんとう} \tabni Bukkyou}
\item \ruby{神道}{しんとう} \tabni Bukkyou
\begin{itemize}
\item introduced through china-->korea--> to japan in the 6th centurary
\item buddhism is the path to enlightement
\item tera, o-tera, ji-in, bukkaku are interchangeable --> use o-tera, all mean buddhist-shrine
\item sanmon --> temple gate with 3 doorways \\
\end{itemize}

%\subsubsection{The Shinto-Buddhistic relationship}
\item The Shinto-Buddhistic Relationship
\begin{itemize}%\setlength\itemsep{-0.4em}
\item \ruby{神仏習合}{しんぶつしゅうごう} \tabni Shinbutsu-Shūgō, is the unity of shintoism und buddhism.
\begin{itemize}%\setlength\itemsep{-0.4em}
\item hotoke/butsu = buddha
\item shin = shinto
\item They believed that shinto gods where reincarations of buddhistic gods or the other way around.
\item There were shinto shrines in buddhistic temples and the other way around.
\item The difference between the two religions where unimportant for the average japanese.
\item This was kept until the meiji-restoration in the 19th century.
\item Until this point the effective ruler of the country had been the \ruby{将軍}{しょうぐん}(Shogun).
\item Shogun means general/military leader.
\item The Shoguns took over in the 12th century and ruled until the meiji restoration.
\end{itemize}
\item \ruby{神仏分離}{しんぶつぶんり} \tabni Shin-Butsu-Bunri, is the separation of shintoism and buddhism.
\begin{itemize}
\item Ordered in 1868 by the government to distinguish the shinto gods from the buddhistic ones. \\
\end{itemize}
\end{itemize}
%\subsubsection{キリスト\ruby{教}{きょう} \tabni Kirisuto-Kyou --> christian religion}
\item キリスト\ruby{教}{きょう} \tabni Kirisuto-Kyou --> christian religion
\begin{itemize}
\item From 1612 to 1873 anti-christian-edicts
\item The religion was introduced to japan around the end of 16th century.
\item Most important christian church(kyoukai) in Japan is the(Oura church) in Nagasaki build in 1864.
\item 1865 hidden christians appeared surprisingly in the church, these christians were called hidden christians, the christian faith had been hidden for over 250years from the beginning of the 17th century until then.
\item The church was destroyed through an atomic bomb in 1945, but rebuild after the war. \\
\end{itemize}
\end{enumerate}

%\end{enumerate}
\section{\ruby{教育}{きょういく} \tabni education}
\subsection{\ruby{文部科学省}{もんぶかがくしょう} ministry of education}
\begin{itemize}
\item monbu = culture,	kagaku = science, sho = ministy
\item Also is concerned with matters of sport such as illegal betting on: soccer, baseball, sumo.
\end{itemize}
\subsection{Gakko-Seido}
\begin{itemize}
\item Education-level
\item Family financial Background
\end{itemize}
\subsubsection{\ruby{義務教育}{ぎむきょういく} Gimu-Kyoiku}
\begin{itemize}
\item compulsary-education for 9 years
\end{itemize}
\begin{enumerate}
\item Sho-gakku --> elementary school
\begin{itemize}
\item students are 6-years old when joining
\item no tuiton for public elementary schools, around 5000-euros for private schools
\end{itemize}
\item Chuu-gakko --> middle school
\begin{itemize}
\item students are 12-years old when joining
\item private schools are expensive but highly sought after, as they are a middle and highschool in one and thus do not require highschool entrance exams. Private schools also tend to offer better preparation for university entrance exams.
\end{itemize}
\end{enumerate}
\subsubsection{Higher Education}
\begin{enumerate}
\item \ruby{高等学校}{こうとうがっこう} --> high school
\begin{itemize}
\item short: 高校 (こうこう)
\item the average school day lasts 5-8 hours
\item students are enrolled between the ages of 15-17
\item students who are older than 17 have a problem fitting in as they will be talked to by everyone with a certain distance, as japanese students are often incapable of speaking plain style to someone their senior, thus excluding said person from the group.
\item The quality of the "senior"-high-school is very important for any future japanese career.
\end{itemize}
\item \ruby{大学}{だいがく} --> high school
\begin{itemize}
\item Students are aged 18+
\item Bachelor 4-years, master 2-years, doctor 3-years
\item There are 100.000 foreign students in Japanese universities of those are 90\% asian, 60\% even chinese.
\item They are differentiated between the \ruby{公立大学}{こうりつだいがく} general public universities, and special public universities so called \ruby{国立大学}{こくりつだいがく} national universities.
\end{itemize}
\end{enumerate}
\subsubsection{\ruby{帝国大学}{ていこくだいがく} The 7 national universities or (former) imperial universities }
\begin{itemize}
\item Also formerly know as \ruby{旧帝大}{きゅうていだい} (The 9 Imperial Universities.)
\item They were founded under the imperial government between 1886 and 1939. \\%<--Wiki %Sotos numbers: 1877 and 1936.
\begin{tabular}{ l l p{14cm} }
Hokkaido &in Sapporo & 18k(20k)students || Vice president  (and founder of predecessor Sapporo Agricultural College) was William Smith Clark, as such the university caused christian scholars to appear (students became christians)\\
Tohoku &in Sendai& 18k students || first to accept female students in 1913\\
Tokyo &in Tokyo& 30k students || oldest and most prestigious university of Japan || called 'University of Tokyo' || many japanese politicians studied here \\
Nagoya &Nagoya& 17k students\\
Kyoto &in Kyoto& 23k students || 2nd university of japan founded 1897 and build with compensation money from the sino-japanese war 1894-1895 || organized as german-styled university which meant students had to learn german\\
\end{tabular}
\begin{tabular}{ l l p{14cm} }
Osaka &in Osaka(Suita City)& 23k students || started as dutch learning school\\
Kyuushuu &in Fukuoka& originally established 1867 as medical school\\
Keijo &in Seoul& \\
Taipei &in Taiwan & \\
\end{tabular}
\item Other Universities
\begin{itemize}
\item Waseda University: 54k students, and one of the most prestigious universities in japan
\item Keio University: 34k students, originally for oranda-gaku, now also one of the most prestigious universities of japan
\item \ruby{上智大学}{じょうち} Jōchi Daigaku --> Sophia University, 12k students 
\item Würzburg University, 32k students
\end{itemize}
\item \ruby{進学率}{しんがくりつ} \tabi university-enrollment-rate \tabroku Japan:50.4\% \tab Deutschland: 35\%
\item \ruby{塾}{じゅく} \tabi --> cram school
\item \ruby{家庭教師}{かていきょうし} \tabi --> private tutor
\item \ruby{浪人}{ろうにん} \tabi originally: wandering samurai without a master to serve, now: waiting for another chance to enter a university
\item \ruby{予備校}{よびこう} prep school for students in der ronin year
\end{itemize}
\subsubsection{\ruby{教育費}{きょういくひ} \tabi educational expenses}
\begin{itemize}
\item 5.280.000 yen to educate a child in japan (117yen = 1euro) thus around 45.000 euros
\item \ruby{授業料}{じゅぎょうりょう}: tuition fees: national: 535.800yen \tabi public:537.857yen \tabi private:742.478yen
\item national--> 4579euro semester, 170euro monthly personal expenses  
\end{itemize}
\ruby {同窓生}{どうそうせい} \tabi ALUMNI
\begin{itemize}
\item important for japanese worklife, a good one is needed, they are depending on ones highschool
\end{itemize}
\end{document}
