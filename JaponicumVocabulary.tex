\documentclass{article}
\usepackage{xeCJK}
\usepackage{enumitem}
\usepackage{titlesec}
\usepackage[dvipsnames]{xcolor}
\usepackage{geometry}
\usepackage[CJK,overlap]{ruby}
 \geometry{
 a4paper,
 %total={170mm,257mm},
 left=10mm,
 top=12mm,
 right=10mm,
 bottom=12mm,
 }
 
\renewcommand{\rubysep}{0.1ex}%3.9ex}
\makeatletter
\renewcommand\paragraph{\@startsection{paragraph}{4}{\z@}%
            {-0.5ex\@plus -.1ex \@minus -.25ex}%
            {-1em}%
            {\normalfont\normalsize\bfseries}}
\renewcommand\subparagraph{\@startsection{subparagraph}{5}{\z@}%
            {-0.1ex\@plus -.1ex \@minus -.25ex}%
            {-1em}%
            {\normalfont\normalsize\bfseries}}
\makeatother

\newcommand\tab[1][1cm]{\hspace*{#1}} 
\newcommand\tabi[1][0.05cm]{\hspace*{#1}}
\newcommand\tabni[1][0.2cm]{\hspace*{#1}}
\newcommand\tabyon[1][0.4cm]{\hspace*{#1}}
\newcommand\tabroku[1][0.6cm]{\hspace*{#1}}

\XeTeXlinebreaklocale "ja"
\XeTeXlinebreakskip=0em plus 0.1em minus 0.01em

\setlist{nolistsep}
\titlespacing*{\section}
{0pt}{1.5ex plus 1ex minus .2ex}{1.5ex plus .2ex}
\titlespacing*{\subsection}
{0pt}{0.5ex plus 1ex minus .2ex}{0.5ex plus .2ex}

\setlength{\parindent}{0pt}
\setCJKmainfont{ipaexm.ttf}
\setcounter{tocdepth}{5}

\begin{document}
\part*{New Vocabulary}
\begin{tabular}{ p{3cm} | l l }
Kanji&Kana&Translation\\ \hline\\[-1em]
覚えています&おぼえています&--> to remember \\ \hline\\[-1em]
覚える &おぼえる&--> to memorize \\ \hline\\[-1em]
忘れました &わすれました&--> to have forgotten \\ \hline\\[-1em]
忘れる &わすれる&--> to forget \\ \hline\\[-1em]
与党 &よとう&--> ruling party \\ \hline\\[-1em]
野党 &やとう&--> opposition party \\ \hline\\[-1em]
嬉しい      &うれしい           &--> happy (1st/2nd person) \\ \hline\\[-1em]
喜ぶ        &よろこぶ           &--> to be delighted (3rd person) /to be glad \\ \hline\\[-1em]
悲しい      &かなしい           &--> sad /sorrowful (1st/2nd person) \\ \hline\\[-1em]
悲しむ      &かなしむ           &--> to be sad (3rd person) /to mourn for /to regret \\ \hline\\[-1em]
欲しい      &ほしい             &--> wanted /wished for /in need of /desired \\ \hline\\[-1em]
欲しがる    &ほしがる           &--> to desire /to want /to wish for /to covet \\ \hline\\[-1em]
怖い        &こわい             &--> scary /frightening /eerie /dreadful \\ \hline\\[-1em]
寂しい      &さびしい           &--> lonely /lonesome /solitary /desolate \\ \hline\\[-1em]
恥ずかしい  &はずかしい         &--> shy /ashamed /embarrassed \\ \hline\\[-1em]
羨ましい    &うらやましい       &--> envious /jealous \\ \hline\\[-1em]
必要&ひつよう&--> to need \\ \hline\\[-1em]
祖父&そふ&--> grandfather \\ \hline\\[-1em]
祖母&そぼ&--> grandmother \\ \hline\\[-1em]
伯父&おじ&--> uncle \\ \hline\\[-1em]
伯母&おば&--> aunt \\ \hline\\[-1em]
従兄弟/従姉妹&いとこ&--> cousin \\ \hline\\[-1em]
ご主人&ごしゅじん&--> someone else's husband \\ \hline\\[-1em]
主人&しゅじん&--> one's husband \\ \hline\\[-1em]
奥さん&おくさん&--> someone else's wife \\ \hline\\[-1em]
家内&かない&--> one's wife \\ \hline\\[-1em]
結婚(を)する&けっこん(を)する&--> to get married \\ \hline\\[-1em]
結婚(を)している&けっこん(を)している&--> to be married \\ \hline\\[-1em]
離婚(を)する&りこん(を)する&--> to divorce \\ \hline\\[-1em]
独身&どくしん&--> single \\ \hline\\[-1em] 
一人っ子&ひとりこ&--> only child \\ \hline\\[-1em]
双子&ふたご&--> twins \\ \hline\\[-1em]
習う&ならう&--> to take lessons in /to be taught /to learn \\ \hline\\[-1em]
何回&なんかい&--> how many times? \\ \hline\\[-1em]
何度&なんど&--> how many times? /how often? %\\ \hline\\[-1em]
\end{tabular}

\newpage
\part*{Vocabulary that we already had}
\begin{tabular}{ p{3cm} | l l }
親戚&しんせき&--> relatives \\ \hline\\[-1em]
兄弟&きょうだい&--> siblings \\ \hline\\[-1em]
兄&あに&--> older brother \\ \hline\\[-1em]
姉&あね&--> older sister \\ \hline\\[-1em]
家族&かぞく&--> family \\ \hline\\[-1em]
両親&りょうしん&--> parents \\ \hline\\[-1em]
何時&いつ&--> when /how soon \\ \hline\\[-1em]
何処&どこ&--> where /what place \\ \hline\\[-1em]
何&なに&--> what \\ \hline\\[-1em]
誰&だれ&--> who %\\ \hline\\[-1em]
\end{tabular}

\part*{カタカナ}
\begin{tabular}{ p{3cm} | l l }
ハズバンド&husband \\ \hline\\[-1em]
ウィーン&Wien(Vienna) \\ \hline\\[-1em]
ジュネーブ&Geneve(Genf)
\end{tabular}
%\footnotetext{する-verb}
\end{document}
