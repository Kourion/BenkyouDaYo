\documentclass{article}
\usepackage{xeCJK}
\usepackage{geometry}
 \geometry{
 a4paper,
 %total={170mm,257mm},
 left=20mm,
 top=20mm,
 }
\setCJKmainfont{ipaexm.ttf}
\begin{document} \large

\section{地理 (ちり) geography	| 自然 (しぜん) nature}
\section{政治 (せいじ) politics}
\section{外交 (がいこう) diplomacy}
%\begin{center}
    \begin{tabular}{  l | l | p{5cm} }
    %\hline				& & \\ \hline
 Kanji 			& Hiragana 			& Romaji \\[5pt] \hline	\\[-1em]
 外交 			& がいこう 				& diplomacy   \\ \hline	\\[-1em]
 外務省			& がいむしょう			& Ministry of Foreign Affairs \\ \hline \\[-1em]
 国交問題			& こっこうもんだい			& diplomatic relations problem \\ \hline \\[-1em]
 サンフランシスコ講和条約  & SFRこうわじょうやく & San Francisco Treaty Sept. 8, 1951)\\ \hline \\[-1em]
 日華平和条約		& にっかへいわじょうやく  & peace treaty between CHI and JAP 1978\\ \hline \\[-1em]
 日ソ共同宣言				& にっきょうどうせんげん 	& JAP-SOV joint declaration \\ \hline \\[-1em]
 日韓基本条約				& にっかんきほんじょうやく	& JAP-SKOR basic treaty\\ \hline \\[-1em]
 日中共同声明				& にちゅきょうどうせいめい	& CHI-JAP joint declaration \\ \hline \\[-1em]
 平和条約					& へいわじょうやく 		& peace treaty \\ \hline \\[-1em]
 領土問題					& りょうどもんだい			& territorial dispute \\ \hline \\[-1em]
 ロシア					& ろしあ				& RUS	\\ \hline \\[-1em]
 北方領土					& ほっぽうりょうど 		& the Northern Territories\\ \hline \\[-1em]
 日露和親条約				& にちろわしんじょうやく		& RUS-JAP peace treaty\\ \hline \\[-1em]
 中国					& ちゅうごく				& CHI \\ \hline \\[-1em]
 尖閣諸島					& せんかくしゅとう			& Senkaku Islands \\ \hline \\[-1em]
 台湾					& たいわん				& Taiwan \\ \hline \\[-1em]
 韓国					& かんこく 				& South Korea\\ \hline \\[-1em]
 竹島					& たけしま				& Liancourt Rocks \\ \hline \\[-1em]
 慰安婦問題				& いあんふもんだい		& comfort women problem\\ \hline \\[-1em]
 北朝鮮					& きたちょうせん			& North Korea \\ \hline \\[-1em]
 防衛問題					& ぼうえいもんだい		& Defence Problem \\ \hline \\[-1em]
 アメリカ					& あめりか				& USA	 \\ \hline \\[-1em]
 貿易問題					& ぼうえきもんだい		& international trade problem\\ \hline \\[-1em]
 ODA				&せいふかいはつえんじょ  &Official Development Assistance \\ \hline \\[-1em]
 国際連合					& こくさいれんごう 		& United Nations \\ \hline \\[-1em]
    \end{tabular}
%\end{center}
\end{document}