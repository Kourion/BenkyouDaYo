\documentclass{article}
\usepackage{xeCJK}
\usepackage[CJK,overlap]{ruby}
\usepackage{geometry}
 \geometry{
 a4paper,
 %total={170mm,257mm},
 left=10mm,
 top=12mm,
 right=10mm,
 bottom=12mm,
 }
\setCJKmainfont{ipaexm.ttf}
\renewcommand{\rubysep}{0.1ex}%3.9ex}
\newcommand\tabni[1][0.2cm]{\hspace*{#1}}
\begin{document} \large

\section{地理 (ちり) geography	| 自然 (しぜん) nature}
\section{政治 (せいじ) politics}
\section{外交 (がいこう) diplomacy}
%\begin{center}
    \begin{tabular}{  l | l | p{7.5cm} }
    %\hline				& & \\ \hline
 Kanji 			& Hiragana 			& Translation \\[5pt] \hline	\\[-1em]
 外交 			& がいこう 				& diplomacy   \\ \hline	\\[-1em]
 外務省			& がいむしょう			& Ministry of Foreign Affairs \\ \hline \\[-1em]
 国交問題			& こっこうもんだい			& diplomatic relations problem \\ \hline \\[-1em]
 サンフランシスコ講和条約  & SFRこうわじょうやく & San Francisco Treaty Sept. 8, 1951)\\ \hline \\[-1em]
 日華平和条約		& にっかへいわじょうやく  & peace treaty between CHI and JAP 1978\\ \hline \\[-1em]
 日ソ共同宣言				& にっきょうどうせんげん 	& JAP-SOV joint declaration \\ \hline \\[-1em]
 日韓基本条約				& にっかんきほんじょうやく	& JAP-SKOR basic treaty\\ \hline \\[-1em]
 日中共同声明				& にちゅきょうどうせいめい	& CHI-JAP joint declaration \\ \hline \\[-1em]
 平和条約					& へいわじょうやく 		& peace treaty \\ \hline \\[-1em]
 領土問題					& りょうどもんだい			& territorial dispute \\ \hline \\[-1em]
 ロシア					& ろしあ				& RUS	\\ \hline \\[-1em]
 北方領土					& ほっぽうりょうど 		& the Northern Territories\\ \hline \\[-1em]
 日露和親条約				& にちろわしんじょうやく		& RUS-JAP peace treaty\\ \hline \\[-1em]
 中国					& ちゅうごく				& CHI \\ \hline \\[-1em]
 尖閣諸島					& せんかくしゅとう			& Senkaku Islands \\ \hline \\[-1em]
 台湾					& たいわん				& Taiwan \\ \hline \\[-1em]
 韓国					& かんこく 				& South Korea\\ \hline \\[-1em]
 竹島					& たけしま				& Liancourt Rocks \\ \hline \\[-1em]
 慰安婦問題				& いあんふもんだい		& comfort women problem\\ \hline \\[-1em]
 北朝鮮					& きたちょうせん			& North Korea \\ \hline \\[-1em]
 防衛問題					& ぼうえいもんだい		& Defence Problem \\ \hline \\[-1em]
 アメリカ					& あめりか				& USA	 \\ \hline \\[-1em]
 貿易問題					& ぼうえきもんだい		& international trade problem\\ \hline \\[-1em]
 ODA				& せいふかいはつえんじょ  &Official Development Assistance \\ \hline \\[-1em]
 国際連合					& こくさいれんごう 		& United Nations \\ \hline \\[-1em]
    \end{tabular}
%\end{center}
\section{\ruby{宗教}{しゅうきょう} \tabni religion | \ruby{社会思想}{しゃかいしそう} \tabni societal-thought }
\begin{tabular}{ l | l | p{7.5cm} }
\ruby{社会}{しゃかい} & しゃかい & society\\ \hline \\[-1em]
宗教 & しゅうきょう & religion\\ \hline \\[-1em]
思想 & しそう & thought; idea; ideology\\ \hline \\[-1em]
恩 & おん & favour\\ \hline \\[-1em]
義理	& ぎり & duty; honour\\ \hline \\[-1em]
義理チョコ	& ぎりちょこ & obligatory-gift chocolate\\ \hline \\[-1em]
人情	& にんじょう & humanity\\ \hline \\[-1em]
恥 & はじ & shame\\ \hline \\[-1em]
根回し & ねまわし & making necessary arrangements\\ \hline \\[-1em]
本音	& ほんね & real intention; motive\\ \hline \\[-1em]
建前	& たてまえ & face; official stance\\ \hline \\[-1em]
年功序列 & ねんこうじょれつ & seniority by length of service\\ \hline \\[-1em]
村八分 & むらはちぶ & ostracism\\ \hline \\[-1em]
賄賂、袖の下 & わいろ, そでのした & bribe\\ \hline \\[-1em]
心付け & こころづ & gratuity; tip\\ \hline \\[-1em]
祝儀、ご祝儀 & しゅうぎ, ごしゅうぎ & congratulations\\ \hline \\[-1em]
お祝い & おいわい & congratulation\\ \hline \\[-1em]
お祝いの贈り物 & おいわいのおくりもの & congratulation-present\\ \hline \\[-1em]
、お祝いの品 & おいわい & congratulation-elegance\\ \hline \\[-1em]
中元、お中元 & ちゅうげん, おなかもと & Bon Festival, Bon Festival gifts\\ \hline \\[-1em]
(お)歳暮 & (お)せいぼ & end of the year, year-end gift\\ \hline \\[-1em]
お返し & おかえし & return gift\\ \hline \\[-1em]
年賀状 & ねんがじょう & New Year's card\\ \hline \\[-1em]
名刺	& めいし & business card \\ \hline \\[-1em]
神道 & しんとう & Shinto\\ \hline \\[-1em]
神社 & じんじゃ & Shinto shrine\\ \hline \\[-1em]
鳥居 & とりい & Shinto shrine archway\\ \hline \\[-1em]
仏教 & ぶっきょう & Buddhism\\ \hline \\[-1em]
(お)寺、寺院、仏閣	& (お)てら, じいん, ぶっかく & temple\\ \hline \\[-1em]
三門	& さんもん & large triple gate to temple\\ \hline \\[-1em]
神仏習合	& しんぶつしゅうごう & syncretism of Shinto and Buddhism\\ \hline \\[-1em]
神仏分離	& しんぶつぶんり & separation of Buddhism and Shintoism\\ \hline \\[-1em]
キリスト教 & キリストきょう & Christianity\\ \hline \\[-1em]
教会 & きょうかい & church\\ \hline \\[-1em]
右翼	& うよく & right-wing (politics)\\ \hline \\[-1em]
神風	& かみかぜ & divine wind\\ \hline \\[-1em]
切腹 & せっぷく & ritual suicide\\ \hline \\[-1em]
\end{tabular}
\end{document}