\documentclass{article}
\usepackage{xeCJK}
\usepackage[CJK,overlap]{ruby}
\usepackage{geometry}
 \geometry{
 a4paper,
 %total={170mm,257mm},
 left=10mm,
 top=6mm,
 right=10mm,
 bottom=12mm,
 }
\setCJKmainfont{ipaexm.ttf}
\renewcommand{\rubysep}{0.1ex}%3.9ex}
\newcommand\tabni[1][0.2cm]{\hspace*{#1}}
\begin{document} \large
\section{地理 (ちり) geography	| 自然 (しぜん) nature}
\begin{tabular}{ l | l | p{7.5cm} }
\ruby{地理}{ちり} &  ちり  & geography \\ \hline \\[-1em]
\ruby{自然}{しぜん} &  しぜん & nature \\ \hline \\[-1em]
\ruby{気候}{ きこう } &  きこう  & climate \\ \hline \\[-1em]
\ruby{琵琶湖}{ びわこ } & びわこ/びわみずうみ & lake biwa(ko)   \\ \hline \\[-1em]
\ruby{北海道}{ほっかいどう} & ほっかいどう & Hokkaido  \\ \hline \\[-1em]
\ruby{本州}{ ほんしゅう } &  ほんしゅう  & Honshuu  \\ \hline \\[-1em]
\ruby{四国}{ しこく } &  しこく & Shikoku  \\ \hline \\[-1em]
\ruby{九州}{ きゅうしゅう } &  きゅうしゅう  & Kyushu  \\ \hline \\[-1em]
\ruby{沖縄}{ おきなわ } &  おきなわ  & Okinawa  \\ \hline \\[-1em]
\ruby{都道府県}{ とどうふけん } &  とどうふけん  & administrative divisions of Japan \\ \hline \\[-1em]
\ruby{札幌}{ さっぽろ } &  さっぽろ & Sapporo  \\ \hline \\[-1em]
\ruby{仙台}{ せんだい } &  せんだい  & Sendai  \\ \hline \\[-1em]
\ruby{東京}{ とうきょう } &  とうきょう  & Tokyo  \\ \hline \\[-1em]
\ruby{名古屋}{ なごや } &  なごや & Nagoya  \\ \hline \\[-1em]
\ruby{京都}{ きょうと } &  きょうと  & Kyoto  \\ \hline \\[-1em]
\ruby{大阪}{ おおさか } &  おおさか  & Osaka  \\ \hline \\[-1em]
\ruby{福岡}{ ふくおか } &  ふくおか  & Fukuoka  \\ \hline \\[-1em]
\ruby{広島}{ ひろしま } &  ひろしま  & Hiroshima  \\ \hline \\[-1em]
\ruby{長崎}{ ながさき } &  ながさき  & Nagasaki  \\ \hline \\[-1em]
\ruby{福島}{ふくしま} & ふくしま &Fukushima  \\ \hline \\[-1em]
\ruby{人口}{ じんこう } &  じんこう  & population  \\ \hline \\[-1em]
\ruby{島}{しま} & しま & island  \\ \hline \\[-1em]
\ruby{富士山}{ ふじさん} & ふじさん & Fuji-San  \\ \hline \\[-1em]
\ruby{火山}{ かざん } &  かざん & volcano  \\ \hline \\[-1em]
\ruby{温泉}{ おんせん } &  おんせん  & onsen  \\ \hline \\[-1em]
\ruby{地震}{ じしん } &  じしん  & earthquake  \\ \hline \\[-1em]
\ruby{津波}{ つなみ } &  つなみ  & tsunami  \\ \hline \\[-1em]
\ruby{気温}{ きおん } &  きおん & atmospheric temperature  \\ \hline \\[-1em]
\ruby{梅雨}{ばいう} & ばいう/つゆ & rainy season  \\ \hline \\[-1em]
\ruby{台風}{ たいふう } &  たいふう  & typhoon  \\ \hline \\[-1em]
\ruby{動物}{ どうぶつ } &  どうぶつ  & animal  \\ \hline \\[-1em]
\ruby{鳥}{とり} & とり & bird   \\ \hline \\[-1em]
\ruby{植物}{ しょくぶつ } &  しょくぶつ  & plant /vegetation   %\\ \hline \\[-1em]
\end{tabular}
\section{政治 (せいじ) politics}
\begin{tabular}{ l | l | p{7.5cm} }
憲法 & けんぽう & constitution \\ \hline \\[-1em]
天皇 & てんのう & Emperor of Japan \\ \hline \\[-1em]
政治 & せいじ & politics; government \\ \hline \\[-1em]
三権分立 & さんけんぶんりつ & separation of powers \\ \hline \\[-1em]
立法 & りっぽう & legislation \\ \hline \\[-1em]
国会 & こっかい & National Diet; parliament; congress \\ \hline \\[-1em]
衆議院 & しゅうぎいん & lower house; House of Representatives \\ \hline \\[-1em]
参議院 & さんぎいん & House of Councillors \\ \hline \\[-1em]
選挙	& せんきょ	& election \\ \hline \\[-1em]
選挙権 & せんきょけん & suffrage; the right to vote \\ \hline \\[-1em]
政党 & せいとう & political party \\ \hline \\[-1em]
自由民主党 & じゆうみんしゅとう & Liberal Democratic Party; LDP\\ \hline \\[-1em]
公明党 & こうめいとう & Komeito /Justice Party \\ \hline \\[-1em]
民進党 & みんしんとう	& Democratic Party \\ \hline \\[-1em]
社会民主党 & しゃかいみんしゅとう & Social Democratic Party \\ \hline \\[-1em]
日本共産党 & にほんきょうさんとう & Japanese Communist Party \\ \hline \\[-1em]
ドイツキリスト\ruby{教民主同盟}{きょうみんしゅどうめい} &  & (CDU) \\ \hline \\[-1em]
ドイツ\ruby{社会民主党}{しゃかいみんしゅとう} &  & (SPD) \\ \hline \\[-1em]
バイエルン &  & Bayern \\ \hline \\[-1em]
キリスト\ruby{教社会同盟}{きょうしゃかいどうめい} &  & (CSU) \\ \hline \\[-1em]
\ruby{同盟90/緑の党}{どうめいきゅうぜろ-みどりのとう} &  & (Bündnis 90 die Grünen) \\ \hline \\[-1em]
\ruby{左翼党}{さよくとう} & & Die Linke \\ \hline \\[-1em]
\ruby{自由民主党}{じゆうみんしゅとう} &  & Liberal Democratic Party; LDP\\ \hline \\[-1em]
行政 & ぎょうせい & administration / executive \\ \hline \\[-1em]
内閣	& ないかく & cabinet(government) \\ \hline \\[-1em]
内閣総理大臣 & ないかくそうりだいじん & primeminister \\ \hline \\[-1em]
(首相) & しゅしょう & chancellor \\ \hline \\[-1em]
大統領 & だいとうりょう & president \\ \hline \\[-1em]
司法 & しほう & administration of justice; Judiciary \\ \hline \\[-1em]
最高裁判所 & さいこうさいばんしょ & Supreme Court \\ \hline \\[-1em]
裁判所 & さいばんしょ & court; courthouse \\ \hline \\[-1em]
天皇明仁 & てんのうあきひと & Akihito Emperor of Japan \\ \hline \\[-1em]
今上天皇 & きんじょうてんのう & the present Emperor \\ \hline \\[-1em]
皇后 & こうごう & (Japanese) empress; queen \\ \hline \\[-1em]
陛下 & へいか & your Majesty; his Majesty; her Majesty \\ \hline \\[-1em]
皇太子徳仁親王 & こうたいしなるひと & Crown Princess Masako \\ \hline \\[-1em]
秋篠宮文仁 & あきしののみやふみひと & Akishino-no-miya Fumihito \\ \hline \\[-1em]
親王 & しんのう & prince of royal blood; Imperial prince \\ \hline \\[-1em]
悠仁親王 & はるかひとししんのう & Prince Hisahito of Akishino \\ \hline \\[-1em]
殿下 & でんか & your/his/her Highness \\ \hline \\[-1em]
宮内庁 & くないちょう & Imperial Household Agency %\\ \hline \\[-1em]
\end{tabular}
\section{外交 (がいこう) diplomacy}
%\begin{center}
\begin{tabular}{  l | l | p{7.5cm} }
%Kanji 					& Hiragana 			& Translation 					\\[5pt] \hline	\\[-1em]
 外交 					& がいこう 				& diplomacy   							\\ \hline \\[-1em]
 外務省					& がいむしょう			& Ministry of Foreign Affairs 			\\ \hline \\[-1em]
 国交問題					& こっこうもんだい			& diplomatic relations problem 			\\ \hline \\[-1em]
 サンフランシスコ講和条約 		& SFRこうわじょうやく 		& San Francisco Treaty (Sept. 8, 1951)	\\ \hline \\[-1em]
 日華平和条約				& にっかへいわじょうやく  	& peace treaty between CHI and JAP 1978	\\ \hline \\[-1em]
 日ソ共同宣言				& にっきょうどうせんげん 	& JAP-SOV joint declaration 			\\ \hline \\[-1em]
 日韓基本条約				& にっかんきほんじょうやく	& JAP-SKOR basic treaty					\\ \hline \\[-1em]
 日中共同声明				& にちゅきょうどうせいめい	& CHI-JAP joint declaration 			\\ \hline \\[-1em]
 平和条約					& へいわじょうやく 		& peace treaty 							\\ \hline \\[-1em]
 領土問題					& りょうどもんだい			& territorial dispute 					\\ \hline \\[-1em]
 ロシア					& ろしあ				& RUS									\\ \hline \\[-1em]
 北方領土					& ほっぽうりょうど 		& the northern territories				\\ \hline \\[-1em]
 日露和親条約				& にちろわしんじょうやく		& RUS-JAP peace treaty					\\ \hline \\[-1em]
 中国					& ちゅうごく				& CHI 									\\ \hline \\[-1em]
 尖閣諸島					& せんかくしょとう			& senkaku islands 						\\ \hline \\[-1em]
 台湾					& たいわん				& Taiwan 								\\ \hline \\[-1em]
 韓国					& かんこく 				& South Korea							\\ \hline \\[-1em]
 竹島					& たけしま				& Liancourt Rocks 						\\ \hline \\[-1em]
 慰安婦問題				& いあんふもんだい		& comfort women problem					\\ \hline \\[-1em]
 北朝鮮					& きたちょうせん			& North Korea 							\\ \hline \\[-1em]
 防衛問題					& ぼうえいもんだい		& Defence Problem						\\ \hline \\[-1em]
 アメリカ					& あめりか				& USA	 								\\ \hline \\[-1em]
 貿易問題					& ぼうえきもんだい		& international trade problem			\\ \hline \\[-1em]
 ODA					& せいふかいはつえんじょ 	& Official Development Assistance 		\\ \hline \\[-1em]
 国際連合					& こくさいれんごう 		& United Nations 						%\\ \hline \\[-1em]
\end{tabular}
%\end{center}
\section{\ruby{宗教}{しゅうきょう} \tabni religion | \ruby{社会思想}{しゃかいしそう} \tabni societal-thought }
\begin{tabular}{ l | p{6.7cm} | p{7.5cm} }
\ruby{社会}{しゃかい} & しゃかい 				& society								\\ \hline \\[-1em]
宗教 			& しゅうきょう 				& religion								\\ \hline \\[-1em]
思想	 			& しそう 					& thought; idea; ideology				\\ \hline \\[-1em]
恩 				& おん 					& favour								\\ \hline \\[-1em]
義理				& ぎり 					& duty; honour							\\ \hline \\[-1em]
義理チョコ			& ぎりちょこ 				& obligatory-gift chocolate				\\ \hline \\[-1em]
人情				& にんじょう					& humanity								\\ \hline \\[-1em]
恥 				& はじ 					& shame									\\ \hline \\[-1em]
根回し 			& ねまわし 				& making necessary arrangements			\\ \hline \\[-1em]
本音				& ほんね 					& real intention; motive				\\ \hline \\[-1em]
建前				& たてまえ 				& face; official stance					\\ \hline \\[-1em]
年功序列 			& ねんこうじょれつ 			& seniority by length of service		\\ \hline \\[-1em]
村八分 			& むらはちぶ 				& ostracism								\\ \hline \\[-1em]
賄賂、袖の下		& わいろ, そでのした 			& bribe									\\ \hline \\[-1em]
心付け 			& こころづ 					& gratuity; tip							\\ \hline \\[-1em]
祝儀、ご祝儀 		& しゅうぎ, ごしゅうぎ 			& congratulations						\\ \hline \\[-1em]
お祝い			& おいわい 				& congratulation						\\ \hline \\[-1em]
お祝いの贈り物 		& おいわいのおくりもの 			& congratulations-present				\\ \hline \\[-1em]
お祝いの品 		& おいわいのひん				& congratulations-gift					\\ \hline \\[-1em]
(お)中元			& (お)ちゅうげん 			& Bon Festival, Bon Festival gifts		\\ \hline \\[-1em]
(お)歳暮 			& (お)せいぼ 				& end of the year, year-end gift		\\ \hline \\[-1em]
お返し 			& おかえし 				& return gift							\\ \hline \\[-1em]
年賀状 			& ねんがじょう 				& New Year's card						\\ \hline \\[-1em]
名刺				& めいし 					& business card 						\\ \hline \\[-1em]
神道 			& しんとう 					& Shinto								\\ \hline \\[-1em]
神社 			& じんじゃ 					& Shinto shrine							\\ \hline \\[-1em]
鳥居 			& とりい 					& Shinto shrine archway					\\ \hline \\[-1em]
仏教 			& ぶっきょう 				& Buddhism								\\ \hline \\[-1em]
(お)寺、寺院、仏閣	& (お)てら, じいん, ぶっかく 	& temple								\\ \hline \\[-1em]
三門				& さんもん 				& large triple gate to temple			\\ \hline \\[-1em]
神仏習合			& しんぶつしゅうごう			& syncretism of Shinto and Buddhism		\\ \hline \\[-1em]
神仏分離			& しんぶつぶんり 				& separation of Buddhism and Shintoism	\\ \hline \\[-1em]
キリスト教 			& キリストきょう 				& Christianity							\\ \hline \\[-1em]
教会 			& きょうかい 				& church								\\ \hline \\[-1em]
右翼				& うよく 					& right-wing (politics)					\\ \hline \\[-1em]
神風				& かみかぜ 				& divine wind							\\ \hline \\[-1em]
切腹 			& せっぷく 					& ritual suicide 					   %\\ \hline \\[-1em]
\end{tabular}
\section{\ruby{教育}{きょういく} \tabni education }
\begin{tabular}{ l | p{6.7cm} | p{7.5cm} }
教育 &  きょういく  & education \\ \hline \\[-1em]
文部科学省、文科省 &  もんぶかがくしょう, もんかしょう  & Ministry of Education, Culture, Sports, Science and Technology; MEXT \\ \hline \\[-1em]
学校制度 &  がっこう せいど   & school-system \\ \hline \\[-1em]
義務教育 &  ぎむきょういく  & compulsory education \\ \hline \\[-1em]
小学校 & しょうがっこう & primary/elementary/grade - school \\ \hline \\[-1em]
中学校 & ちゅうがっこう & middle school \\ \hline \\[-1em]
高等学校、高校 &  こうとうがっこう /  こうこう  & high school \\ \hline \\[-1em]
大学 &  だいがく  & university/college\\ \hline \\[-1em]
大学院  & だいがくいん  & graduate school \\ \hline \\[-1em]
修士課程  &  しゅうしかてい  & master's course \\ \hline \\[-1em]
博士課程  &  はくしかてい  & doctoral course \\ \hline \\[-1em]
国立大学 &  こくりつだいがく  & national university \\ \hline \\[-1em]
公立大学 &  こうりつだいがく  & public university\\ \hline \\[-1em]
私立大学 &  しりつだいがく & private university\\ \hline \\[-1em]
帝国大学、旧帝大 &  ていこくだいがく/きゅうていだい  & (former) imperial university \\ \hline \\[-1em]
七帝大 &   ななていだい & national university  \\ \hline \\[-1em]
北海道大学 & ほっかいどうだいがく & Hokkaido University (Baa)\\ \hline \\[-1em]
東北大学 &  とうほくだいがく  & Northeastern University (A)\\ \hline \\[-1em]
東京大学 & とうきょうだいがく & Tokyo University (Aaa 1st)\\ \hline \\[-1em]
名古屋大学 &  なごやだいがく & Nagoya University (A)\\ \hline \\[-1em]
京都大学 &  きょうと だいがく& Kyoto University (Aaa 2nd)\\ \hline \\[-1em]
大阪大学 &  おおさか だいがく& Osaka University (Aa)\\ \hline \\[-1em]
九州大学 &  きゅうしゅうだいがく& Kyuushuu University (A)\\ \hline \\[-1em]
早稲田大学 & わせだだいがく& Waseda University (A 8th)\\ \hline \\[-1em]
慶応大学 &  けいおう だいがく& Keio University \\ \hline \\[-1em]
上智大学 &  じょうちだいがく& Sophia University (Baa {Same as Würzburg})\\ \hline \\[-1em]
進学率 &  しんがくりつ  & rate of advancement to the next higher level of education\\ \hline \\[-1em]
塾 & じゅく &  cram(ming) school \\ \hline \\[-1em]
家庭教師 &  かていきょうし  & private tutor\\ \hline \\[-1em]
浪人 &  ろうにん & ronin \\ \hline \\[-1em]
予備校 &  よびこう & prep school (ronin year) \\ \hline \\[-1em]
教育費 &  きょういくひ & education or school expenses\\ \hline \\[-1em]
学費  &  がくひ & tuition; school expenses\\ \hline \\[-1em]
授業料 &  じゅぎょうりょう & tuition fee; course fee\\ \hline \\[-1em]
生活費 &  せいかつひ & living expenses \\ \hline \\[-1em]
学閥 &  がくばつ & alma mater clique; old school tie %\\ \hline \\[-1em]
\end{tabular}
\section{\ruby{経済}{けいざい} \tabni economy }
\begin{tabular}{ l | l | p{7.5cm} }
企業 & きぎょう & business \\ \hline \\[-1em]
大企業 & だいきぎょう & large company \\ \hline \\[-1em]
中小企業 & ちゅうしょうきぎょう & Small and medium enterprises \\ \hline \\[-1em]
豊田自動車 & とよだじどうしゃ & Toyota automobiles \\ \hline \\[-1em]
任天堂 & にんてんどう & Nintendo \\ \hline \\[-1em]
ソニー & そにい & Sony \\ \hline \\[-1em]
財閥 & ざいばつ & financial conglomerate \\ \hline \\[-1em]
三菱 & みつびし & Mitsubishi \\ \hline \\[-1em]
三井 & みつい & Mitsui \\ \hline \\[-1em]
住友 & すみとも & Sumitomo \\ \hline \\[-1em]
財閥解体 & ざいばつかいたい & dissolution of zaibatsu \\ \hline \\[-1em]
六大グループ & ろくだい-ぐるうぷ & the six elements group \\ \hline \\[-1em]
三菱グループ & みつびし-ぐるうぷ & Mitsubishi Group \\ \hline \\[-1em]
三井グループ & みつい-ぐるうぷ & Mitsui group \\ \hline \\[-1em]
住友グループ & すみとも-ぐるうぷ & Sumitomo Group \\ \hline \\[-1em]
バブル経済 & バブルけいざい & Economic bubble \\ \hline \\[-1em]
三大メガバンク & さんだいめがばんく & the big three banks \\ \hline \\[-1em]
三菱UFJフィナンシャル && Mitsubishi UFJ Financial (Group) \\ \hline \\[-1em]
三井住友フィナンシャル && Sumitomo Mitsui Financial (Group) \\ \hline \\[-1em]
みずほフィナンシャル && Mizuho Financial (Group) \\ \hline \\[-1em]
ソフトバンク & そふとばんく & SoftBank (group) \\ \hline \\[-1em]
GDP & じー でぃーぴー & GDP \\ \hline \\[-1em]
経済成長率 & けいざいせいちょうりつ & rate of economic growth \\ \hline \\[-1em]
労働時間 & ろうどうじかん & working hours \\ \hline \\[-1em]
失業率 & しつぎょうりつ & unemployment rate \\ \hline \\[-1em]
非正社員 & ひせいしゃいん & non-fulltime/permanent employee\\ \hline \\[-1em]
為替レート & かわせれえと & exchange rate \\ \hline \\[-1em]
外貨準備高 & がいかじゅんびだか & total amount of foreign currency reserves \\ \hline \\[-1em]
貿易(相手国) & ぼうえき(あいてこく) & International trade (partner country)\\ \hline \\[-1em]
輸出 & ゆしゅつ & export \\ \hline \\[-1em]
輸入 & ゆにゅう & import \\ % \\[-1em] %\hline
\end{tabular}
\section{\ruby{文化}{ぶんか} \tabni culture }
\begin{tabular}{ l | l | p{11.5cm} }
衣 & い (ころも) & clothes \\ \hline \\[-1em]
体格 &  たいかく  & physique; constitution \\ \hline \\[-1em]
平均寿命&  へいきんじゅみょう & life expectancy \\ \hline \\[-1em]
死亡率 &  しぼうりつ  & death rate; mortality  \\ \hline \\[-1em]
着物 &  きもの & kimono jap-clothing; jap-clothes  \\ \hline \\[-1em]
和服 &  わふく & traditional Japanese clothes \\ \hline \\[-1em]
冠婚葬祭 &  かんこんそうさい & important ceremonial occasions in family relationships \\ \hline \\[-1em]
七五三 & しちごさん & festival (shrine visit) by children aged 7, 5 and 3 \\ \hline \\[-1em]
成人式 &  せいじんしき & coming-of-age ceremony \\ \hline \\[-1em]
卒業式 & そつぎょうしき & graduation ceremony \\ \hline \\[-1em]
結婚式 &  けっこんしき & marriage ceremony; wedding; nuptials \\ \hline \\[-1em]
披露宴 &   ひろうえん  & reception (wedding) \\ \hline \\[-1em]
帯 & おび & obi; kimono sash  \\ \hline \\[-1em]
紋付羽織袴 &もんつきはおりはかま& Japanese male formal attire decorated with one's family crest \\ \hline \\[-1em]
紋付& もんつき &clothing (e.g. kimono) decorated with one's family crest \\ \hline \\[-1em]
紋&もん&(family) crest; coat of arms \\ \hline \\[-1em]
羽織 & はおり &haori (Japanese formal coat) \\ \hline \\[-1em]
袴 & はかま & hakama; man's formal divided skirt \\ \hline \\[-1em]
足袋 & たび & tabi; Japanese socks (with split toe) \\ \hline \\[-1em]
雪駄 & せった & leather-soled sandals \\ \hline \\[-1em]
下駄 & げた & geta; Japanese wooden clogs \\ \hline \\[-1em]
草履 & ぞうり & zori; Japanese sandals \\ \hline \\[-1em]
振袖 & ふりそで & long-sleeved kimono \\ \hline \\[-1em]
留袖 & とめそで & formal, usually black, kimono with designs along the bottom of the skirt worn by married women on ceremonial occasions; \\ \hline \\[-1em]
日本髪 &にほんがみ&traditional Japanese women's hairstyle \\ \hline \\[-1em]
浴衣 & ゆかた & yukata (light cotton kimono worn in the summer or used as a bathrobe) \\ \hline \\[-1em]
手拭 & てぬぐい & japanese (hand) towel \\ \hline \\[-1em]
うちわ & うちわ &  uchiwa; type of traditional Japanese handheld fan  \\ \hline \\[-1em]
扇子 &  せんす  & folding fan \\ \hline \\[-1em]
懐紙 &  かいし & paper folded and tucked inside the front of one's kimono (esp. for use at the tea ceremony) %\\ \hline \\[-1em]
\end{tabular}
\section{\ruby{食}{しょく} \tabni food}
\begin{tabular}{ l | l | p{11.5cm} }														%is voc marked words are already anki vocs, the others maybe should or shouldnt be added later	
ご飯            &ごはん                         &cooked rice \\ \hline \\[-1em]
米              &こめ                           &uncooked rice \\ \hline \\[-1em]
稲              &いね                           &rice plant \\ \hline \\[-1em]
日本料理        &にほんりょうり                 &japanese cuisine \\ \hline \\[-1em]
和食            &わしょく                       &japanese food \\ \hline \\[-1em]
どんぶり        &                               &porcelain bowl \\ \hline \\[-1em]
丼物            &どんぶりもの                   &bowl of rice with food on top \\ \hline \\[-1em]
なべ            &                               &stew; hot pot \\ \hline \\[-1em]
鍋料理          &なべりょうり                   &hot pot cooking \\ \hline \\[-1em]
しゃぶしゃぶ    &                               &thinly sliced meat boiled quickly with vegetables \\ \hline \\[-1em]
すき焼き        &すきやき                       &sukiyaki; thin slices of beef \\ \hline \\[-1em]
天ぷら          &てんぷら                       &tempura; deep-fried fish and vegetables in a light batter \\ \hline \\[-1em]
刺身            &さしみ                         &sashimi (raw sliced fish, shellfish or crustaceans) \\ \hline \\[-1em]
会席料理        & かいせきりょうり              &set of dishes served on an individual tray for entertaining guests; banquet \\ \hline \\[-1em]
寿司            &すし                           &anything made with vinegared rice \\ \hline \\[-1em]
にぎり寿司      &にぎりずし                     &nigirizushi /hand-formed sushi with a topping of seafood \\ \hline \\[-1em]
巻き寿司        &まきずし                       &makizushi; sushi made rolled in nori seaweed with a core of filling \\ \hline \\[-1em]
手巻き寿司      &てまきずし                     &hand-rolled sushi (usually cone-shaped) \\ \hline \\[-1em]
稲荷寿司        &いなりずし                     &sushi wrapped in fried tofu \\ \hline \\[-1em]	
シャリ          &                               &sushi rice; rice seasoned with vinegar, sugar and salt \\ \hline \\[-1em]
ムラサキ        &                               &murasaki /Purple Gromwell \\ \hline \\[-1em]
醤油            &しょうゆ                       &soy sauce \\ \hline \\[-1em]
(サビ)わさび    &                               &(at a sushi shop) wasabi \\ \hline \\[-1em]
アガリ          &                               & \\ \hline \\[-1em]
お茶            &おちゃ                         &green tea \\ \hline \\[-1em]																		%is voc
抹茶            &まっちゃ                       &matcha; powdered green tea   \\ \hline \\[-1em]														%is voc
玉露            &ぎょくろ                       &high-quality green tea  \\ \hline \\[-1em]															%is voc
煎茶            &せんちゃ                       &green leaf tea; non-powdered tea (as opposed to matcha) \\ \hline \\[-1em]							%is voc
番茶            &ばんちゃ                       &coarse tea \\ \hline \\[-1em]																		%is voc
麺              &めん                           &noodles \\ \hline \\[-1em]																		%is voc
饂飩            &うどん                         &udon; thick Japanese wheat noodles \\ \hline \\[-1em] 					%is voc
そうめん        &                               &fine white noodles \\ \hline \\[-1em]							
そば            &                               &soba; Japanese buckwheat noodles \\ \hline \\[-1em]
ラーメン        &                               &ramen; Chinese-style noodles \\ \hline \\[-1em]
日本酒          &にほんしゅ                     &sake; Japanese rice wine \\ \hline \\[-1em]															%is voc
焼酎            &しょうちゅう                   &Japanese spirit distilled from sweet potatoes, rice, etc.   \\ \hline \\[-1em]						%is voc
ビール          &                               &beer \\ \hline \\[-1em]
発泡酒          &はっぽうしゅ                   &happoshu; low-malt beer-like beverag /sparkling wine \\ \hline \\[-1em]
第三のビール    &だいさんのビール               &cheap, malt-free beer-like alcoholic beverage \\ \hline \\[-1em]
みりん          &                               &mirin; type of sweet rice wine used in cooking \\ \hline \\[-1em]
酢              &す                             &vinegar \\ \hline \\[-1em]																		%is voc
味噌            &みそ                           &miso; fermented condiment usu. made from soybeans  \\ \hline \\[-1em]
出汁            &だし                           &dashi (Japanese soup stock made from fish and kelp) \\ \hline \\[-1em]
味の素          &あじのもと                     &Ajinomoto, brand name of monosodium glutamate (MSG) \\ \hline \\[-1em]
梅干し          &うめぼし                       &umeboshi; pickled ume; pickled plum  \\ \hline \\[-1em]												%is voc
漬け物          &つけもの                       &tsukemono; Japanese pickled vegetables  \\ \hline \\[-1em]
\end{tabular}
\section{\ruby{住}{じゅう} \tabni housing}
\begin{tabular}{ l | l | p{10.5cm} }
玄関        &げんかん   &entry hall; vestibule; foyer; entryway;    \\ \hline
畳          &たたみ     &tatami mat; Japanese straw floor coverings \\ \hline
座布団      &ざぶとん   &zabuton (flat floor cushion for sitting or kneeling) \\ \hline
正座    &せいざ &seiza; kneeling with the tops of the feet flat on the floor\\ \hline
あぐら      &           &sitting cross-legged {lit. foreign/barbarian sitting} \\ \hline
座椅子      &ざいす     &tatami chair; legless chair \\ \hline
布団        &ふとん     &futon (quilted Japanese-style mattress laid out on the floor) \\ \hline
空調        &くうちょう &air conditioning /AC \\ \hline
エアコン    &えあこん   &air conditioning /AC \\ \hline
クーラー    &くうらあ   &air conditioning /AC \\ \hline
扇風機      &せんぷうき &electric fan \\ \hline
風鈴        &ふうりん   &wind chime; wind bell \\ \hline
金魚        &きんぎょ   &goldfish \\ \hline
こたつ      &           &kotatsu; table over an electric heater \\ \hline
火鉢        &ひばち     &brazier; hibachi \\ \hline 
環境問題        &かんきょうもんだい         &environmental issue \\ \hline
エネルギー問題  &えねるぎいもんだい         &energy problems \\ \hline
発電所          &はつでんしょ               &power plant; power station \\ \hline
原子力発電所    &げんしりょくはつでんしょ   &nuclear power plant; nuclear power station \\ \hline
原発            &げんぱつ                   &nuclear power plant; nuclear power generation \\ \hline
原子力発電      &げんしりょくはつでん       &nuclear power/electricity generation \\ \hline
地熱            &じねつ                     &geothermy; terrestrial heat \\ \hline
太陽光          &たいようこう               &sunlight \\ \hline
風力            &ふうりょく                 &wind power \\ \hline
潮力            &ちょうりょく               &tidal energy \\ \hline
火力            &かりょく                   &heating power; steam power; || military: firepower \\ \hline
水力            &すいりょく                 &hydropower; water power \\ \hline
二酸化炭素      &にさんかたんそ             &carbon dioxide \\ \hline
地球温暖化      &ちきゅうおんだんか         &global warming \\ \hline
黄砂            &こうさ                     &loess; bai; yellow river dust/sand; asian dust \\ \hline
PM2.5           &                           &particles smaller than 2.5 micrometers
\end{tabular}
\section{\ruby{歴史}{れきし} \tabni history}
\begin{tabular}
時代       &じだい &period; epoch; era; age \\ \hline
世紀       &       &
縄文時代
弥生時代
大和時代
飛鳥時代
奈良時代
平安時代
藤原時代
院政時代
鎌倉時代
南北朝時代
室町時代
戦国時代
安土桃山時代
江戸時代
明治時代
大正時代
昭和時代
平成時代
今上天皇
年表 Nenpyô
西紀 Seiki、西暦 Seireki
干支 Eto
中国 Chûgoku
東北 Tôhoku・
朝鮮 Chôsen
日本 Nihon, Nippon
日紀 Nikki
前660年 Zen 660 nen 、紀元前660年 Kigen-zen 660 nen
周 Shû 
惠王 Kei-ô
神武(天皇)Jinmu-tennô 
中華人民共和國 Chûka-jinmin-kyôwa-koku
中華民國(台灣)Chûka-min-koku
朝鮮民主主義人民共和國(北朝鮮) Chôsen-minshu-shugi-jinmin-kyôwa-koku, Kita chôsen,
大韓民國(南朝鮮) Dai-kann-min-koku, Minami chôsen
昭和時代 Shôwa-jidai
昭和天皇 Shôwa-tennô
平成時代 Heisei-jidai
今上天皇 Kinjô-tennô
2017年Ni-sen-jûnana-nen 、平成29年 Heisei-nijû-kyû-nen
暦
十干
甲
乙
丙
丁
戊
己
庚
辛
壬
癸
十二支
子
丑
寅
卯
辰
巳
午
未
申
酉
戌
亥
丙午
1960年の干支は。
1996年の干支は。
\end{tabular}
\end{document}
