\documentclass{article}
\usepackage{xeCJK}
\usepackage[CJK,overlap]{ruby}
\usepackage{geometry}
 \geometry{
 a4paper,
 %total={170mm,257mm},
 left=10mm,
 top=12mm,
 right=10mm,
 bottom=12mm,
 }
\setCJKmainfont{ipaexm.ttf}
\renewcommand{\rubysep}{0.1ex}%3.9ex}
\newcommand\tabni[1][0.2cm]{\hspace*{#1}}
\begin{document} \large
\section{地理 (ちり) geography	| 自然 (しぜん) nature}
\begin{tabular}{ l | l | p{7.5cm} }
\ruby{地理}{ちり} &  ちり  & geography \\ \hline \\[-1em]
\ruby{自然}{しぜん} &  しぜん & nature \\ \hline \\[-1em]
\ruby{気候}{ きこう } &  きこう  & climate \\ \hline \\[-1em]
\ruby{琵琶湖}{ びわこ } & びわこ/びわみずうみ & lake biwa(ko)   \\ \hline \\[-1em]
\ruby{北海道}{ほっかいどう} & ほっかいどう & Hokkaido  \\ \hline \\[-1em]
\ruby{本州}{ ほんしゅう } &  ほんしゅう  & Honshuu  \\ \hline \\[-1em]
\ruby{四国}{ しこく } &  しこく & Shikoku  \\ \hline \\[-1em]
\ruby{九州}{ きゅうしゅう } &  きゅうしゅう  & Kyushu  \\ \hline \\[-1em]
\ruby{沖縄}{ おきなわ } &  おきなわ  & Okinawa  \\ \hline \\[-1em]
\ruby{都道府県}{ とどうふけん } &  とどうふけん  & administrative divisions of Japan \\ \hline \\[-1em]
\ruby{札幌}{ さっぽろ } &  さっぽろ & Sapporo  \\ \hline \\[-1em]
\ruby{仙台}{ せんだい } &  せんだい  & Sendai  \\ \hline \\[-1em]
\ruby{東京}{ とうきょう } &  とうきょう  & Tokyo  \\ \hline \\[-1em]
\ruby{名古屋}{ なごや } &  なごや & Nagoya  \\ \hline \\[-1em]
\ruby{京都}{ きょうと } &  きょうと  & Kyoto  \\ \hline \\[-1em]
\ruby{大阪}{ おおさか } &  おおさか  & Osaka  \\ \hline \\[-1em]
\ruby{福岡}{ ふくおか } &  ふくおか  & Fukuoka  \\ \hline \\[-1em]
\ruby{広島}{ ひろしま } &  ひろしま  & Hiroshima  \\ \hline \\[-1em]
\ruby{長崎}{ ながさき } &  ながさき  & Nagasaki  \\ \hline \\[-1em]
\ruby{福島}{ふくしま} & ふくしま &Fukushima  \\ \hline \\[-1em]
\ruby{人口}{ じんこう } &  じんこう  & population  \\ \hline \\[-1em]
\ruby{島}{しま} & しま & island  \\ \hline \\[-1em]
\ruby{富士山}{ ふじさん} & ふじさん & Fuji-San  \\ \hline \\[-1em]
\ruby{火山}{ かざん } &  かざん & volcano  \\ \hline \\[-1em]
\ruby{温泉}{ おんせん } &  おんせん  & onsen  \\ \hline \\[-1em]
\ruby{地震}{ じしん } &  じしん  & earthquake  \\ \hline \\[-1em]
\ruby{津波}{ つなみ } &  つなみ  & tsunami  \\ \hline \\[-1em]
\ruby{気温}{ きおん } &  きおん & atmospheric temperature  \\ \hline \\[-1em]
\ruby{梅雨}{ばいう} & ばいう/つゆ & rainy season  \\ \hline \\[-1em]
\ruby{台風}{ たいふう } &  たいふう  & typhoon  \\ \hline \\[-1em]
\ruby{動物}{ どうぶつ } &  どうぶつ  & animal  \\ \hline \\[-1em]
\ruby{鳥}{とり} & とり & bird   \\ \hline \\[-1em]
\ruby{植物}{ しょくぶつ } &  しょくぶつ  & plant / vegetation   \\ \hline \\[-1em]
\end{tabular}
\section{政治 (せいじ) politics}
\begin{tabular}{ l | l | p{7.5cm} }
憲法 & けんぽう & constitution \\ \hline \\[-1em]
天皇 & てんのう & Emperor of Japan \\ \hline \\[-1em]
政治 & せいじ & politics; government \\ \hline \\[-1em]
三権分立 & さんけんぶんりつ & separation of powers \\ \hline \\[-1em]
立法 & りっぽう & legislation \\ \hline \\[-1em]
国会 & こっかい & National Diet; parliament; congress \\ \hline \\[-1em]
衆議院 & しゅうぎいん & lower house; House of Representatives \\ \hline \\[-1em]
参議院 & さんぎいん & House of Councillors \\ \hline \\[-1em]
選挙	& せんきょ	& election \\ \hline \\[-1em]
選挙権 & せんきょけん & suffrage; the right to vote \\ \hline \\[-1em]
政党 & せいとう & political party \\ \hline \\[-1em]
自由民主党 & じゆうみんしゅとう & Liberal Democratic Party; LDP\\ \hline \\[-1em]
公明党 & こうめいとう & Komeito / Justice Party \\ \hline \\[-1em]
民進党 & みんしんとう	& Democratic Party \\ \hline \\[-1em]
社会民主党 & しゃかいみんしゅとう & Social Democratic Party \\ \hline \\[-1em]
日本共産党 & にほんきょうさんとう & Japanese Communist Party \\ \hline \\[-1em]
ドイツキリスト\ruby{教民主同盟}{きょうみんしゅどうめい} &  & (CDU) \\ \hline \\[-1em]
ドイツ\ruby{社会民主党}{しゃかいみんしゅとう} &  & (SPD) \\ \hline \\[-1em]
バイエルン &  & Bayern \\ \hline \\[-1em]
キリスト\ruby{教社会同盟}{きょうしゃかいどうめい} &  & (CSU) \\ \hline \\[-1em]
\ruby{同盟90/緑の党}{どうめいきゅうぜろ-みどりのとう} &  & (Bündnis 90 die Grünen) \\ \hline \\[-1em]
\ruby{左翼党}{さよくとう} & & Die Linke \\ \hline \\[-1em]
\ruby{自由民主党}{じゆうみんしゅとう} &  & Liberal Democratic Party; LDP\\ \hline \\[-1em]
行政 & ぎょうせい & administration / executive \\ \hline \\[-1em]
内閣	& ないかく & cabinet(government) \\ \hline \\[-1em]
内閣総理大臣 & ないかくそうりだいじん & primeminister \\ \hline \\[-1em]
(首相) & しゅしょう & chancellor \\ \hline \\[-1em]
大統領 & だいとうりょう & president \\ \hline \\[-1em]
司法 & しほう & administration of justice; Judiciary \\ \hline \\[-1em]
最高裁判所 & さいこうさいばんしょ & Supreme Court \\ \hline \\[-1em]
裁判所 & さいばんしょ & court; courthouse \\ \hline \\[-1em]
天皇明仁 & てんのうあきひと & Akihito Emperor of Japan \\ \hline \\[-1em]
今上天皇 & きんじょうてんのう & the present Emperor \\ \hline \\[-1em]
皇后 & こうごう & (Japanese) empress; queen \\ \hline \\[-1em]
陛下 & へいか & your Majesty; his Majesty; her Majesty \\ \hline \\[-1em]
皇太子徳仁親王 & こうたいしなるひと & Crown Princess Masako \\ \hline \\[-1em]
秋篠宮文仁 & あきしののみやふみひと & Akishino-no-miya Fumihito \\ \hline \\[-1em]
親王 & しんのう & prince of royal blood; Imperial prince \\ \hline \\[-1em]
悠仁親王 & はるかひとししんのう & Prince Hisahito of Akishino \\ \hline \\[-1em]
殿下 & でんか & your/his/her Highness \\ \hline \\[-1em]
宮内庁 & くないちょう & Imperial Household Agency \\ \hline \\[-1em]
\end{tabular}
\section{外交 (がいこう) diplomacy}
%\begin{center}
    \begin{tabular}{  l | l | p{7.5cm} }
    %\hline				& & \\ \hline
% Kanji 			& Hiragana 			& Translation \\[5pt] \hline	\\[-1em]
 外交 			& がいこう 				& diplomacy   \\ \hline	\\[-1em]
 外務省			& がいむしょう			& Ministry of Foreign Affairs \\ \hline \\[-1em]
 国交問題			& こっこうもんだい			& diplomatic relations problem \\ \hline \\[-1em]
 サンフランシスコ講和条約  & SFRこうわじょうやく & San Francisco Treaty Sept. 8, 1951)\\ \hline \\[-1em]
 日華平和条約		& にっかへいわじょうやく  & peace treaty between CHI and JAP 1978\\ \hline \\[-1em]
 日ソ共同宣言				& にっきょうどうせんげん 	& JAP-SOV joint declaration \\ \hline \\[-1em]
 日韓基本条約				& にっかんきほんじょうやく	& JAP-SKOR basic treaty\\ \hline \\[-1em]
 日中共同声明				& にちゅきょうどうせいめい	& CHI-JAP joint declaration \\ \hline \\[-1em]
 平和条約					& へいわじょうやく 		& peace treaty \\ \hline \\[-1em]
 領土問題					& りょうどもんだい			& territorial dispute \\ \hline \\[-1em]
 ロシア					& ろしあ				& RUS	\\ \hline \\[-1em]
 北方領土					& ほっぽうりょうど 		& the Northern Territories\\ \hline \\[-1em]
 日露和親条約				& にちろわしんじょうやく		& RUS-JAP peace treaty\\ \hline \\[-1em]
 中国					& ちゅうごく				& CHI \\ \hline \\[-1em]
 尖閣諸島					& せんかくしゅとう			& Senkaku Islands \\ \hline \\[-1em]
 台湾					& たいわん				& Taiwan \\ \hline \\[-1em]
 韓国					& かんこく 				& South Korea\\ \hline \\[-1em]
 竹島					& たけしま				& Liancourt Rocks \\ \hline \\[-1em]
 慰安婦問題				& いあんふもんだい		& comfort women problem\\ \hline \\[-1em]
 北朝鮮					& きたちょうせん			& North Korea \\ \hline \\[-1em]
 防衛問題					& ぼうえいもんだい		& Defence Problem \\ \hline \\[-1em]
 アメリカ					& あめりか				& USA	 \\ \hline \\[-1em]
 貿易問題					& ぼうえきもんだい		& international trade problem\\ \hline \\[-1em]
 ODA				& せいふかいはつえんじょ  &Official Development Assistance \\ \hline \\[-1em]
 国際連合					& こくさいれんごう 		& United Nations \\ \hline \\[-1em]
    \end{tabular}
%\end{center}
\section{\ruby{宗教}{しゅうきょう} \tabni religion | \ruby{社会思想}{しゃかいしそう} \tabni societal-thought }
\begin{tabular}{ l | p{6.7cm} | p{7.5cm} }
\ruby{社会}{しゃかい} & しゃかい & society\\ \hline \\[-1em]
宗教 & しゅうきょう & religion\\ \hline \\[-1em]
思想 & しそう & thought; idea; ideology\\ \hline \\[-1em]
恩 & おん & favour\\ \hline \\[-1em]
義理	& ぎり & duty; honour\\ \hline \\[-1em]
義理チョコ	& ぎりちょこ & obligatory-gift chocolate\\ \hline \\[-1em]
人情	& にんじょう & humanity\\ \hline \\[-1em]
恥 & はじ & shame\\ \hline \\[-1em]
根回し & ねまわし & making necessary arrangements\\ \hline \\[-1em]
本音	& ほんね & real intention; motive\\ \hline \\[-1em]
建前	& たてまえ & face; official stance\\ \hline \\[-1em]
年功序列 & ねんこうじょれつ & seniority by length of service\\ \hline \\[-1em]
村八分 & むらはちぶ & ostracism\\ \hline \\[-1em]
賄賂、袖の下 & わいろ, そでのした & bribe\\ \hline \\[-1em]
心付け & こころづ & gratuity; tip\\ \hline \\[-1em]
祝儀、ご祝儀 & しゅうぎ, ごしゅうぎ & congratulations\\ \hline \\[-1em]
お祝い & おいわい & congratulation\\ \hline \\[-1em]
お祝いの贈り物 & おいわいのおくりもの & congratulation-present\\ \hline \\[-1em]
、お祝いの品 & おいわい & congratulation-elegance\\ \hline \\[-1em]
中元、お中元 & ちゅうげん, おなかもと & Bon Festival, Bon Festival gifts\\ \hline \\[-1em]
(お)歳暮 & (お)せいぼ & end of the year, year-end gift\\ \hline \\[-1em]
お返し & おかえし & return gift\\ \hline \\[-1em]
年賀状 & ねんがじょう & New Year's card\\ \hline \\[-1em]
名刺	& めいし & business card \\ \hline \\[-1em]
神道 & しんとう & Shinto\\ \hline \\[-1em]
神社 & じんじゃ & Shinto shrine\\ \hline \\[-1em]
鳥居 & とりい & Shinto shrine archway\\ \hline \\[-1em]
仏教 & ぶっきょう & Buddhism\\ \hline \\[-1em]
(お)寺、寺院、仏閣	& (お)てら, じいん, ぶっかく & temple\\ \hline \\[-1em]
三門	& さんもん & large triple gate to temple\\ \hline \\[-1em]
神仏習合	& しんぶつしゅうごう & syncretism of Shinto and Buddhism\\ \hline \\[-1em]
神仏分離	& しんぶつぶんり & separation of Buddhism and Shintoism\\ \hline \\[-1em]
キリスト教 & キリストきょう & Christianity\\ \hline \\[-1em]
教会 & きょうかい & church\\ \hline \\[-1em]
右翼	& うよく & right-wing (politics)\\ \hline \\[-1em]
神風	& かみかぜ & divine wind\\ \hline \\[-1em]
切腹 & せっぷく & ritual suicide\\ \hline \\[-1em]
\end{tabular}
\section{\ruby{教育}{きょういく} \tabni Education }
\begin{tabular}{ l | p{6.7cm} | p{7.5cm} }
教育 &  きょういく  & education \\ \hline \\[-1em]
文部科学省、文科省 &  もんぶかがくしょう, もんかしょう  & Ministry of Education, Culture, Sports, Science and Technology; MEXT \\ \hline \\[-1em]
学校制度 &  がっこう せいど   & school-system \\ \hline \\[-1em]
義務教育 & & \\ \hline \\[-1em]
小学校 & & \\ \hline \\[-1em]
中学校 & & \\ \hline \\[-1em]
高等学校、高校 & & \\ \hline \\[-1em]
大学 & & \\ \hline \\[-1em]
大学院  & & \\ \hline \\[-1em]
修士課程  & & \\ \hline \\[-1em]
博士課程  & & \\ \hline \\[-1em]
国立大学 & & \\ \hline \\[-1em]
公立大学 & & \\ \hline \\[-1em]
私立大学 & & \\ \hline \\[-1em]
帝国大学、旧帝大 & & \\ \hline \\[-1em]
七帝大 & & \\ \hline \\[-1em]
北海道大学 & & \\ \hline \\[-1em]
東北大学 & & \\ \hline \\[-1em]
東京大学 & & \\ \hline \\[-1em]
名古屋大学 & & \\ \hline \\[-1em]
京都大学 & & \\ \hline \\[-1em]
大阪大学 & & \\ \hline \\[-1em]
九州大学 & & \\ \hline \\[-1em]
早稲田大学 & & \\ \hline \\[-1em]
慶応大学 & & \\ \hline \\[-1em]
上智大学 & & \\ \hline \\[-1em]
進学率 & & \\ \hline \\[-1em]
塾 & & \\ \hline \\[-1em]
家庭教師 & & \\ \hline \\[-1em]
浪人 & & \\ \hline \\[-1em]
予備校 & & \\ \hline \\[-1em]
教育費 & & \\ \hline \\[-1em]
学費  & & \\ \hline \\[-1em]
授業料 & & \\ \hline \\[-1em]
生活費 & & \\ \hline \\[-1em]
学閥 & & \\ \hline \\[-1em]
\end{tabular}
\end{document}