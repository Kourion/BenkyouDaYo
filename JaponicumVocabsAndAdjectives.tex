\documentclass{article}
\usepackage{xeCJK}
\usepackage{enumitem}
\usepackage{titlesec}
\usepackage[dvipsnames]{xcolor}
\usepackage{geometry}
\usepackage[CJK,overlap]{ruby}
 \geometry{
 a4paper,
 %total={170mm,257mm},
 left=10mm,
 top=12mm,
 right=10mm,
 bottom=12mm,
 }
 
\renewcommand{\rubysep}{0.1ex}%3.9ex}
\makeatletter
\renewcommand\paragraph{\@startsection{paragraph}{4}{\z@}%
            {-0.5ex\@plus -.1ex \@minus -.25ex}%
            {-1em}%
            {\normalfont\normalsize\bfseries}}
\renewcommand\subparagraph{\@startsection{subparagraph}{5}{\z@}%
            {-0.1ex\@plus -.1ex \@minus -.25ex}%
            {-1em}%
            {\normalfont\normalsize\bfseries}}
\makeatother

\newcommand\tab[1][1cm]{\hspace*{#1}} 
\newcommand\tabi[1][0.05cm]{\hspace*{#1}}
\newcommand\tabni[1][0.2cm]{\hspace*{#1}}
\newcommand\tabyon[1][0.4cm]{\hspace*{#1}}
\newcommand\tabroku[1][0.6cm]{\hspace*{#1}}

\XeTeXlinebreaklocale "ja"
\XeTeXlinebreakskip=0em plus 0.1em minus 0.01em

\setlist{nolistsep}
\titlespacing*{\section}
{0pt}{1.5ex plus 1ex minus .2ex}{1.5ex plus .2ex}
\titlespacing*{\subsection}
{0pt}{0.5ex plus 1ex minus .2ex}{0.5ex plus .2ex}

\setlength{\parindent}{0pt}
\setCJKmainfont{ipaexm.ttf}
\setcounter{tocdepth}{5}

\begin{document}	%\large {1.5ex \@plus .2ex}%

%\tableofcontents
%\newpage

\section{Vocabulary}
%\begin{itemize}
%\item おぼえています --> to remember
%\item おぼえる --> to memorize
%\item わすれました --> to have forgotten
%\item わすれる --> to forget
%\item よとう --> ruling party
%\item やとう --> opposition party
%\end{itemize}
\begin{tabular}{ l | l l }
Kanji&Kana&Translation\\ \hline\\[-1em]
おぼえています &&--> to remember \\ \hline\\[-1em]
おぼえる &&--> to memorize \\ \hline\\[-1em]
わすれました &&--> to have forgotten \\ \hline\\[-1em]
わすれる &&--> to forget \\ \hline\\[-1em]
よとう &&--> ruling party \\ \hline\\[-1em]
やとう &&--> opposition party \\ \hline\\[-1em]
必要&ひつよう&--> to need
\end{tabular}
\section{な-Adjectives}
%\begin{itemize}
%\item \ruby{上手}{じょうず} (じょうず) --> good at
%\item \ruby{下手}{へた} (へた) --> bad at
%\item \ruby{得意}{とくい} (とくい) --> good at and likes
%\item \ruby{苦手}{にがて} (にがて) --> bad at and dislikes
%\end{itemize}
\begin{tabular}{ l | l l }
Kanji&Kana&Translation\\ \hline\\[-1em]
上手&じょうず&-->good at \\ \hline\\[-1em]
下手&へた&-->bad at \\ \hline\\[-1em]
得意&とくい&-->good at and likes \\ \hline\\[-1em]
苦手&にがて&-->bad at and dislikes \\ %\hline\\[-1em]
\end{tabular}
\end{document}