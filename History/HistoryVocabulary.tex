\documentclass{article}
\usepackage{xeCJK}
\usepackage{enumitem}
\usepackage{titlesec}
\usepackage[dvipsnames]{xcolor}
\usepackage{fancyhdr}
\usepackage{geometry}
\usepackage[CJK,overlap]{ruby}
\geometry{ a4paper, %total={170mm,257mm},
	left=5mm, top=10mm, right=5mm, bottom=12mm, } \pagestyle{fancy}
\lhead{}
\chead{}
\rhead{}
\lfoot{}
\cfoot{\thepage}
\rfoot{v0.7} %(23.JUN.2017)


\renewcommand{\rubysep}{0.1ex}%3.9ex}
\makeatletter\renewcommand\paragraph{\@startsection{paragraph}{4}{\z@}{-0.5ex\@plus -.1ex \@minus -.25ex}{-1em}{\normalfont\normalsize\bfseries}}
\renewcommand\subparagraph{\@startsection{subparagraph}{5}{\z@}{-0.1ex\@plus -.1ex \@minus -.25ex}{-1em}{\normalfont\normalsize\bfseries}}\makeatother
\renewcommand{\headrulewidth}{0pt} \renewcommand{\footrulewidth}{0pt}
\newcommand\tab[1][1cm]{\hspace*{#1}} \newcommand\tabi[1][0.05cm]{\hspace*{#1}}\newcommand\tabni[1][0.2cm]{\hspace*{#1}}
\newcommand\tabyon[1][0.4cm]{\hspace*{#1}}\newcommand\tabroku[1][0.6cm]{\hspace*{#1}}
\XeTeXlinebreaklocale "ja" \XeTeXlinebreakskip=0em plus 0.1em minus 0.01em \setlist{nolistsep}
\titlespacing*{\section}{0pt}{1.5ex plus 1ex minus .2ex}{1.5ex plus .2ex}
\titlespacing*{\subsection}{0pt}{0.5ex plus 1ex minus .2ex}{0.5ex plus .2ex}
\setlength{\parindent}{0pt}\setCJKmainfont{ipaexm.ttf}\setcounter{tocdepth}{5}

\begin{document} \large

\section*{歴史(れきし) History}
\begin{tabular}{l | l | p{10.5cm}}
時代 & じだい & period; epoch \\ \hline \\[-1em]
世紀 & せいき & century \\ \hline \\[-1em]
縄文時代 & じょうもんじだい & Jōmon period​ \\ \hline \\[-1em]
弥生時代 & やよいじだい & Yayoi period (c. 300 BCE-300 CE) \\ \hline \\[-1em]
卑弥呼 & ぴみく & Pimiko/Himiko (175-248 CE) was an obscure shaman queen of Yamataikoku in ancient Wa \\ \hline \\[-1em]
大和時代 & やまとじだい & Yamato period (300-550 CE) \\ \hline \\[-1em]​
天皇 & てんのう & Emperor of Japan​ \\ \hline \\[-1em]
飛鳥時代 & あすかじだい & Asuka period (550-710 CE)​ \\ \hline \\[-1em]
奈良時代 & ならじだい & Nara period (710-794 CE)​ \\ \hline \\[-1em]
平安時代 & へいあんじだい & Heian period (794-1185)​ \\ \hline \\[-1em]
平安京(京都) & へいあんきょう(きょうと) & Heian-kyō​ (Kyoto) \\ \hline \\[-1em]
鎌倉時代 & かまくらじだい & Kamakura period (1185-1333 CE)​ \\ \hline \\[-1em]
将軍 & しょうぐん & general; shogun \\ \hline \\[-1em]
侍 & さむらい & samurai; warrior \\ \hline \\[-1em]
武士 & ぶし & samurai; warrior \\ \hline \\[-1em]
神風 & かみかぜ & Divine Wind \\ \hline \\[-1em]
南北朝時代 & なんぼくちょうじだい & period of the northern and southern dynasties (Japan, 1336-92)​ \\ \hline \\[-1em]
室町時代 & むろまちじだい & Muromachi period (1333-1573 CE)​ \\ \hline \\[-1em]
安土桃山時代 & あづちももやまじだい & Azuchi-Momoyama period (approx. 1558-1600 CE) \\ \hline \\[-1em]​
江戸時代 & えどじだい & Edo period (1603-1868 CE)​ \\ \hline \\[-1em]
江戸(東京) & えど(とうきょう) & Edo(Tokyo) \\ \hline \\[-1em]
長崎 & ながさき & Nagasaki \\ \hline \\[-1em]
明治時代 & めいじじだい & Meiji period (1868-1912) \\ \hline \\[-1em]​
大正時代 & たいしょうじだい & Taisho period (1912-1926 CE)​ \\ \hline \\[-1em]
昭和時代 & しょうわじだい & Showa period (1926-1989 CE)​ \\ \hline \\[-1em]
平成時代 & へいせいじだい & Heisei period (1989-)​ %\\ \hline \\[-1em]
\end{tabular}

\section{中国( ちゅうごく) China}
\begin{tabular}{p{4.5cm} | p{4.5cm} | p{10.5cm}}
清 & しん &Qing (1616-1912)\\ \hline \\[-1em]
北京& ペキン & Beijing; Peking\\ \hline \\[-1em]
南京 & ナンキン & Nanjin; Nankin\\ \hline \\[-1em]
広州 & こうしゅう & Guangzhou; Canton\\ \hline \\[-1em]
朝貢 & ちょうこう & bringing tribute; Tribute\\ \hline \\[-1em]
弁髪 & べんぱつ & pigtail; Queue (hairstyle)\\ \hline \\[-1em]
朝鮮&  ちょうせん & Korea (1392-1910)\\ \hline \\[-1em]
---- & ソウル & Seoul \\ \hline \\[-1em]
江華島&くんわとう&Ganghwa Island\\ \hline \\[-1em]
日本& にほん & 660 B.C.E.\\ \hline \\[-1em]
将軍 & しょうぐん &general; shogun\\ \hline \\[-1em]
江戸& えど & Edo\\ \hline \\[-1em]
京都& きょうと &Kyoto\\ \hline \\[-1em]
大阪& おおさか &Osaka\\ \hline \\[-1em]
長崎& ながさき &Nagasaki\\ \hline \\[-1em]
鎖国& さこく &Sakoku; national isolation\\ \hline \\[-1em]
琉球& りゅうきゅう &Ryukyu\\ \hline \\[-1em]
台湾& たいわん &Taiwan\\ \hline \\[-1em]
鄭成功&ていせいこう&Koxinga Ancestral Shrine\\ \hline \\[-1em]
マカオ&まかお&Macau\\ \hline \\[-1em]
ネルチンスク条約 &Nerchinskじょうやく&Treaty of Nerchinsk\\ \hline \\[-1em]
アムール川&アムールがわ&Amur River%\\ \hline \\[-1em]
\end{tabular}

\section{1804-1848}
\begin{tabular}{p{4.5cm} | p{4.5cm} | p{10.5cm}}
1804 レザノフの来航&れざのふのらいこう&Arrival of Rezanov \\ \hline \\[-1em]
1808 フェートン号事件&ふぇーとんごう・じけん&Phaeton-Incident \\ \hline \\[-1em]
1828 シーボルト事件&しーぼると・じけん&Philipp Franz von Siebold Incident \\ \hline \\[-1em]
1840-42 阿片戦争&あへん・せんそう&First Opium War​ \\ \hline \\[-1em]
1846-1848 米墨戦争&べいぼくせんそう&Mexican-American War (1846-1848)​
\end{tabular}

\section{1851-1865}
\begin{tabular}{p{4.5cm} | p{4.5cm} | p{10.5cm}}
1851-64 太平天国の乱 & たいへいてんごく・の・らん  &Taiping Rebellion \\ \hline \\[-1em]
1853 黒船来航 & くろ・ふね・らいこう & Arrival of the black ships (Perry Expedition) \\ \hline \\[-1em]
1856-60 アロー号事件 & あろーごう・じけん & Arrow-Incident \\ \hline \\[-1em]
1856-60 第二次アヘン戦争 & だい・にじ・あへん・せんそう & second opium war \\ \hline \\[-1em]
1858 日米修好通商条約 & にち・べい・しゅうこう・つうしょう・じょうやく&JAP-USA-Commerce-Treaty \\ \hline \\[-1em]
1861-65 アメリカ南北戦争 & あめりか・なんぽく・せんそう & USA-Civil-War
\end{tabular}

\section{1863-1869}
\begin{tabular}{p{1.8cm} | p{3.0cm} | p{4.5cm} | p{7.5cm}}
1863 & 薩英戦争 & さつえいせんそう & Anglo-Satsuma War (bombardement of kagoshima) \\ \hline \\[-1em]
1863 & 尊王攘夷 & そんのうじょうい & expulsion of the barbarians \\ \hline \\[-1em]
1863, 64 & 下関戦争 & しものせきせんそう & Battles for Shimonoseki straits \\ \hline \\[-1em]
1865 & イギリス公使パークス着任 & イギリスこうしパークスちゃくにん & british envoy parkes takes up a new post \\ \hline \\[-1em]
1866 & 丙寅洋擾 & へいとらよう & French campaign against Korea (1866)​ \\ \hline \\[-1em]
1867 & 大政奉還 & たいせいほうかん & restoration of imperial rule \\ \hline \\[-1em]
1868-69 & 戊辰戦争 & ぼしんせんそう & war of the fifth sexagenary cycle​ \\ \hline \\[-1em]
1869 & 蝦夷共和国成立 & えぞきょうわこくせいりつ & foundation of the Republic of Ezo​ \\ \hline \\[-1em]
\end{tabular}

\section{1871-1888}
\begin{tabular}{p{1.8cm} | p{3.0cm} | p{4.5cm} | p{7.5cm}}
1871-73 & 岩倉使節団、征韓論 & いわくらしせつだん,せいかんろん & Iwakura Mission, Seikanron \\ \hline \\[-1em]
1874 & 台湾出兵 & たいわんしゅっぺい & Japanese Expedition of 1874 to Taiwan \\ \hline \\[-1em]
1875 & 樺太・千島交換条約 & からふとちしまこうかんじょうやく & Treaty of Saint Petersburg (1875) \\ \hline \\[-1em]
1875 & 江華島事件 & こうかとうじけん & Ganghwa Island incident \\ \hline \\[-1em]
1875 & & ハインリッヒ・エドムント・ナウマン & Heinrich Edmund Naumann \\ \hline \\[-1em]
1876 & 松野クララ & クララ・チーデルマン & Clara Louise Zitelmann (Matsuno Kurara)\\ \hline \\[-1em]
1876 & 日朝修好条規 & にっちょうしゅうこうじょうき & Ganghwa Japan–Korea Treaty of 1876 \\ \hline \\[-1em]
1876 & & エルヴィン・フォン・ベルツ & Erwin Bälz \\ \hline \\[-1em]
1877 & 西南戦争 & せいなんせんそう & Satsuma Rebellion \\ \hline \\[-1em]
1882 & 壬午事変 & みずのえうまじへん & 壬午事変 \\ \hline \\[-1em]
1882 & & パウル・ゲオルク・フォン・メレンドルフ & Paul Georg von Möllendorff \\ \hline \\[-1em]
1882 & 三国同盟 &  さんごくどうめい & Triple Alliance (1882) \\ \hline \\[-1em]
1884-85 & 清仏戦争 & しんふつせんそう & Sino-French War (1884-1885) \\ \hline \\[-1em]
1884 & 甲申政変 & きのえさるせいへん & Gapsin Coup \\ \hline \\[-1em]
1885 & 日中、天津条約 & ごにちちゅう, てんしんじょうやく & Sino-Japanese, Treaty of Tianjin (1858, 1885) \\ \hline \\[-1em]
1888 & 北洋艦隊 & ほくようかんたい & Beiyang Fleet
\end{tabular}

\section{1889-1897}
\begin{tabular}{p{1.8cm} |p{3.0cm} | p{4.5cm} | p{8.5cm}}
1889,90 & 大日本帝国憲法 & だいにっぽんていこくけんぽう & Constitution of the Empire of Japan (Meiji era) \\ \hline \\[-1em]
1894 & 露仏同盟 & ろふつどうめい & Franco-Russian Alliance \\ \hline \\[-1em]
1891-1916 & シベリア鉄道 & シベリアてつどう & Trans-Siberian Railway \\ \hline \\[-1em]
1894 & 甲午農民戦争 &  こうごのうみんせんそう & Donghak Peasant Revolution \\ \hline \\[-1em]
1894-95 & 日清戦争 & にっしんせんそう & First Sino-Japanese war (1894-1895) \\ \hline \\[-1em]
1894-95 & 下関条約 & しものせきじょうやく & Treaty of Shimonoseki \\ \hline \\[-1em]
1894-95 & 三国干渉 &  さんごくかんしょう & Triple Intervention (diplomatic intervention by Russia, Germany and France over the terms of the Treaty of Shimonoseki) \\ \hline \\[-1em]
1895 & 閔妃暗殺	& びんひあんさつ  & Queen Min /Empress Myeongseong -'s assassination \\ \hline \\[-1em]
1897-1910 & 大韓帝国 &  だいかんていこく & Korean Empire (1897-1910) %\\ \hline \\[-1em]
\end{tabular}


\section{1898-1912}
\begin{tabular}{p{1.8cm} | p{3.4cm} | p{5.91cm} | p{7.0cm}}
1898 & ドイツ膠州湾租借 & ドイツこうしゅうわんそしゃく & Kiautschou Bay concession \\ \hline \\[-1em]
1900 & 義和団の乱 & ぎわだんらん & Boxer Rebellion / Yihetuan Movement​ \\ \hline \\[-1em]
1902 & 日英同盟 & にちえいどうめい & Anglo-Japanese Alliance (1902-1923)​ \\ \hline \\[-1em]
1904-05 & 日露戦争 & にちろせんそう & Russo-Japanese War (1904-5)​ \\ \hline \\[-1em]
1904-05 & ポーツマス条約 & ポーツマスじょうやく & Treaty of Portsmouth (ending the Russo-Japanese War, 1905)​ \\ \hline \\[-1em]
1906 & 関東軍 & かんとうぐん & Kwantung Army (Japanese armed forces in Manchukuo)​ \\ \hline \\[-1em]
1919 & 正式発足 & せいしきほっそく & formal inaugoration \\ \hline \\[-1em]
1906-45 & 南満州鉄道 & みなみまんしゅうてつどう & South Manchuria Railway \\ \hline \\[-1em]
1907 & ハーグ密使事件 & ハーグみっしじけん & Hague Secret Emissary Affair \\ \hline \\[-1em]
1910 & 韓国併合 & かんこくへいごう & annexation of Korea (by Japan; 1910-1945)​ \\ \hline \\[-1em]
1911-12 & 辛亥革命 & しんがいかくめい & Xinhai Revolution (1911)​ \\ \hline \\[-1em]
1912 & 中華民国成立 & ちゅうかみんこくせいりつ & foundation of the Republic of China (Taiwan)​ \\ \hline \\[-1em]
\end{tabular}

\end{document}
