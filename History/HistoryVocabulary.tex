\documentclass{article}
\usepackage{xeCJK}
\usepackage{enumitem}
\usepackage{titlesec}
\usepackage[dvipsnames]{xcolor}
\usepackage{fancyhdr}
\usepackage{geometry}
\usepackage[CJK,overlap]{ruby}
\geometry{ a4paper, %total={170mm,257mm},
	left=10mm, top=10mm, right=10mm, bottom=12mm, } \pagestyle{fancy}
\lhead{}
\chead{}
\rhead{}
\lfoot{}
\cfoot{\thepage}
\rfoot{v0.1} %(04.MAY.2017)


\renewcommand{\rubysep}{0.1ex}%3.9ex}
\makeatletter\renewcommand\paragraph{\@startsection{paragraph}{4}{\z@}{-0.5ex\@plus -.1ex \@minus -.25ex}{-1em}{\normalfont\normalsize\bfseries}}
\renewcommand\subparagraph{\@startsection{subparagraph}{5}{\z@}{-0.1ex\@plus -.1ex \@minus -.25ex}{-1em}{\normalfont\normalsize\bfseries}}\makeatother
\renewcommand{\headrulewidth}{0pt} \renewcommand{\footrulewidth}{0pt}
\newcommand\tab[1][1cm]{\hspace*{#1}} \newcommand\tabi[1][0.05cm]{\hspace*{#1}}\newcommand\tabni[1][0.2cm]{\hspace*{#1}}
\newcommand\tabyon[1][0.4cm]{\hspace*{#1}}\newcommand\tabroku[1][0.6cm]{\hspace*{#1}}
\XeTeXlinebreaklocale "ja" \XeTeXlinebreakskip=0em plus 0.1em minus 0.01em \setlist{nolistsep}
\titlespacing*{\section}{0pt}{1.5ex plus 1ex minus .2ex}{1.5ex plus .2ex}
\titlespacing*{\subsection}{0pt}{0.5ex plus 1ex minus .2ex}{0.5ex plus .2ex}
\setlength{\parindent}{0pt}\setCJKmainfont{ipaexm.ttf}\setcounter{tocdepth}{5}

\begin{document} \large

\section*{歴史(れきし) History}
\begin{tabular}{l | l | p{10.5cm}}
時代 & じだい & period; epoch \\ \hline \\[-1em]
世紀 & せいき & century \\ \hline \\[-1em]
縄文時代 & じょうもんじだい & Jōmon period​ \\ \hline \\[-1em]
弥生時代 & やよいじだい & Yayoi period (c. 300 BCE-300 CE) \\ \hline \\[-1em]
卑弥呼 & ぴみく & Pimiko/Himiko (175-248 CE) was an obscure shaman queen of Yamataikoku in ancient Wa \\ \hline \\[-1em]
大和時代 & やまとじだい & Yamato period (300-550 CE) \\ \hline \\[-1em]​
天皇 & てんのう & Emperor of Japan​ \\ \hline \\[-1em]
飛鳥時代 & あすかじだい & Asuka period (550-710 CE)​ \\ \hline \\[-1em]
奈良時代 & ならじだい & Nara period (710-794 CE)​ \\ \hline \\[-1em]
平安時代 & へいあんじだい & Heian period (794-1185)​ \\ \hline \\[-1em]
平安京(京都) & へいあんきょう(きょうと) & Heian-kyō​ (Kyoto) \\ \hline \\[-1em]
鎌倉時代 & かまくらじだい & Kamakura period (1185-1333 CE)​ \\ \hline \\[-1em]
将軍 & しょうぐん & general; shogun \\ \hline \\[-1em]
侍 & さむらい & samurai; warrior \\ \hline \\[-1em]
武士 & ぶし & samurai; warrior \\ \hline \\[-1em]
神風 & かみかぜ & Divine Wind \\ \hline \\[-1em]
南北朝時代 & なんぼくちょうじだい & period of the northern and southern dynasties (Japan, 1336-92)​ \\ \hline \\[-1em]
室町時代 & むろまちじだい & Muromachi period (1333-1573 CE)​ \\ \hline \\[-1em]
安土桃山時代 & あづちももやまじだい & Azuchi-Momoyama period (approx. 1558-1600 CE) \\ \hline \\[-1em]​
江戸時代 & えどじだい & Edo period (1603-1868 CE)​ \\ \hline \\[-1em]
江戸(東京) & えど(とうきょう) & Edo(Tokyo) \\ \hline \\[-1em]
長崎 & ながさき & Nagasaki \\ \hline \\[-1em]
明治時代 & めいじじだい & Meiji period (1868-1912) \\ \hline \\[-1em]​
大正時代 & たいしょうじだい & Taisho period (1912-1926 CE)​ \\ \hline \\[-1em]
昭和時代 & しょうわじだい & Showa period (1926-1989 CE)​ \\ \hline \\[-1em]
平成時代 & へいせいじだい & Heisei period (1989-)​ %\\ \hline \\[-1em]
\end{tabular}


\end{document}
