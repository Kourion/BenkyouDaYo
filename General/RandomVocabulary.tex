\documentclass{article}
\usepackage{xeCJK}
\usepackage{enumitem}
\usepackage{titlesec}
\usepackage[dvipsnames]{xcolor}
\usepackage{fancyhdr}
\usepackage{geometry}
\usepackage[CJK,overlap]{ruby}
 \geometry{
 a4paper,
 %total={170mm,257mm},
 left=5mm,
 top=8mm,
 right=5mm,
 bottom=12mm,
 }
 
\renewcommand{\rubysep}{0.1ex}%3.9ex}
\makeatletter
\renewcommand\paragraph{\@startsection{paragraph}{4}{\z@}%
            {-0.5ex\@plus -.1ex \@minus -.25ex}%
            {-1em}%
            {\normalfont\normalsize\bfseries}}
\renewcommand\subparagraph{\@startsection{subparagraph}{5}{\z@}%
            {-0.1ex\@plus -.1ex \@minus -.25ex}%
            {-1em}%
            {\normalfont\normalsize\bfseries}}
\makeatother

\pagestyle{fancy}
\lhead{}
\chead{}
\rhead{}
\lfoot{}
\cfoot{\thepage}
\rfoot{v0.5}

\newcommand\tab[1][1cm]{\hspace*{#1}} 
\newcommand\tabi[1][0.05cm]{\hspace*{#1}}
\newcommand\tabni[1][0.2cm]{\hspace*{#1}}
\newcommand\tabyon[1][0.4cm]{\hspace*{#1}}
\newcommand\tabroku[1][0.6cm]{\hspace*{#1}}

\XeTeXlinebreaklocale "ja"
\XeTeXlinebreakskip=0em plus 0.1em minus 0.01em

\setlist{nolistsep}
\titlespacing*{\section}
{0pt}{1.5ex plus 1ex minus .2ex}{1.5ex plus .2ex}
\titlespacing*{\subsection}
{0pt}{0.5ex plus 1ex minus .2ex}{0.5ex plus .2ex}

\setlength{\parindent}{0pt}
\setCJKmainfont{ipaexm.ttf}
\setcounter{tocdepth}{5}

\begin{document}
\part*{Random Vocabulary}
\begin{tabular}{ l | l p{14cm} }
Kanji&Kana&Translation\\ \hline\\[-1em]
予定 &よてい &--> plan /schedule \\ \hline\\[-1em]
発音 & はつおん	&--> pronunciation 	\\ \hline\\[-1em]
惜しい &おしい &--> regrettable 							\\ \hline\\[-1em]
---- &きざ &--> smug /pompous /snobby					\\ \hline\\[-1em]
---- &こぼす &--> to spill /to drop /to shed (tears)		\\ \hline\\[-1em]
見せる &みせる &--> to show /to display					\\ \hline\\[-1em]
込む &こむ &--> to be crowded							\\ \hline\\[-1em]
頑張れ &がんばれ &--> go for it /keep at it /do your best	\\ \hline\\[-1em]
頑張る &がんばる &--> to keep at it /to do one's best		\\ \hline\\[-1em]
手袋 &てぶくろ &--> gloves /mittens						\\ \hline\\[-1em]
あだ名 &あだな &--> nickname								\\ \hline\\[-1em]
真面目 &まじめ &--> serious /honest /sober /grave			\\ \hline\\[-1em]
一杯 &いっぱい &--> full									\\ \hline\\[-1em]
酔っ払い &よっぱらい &--> drunkard								\\ \hline\\[-1em]
酔っ払う &よっぱらう &--> to get drunk						\\ \hline\\[-1em]
未定 &みてい &--> not yet fixed; undecided; pending		\\ \hline\\[-1em]
%未定 &みてい &--> not yet fixed; undecided; pending   \\ \hline\\[-1em]
香辛料 &こうしんりょう &--> spices /condiment						\\ \hline\\[-1em]
親友 &しんゆう &--> close friend							\\ \hline\\[-1em]
他 &ほか &--> other (place, thing) /the rest			\\ \hline\\[-1em]
助ける &たすける &--> to help /to save /to rescue        \\ \hline\\[-1em]
法学 &ほうがく &--> law; jurisprudence  \\ \hline\\[-1em]
航空工学 &こうくうこうがく &--> aeronautical engineering   \\ \hline\\[-1em]
情報学 &じょうほうがく &--> informatics   \\ \hline\\[-1em]
煩わしい &わずらわしい &--> troublesome; annoying; complicated   \\ \hline\\[-1em]
---- &めんどくさい &--> bothersome; tiresome; bother to do   \\ \hline\\[-1em]
%じゃが芋 &じゃがいも &--> potato   \\ \hline\\[-1em] //Now as Katakana
性行為 &せいこうい &--> sex act; intercourse   \\ \hline\\[-1em]
駒 &こま &--> piece (in shogi, chess, etc.)   \\ \hline\\[-1em]
詩人 &しじん &--> poet   \\ \hline\\[-1em]
吸血鬼 &きゅうけつき &--> vampire; bloodsucker   \\ \hline\\[-1em]
血液 &けつえき &--> blood   \\ \hline\\[-1em]
学ぶ &まなぶ &--> to study (in depth); to learn; to take lessons in   \\ \hline\\[-1em]
単語 &たんご &--> word; vocabulary;   \\ \hline\\[-1em]
---- &しょうがない &--> (しゃあない) it can't be helped; it is inevitable; c'est la vie   \\ \hline\\[-1em]
理由 &りゆう &--> reason; pretext; motive   \\ \hline\\[-1em]
混ぜる &まぜる &--> to mix; to stir; to blend   \\ \hline\\[-1em]
---- &ませる &--> to be mature for one's age; to seem grown-up; to seem mature   \\ \hline\\[-1em]
侮辱 &ぶじょく &--> insult; contempt; slight   \\ \hline\\[-1em]
---- &ばいた &--> prostitute; whore   \\ \hline\\[-1em]
---- &なるほど &--> I see; That's right!; Indeed   \\ \hline\\[-1em]
最悪 &さいあく &--> the worst   \\ \hline\\[-1em]
雑 &ざつ &--> rough; crude; sloppy; messy   \\ \hline\\[-1em]
気持ちいい &きもちいい &--> good feeling; feeling good \\ \hline\\[-1em]
気持ち &きもち &--> feeling; sensation; mood \\ \hline\\[-1em]
辞める &やめる &--> to resign; to retire; to quit \\ \hline\\[-1em]
難い &にくい &--> difficult to ...; hard to ... \\ \hline\\[-1em]
---- &しにくい &--> hard to do \\ \hline\\[-1em]
邪魔 &じゃま &--> hindrance; obstacle; nuisance \\ \hline\\[-1em]
真っ只中 &まっただなか &--> right in the midst of; right at the height of \\ \hline\\[-1em]
---- &そがれる &--> to be dampened; to be discouraged; to be weakened; to be reduced   \\ \hline\\[-1em]
\end{tabular} \\
\begin{tabular}{ l | l p{14cm}  }
喫茶店 &きっさてん &--> coffee shop; tearoom; coffee lounge; coffeehouse; cafe \\ \hline\\[-1em]
やる気 &やるき &--> willingness (e.g. to do something); eagerness; motivation; inspiration; determination; high aspirations \\ \hline\\[-1em]
触る &さわる &--> to touch; to feel \\ \hline\\[-1em]
会計 &かいけい &--> account; finance; accountant; treasurer; paymaster; reckoning; bill \\ \hline\\[-1em]
飽きる &あきる &--> to get tired of; to lose interest in; to have enough \\ \hline\\[-1em]
---- &なんだっけ &--> what is it? \\ \hline\\[-1em]
帰国 &きこく &--> return to one's country \\ \hline\\[-1em]
請う &こう &--> to beg; to ask; to request; to invite \\ \hline\\[-1em]
---- &けど &--> but; however; although \\ \hline\\[-1em]
大事 &だいじ &--> important; valuable; serious matter \\ \hline\\[-1em]
世話 &せわ &--> looking after; help; aid; assistance \\ \hline\\[-1em]
確かに &たしかに &--> surely; certainly \\ \hline\\[-1em]
単行本 &たんこうぼん &--> special book; separate volume; book of lectures \\ \hline\\[-1em]
---- &やっぱり &--> too; also; likewise; either; still; as before; even so; either way; nonetheless; in any event; all the same \\ \hline\\[-1em]
辞める &やめる &--> to resign; to retire; to quit; to leave (one's job, etc.) \\ \hline\\[-1em]
---- &やる &--> to do; to undertake; to perform; to play (a game); to study \\ \hline\\[-1em]
謎 &なぞ &--> riddle; puzzle; enigma; mystery \\ \hline\\[-1em]
無神論者 &むしんろんしゃ &--> atheist \\ \hline\\[-1em]
怠惰 &たいだ &--> laziness; idleness; sloth \\ \hline\\[-1em]
熟語 &じゅくご &--> idiom; idiomatic phrase \\ \hline\\[-1em]
---- &だけ &--> as (e.g. as much as /as soon as); only; just; merely; \\ \hline\\[-1em]
言葉 &ことば &--> language; dialect \\ \hline\\[-1em]
---- &しゃべる &--> to talk; to chat; to chatter \\ \hline\\[-1em]
明々後日 & しあさって &--> in three days' time; two days after tomorrow \\ \hline\\[-1em]
暗記 & あんき &--> memorisation; learning by heart \\ \hline\\[-1em]
付き合う & つきあう &--> to associate with; to keep company with; to go out with; to accompany someone \\ \hline\\[-1em]
翻訳 & ほんやく &--> translation \\ \hline\\[-1em]
慕う & したう &--> to adore /to idolize /to yearn for\\ \hline\\[-1em]
---- & もうすぐ &--> very soon \\ \hline\\[-1em]
雑学 & ざつがく &--> trivia \\ \hline\\[-1em]
方言 & ほうげん &--> dialect \\ \hline\\[-1em]
嫌 & いや &--> disagreeable; detestable; unpleasant; reluctant \\ \hline\\[-1em]
怒る & おこる &--> to get angry; to get mad \\ \hline\\[-1em]
諦める & あきらめる &--> to give up; to abandon (hope, plans); to resign oneself (to) \\ \hline\\[-1em]
頼む & たのむ &--> to request; to beg; to ask /to entrust \\ \hline\\[-1em]
黙る & だまる &--> to be silent \\ \hline\\[-1em]
出かける & でかける &--> to go out (e.g. on an excursion or outing) /to leave /to depart \\ \hline\\[-1em]
---- & めんどくさがり &--> tending to find things bothersome; to find most things bothersome \\ \hline\\[-1em]
漏れる & もれる &--> to leak out; to escape; to come through; to shine through; to filter out \\ \hline\\[-1em]
鳩 & はと &--> dove \\ \hline\\[-1em]
---- & ともかく &--> anyhow; anyway; somehow or other; generally speaking; in any case; \\ \hline\\[-1em]
結果 & けっか &--> result; consequence; outcome; effect \\ \hline\\[-1em]
---- & はちみつ &--> honey \\ \hline\\[-1em]
支払い & しはらい &--> payment \\ \hline\\[-1em]
丁度いい & ちょうどいい  &--> just right (time, size, length, etc.) \\ \hline\\[-1em]
壊れる & こわれる &--> to be broken \\ \hline\\[-1em]
---- & つまらない &--> dull; uninteresting; boring; tedious \\ \hline\\[-1em]
合唱 & がっしょう &--> chorus \\ \hline\\[-1em]
準備 & じゅんび &--> preparation \\ \hline\\[-1em]
怠け者 & なまけもの &--> lazy person\\ \hline\\[-1em]
怠ける & なまける &--> to be idle \\ \hline\\[-1em]
洒落 & しゃれ &--> pun-style-witz\\ \hline\\[-1em]
\end{tabular} \\
\begin{tabular}{ l | l p{14cm}  }
武術 & ぶじゅつ &--> martial arts \\ \hline\\[-1em]
---- & めっちゃ &--> very; extremely; excessively \\ \hline\\[-1em]
意味がない & いみがない &--> doesn't make sense; having no point \\ \hline\\[-1em]
燃える & もえる &--> to burn\\ \hline\\[-1em]
---- & どちらも &--> both /either\\ \hline\\[-1em]
---- & やばい &--> awful; terrible; crap /dangerous; risky /terrific; amazing; cool\\ \hline\\[-1em]
---- & まじ &--> seriously\\ \hline\\[-1em]
---- & ひどい  &--> cruel; heartless; hard; harsh; severe /violent; intense /awful /outrageous; unfair; unjust \\ \hline\\[-1em]
流れ星 & ながれぼし &--> shooting star \\ \hline\\[-1em]
有罪 & ゆうざい &--> guilt \\ \hline\\[-1em]
罪 & つみ &--> crime; sin; wrongdoing; indiscretion, fault; responsibility; culpability\\ \hline\\[-1em]
---- & やってみる &--> to have a go; to try and do (something); to take a chance with something \\ \hline\\[-1em]
---- & しつこい &-->  insistent; obstinate; persistent; /too rich (taste, etc.); fatty; heavy; \\ \hline\\[-1em]
失礼 & しつれい &--> discourtesy; impoliteness /to be rude /Excuse me / Goodbye\\ \hline\\[-1em]
冗談 & じょうだん &--> jest; joke; funny story \\ \hline\\[-1em]
別 & べつ &--> distinction; difference; different; another; particular; separate; extra; exception \\ \hline\\[-1em]
---- & ときめく &--> to throb /to flutter \\ \hline\\[-1em]
確からしい & たしから &--> probable; likely \\ \hline\\[-1em]
---- & かしこまりました &--> certainly! \\ \hline\\[-1em]
呪い & のろい &--> curse; spell; malediction \\ \hline\\[-1em]
---- & いいかも &--> desirable prey; easy game; easy mark; easy prey; easy target \\ \hline\\[-1em]
---- & ペラペラ &--> fluency /fluent \\ \hline\\[-1em]
---- & サバ &--> makrele \\ \hline\\[-1em]
---- & やった  &--> hooray /yowzer! /whoopee! /yes! \\ \hline\\[-1em]
---- & ジャガイモ &--> potato \\ \hline\\[-1em]
気まぐれ & きまぐれ &--> whim; caprice; whimsy; fickle; moody; uneven temper \\ \hline\\[-1em]
起こる & おこる &--> to occur; to happen \\ \hline\\[-1em]
---- & おしゃべり &--> chatty; talkative; chatterbox; blabbermouth; chattering; talk; idle talk; chat; chitchat; gossip \\ \hline\\[-1em]
---- & スピーカー &--> speaker /loudspeaker \\ \hline\\[-1em]
決まる & きまる &--> to be decided; to be settled \\ \hline\\[-1em]
決める & きめる &--> to decide; to choose; to determine; to make up one's mind \\ \hline\\[-1em]
設計図 & せっけいず &--> blueprint /plan \\ \hline\\[-1em]
生徒 & せいと &--> pupil \\ \hline\\[-1em]
---- & ほら /ほれ &--> look!; look out!; hey!; look at me!; there you are! \\ \hline\\[-1em]
実 & じつ &--> truth /reality \\ \hline\\[-1em]
言い訳 & いいわけ &--> excuse \\ \hline\\[-1em]
訪ねる & たずねる &--> to visit \\ \hline\\[-1em]
---- & だめ /あかん &--> no good \\ \hline\\[-1em]
他 & ほか &--> other \\ \hline\\[-1em]
親友 & しんゆう &--> best friend \\ \hline\\[-1em]
大人 & おとな &--> adult \\ \hline\\[-1em]
大人っぽい & おとなっぽい &--> adult-like \\ \hline\\[-1em]
意地悪い & いじわるい &--> ill-natured; mean; nasty; sadistic; cruel; mean-spirited \\ \hline\\[-1em]
腹黒い & はらぐろい &--> mean; malicious; scheming; wicked; black-hearted \\ \hline\\[-1em]
語る & かたる &--> to talk about; to speak of; to tell; to narrate \\ \hline\\[-1em]
懐かしい & なつかしい &--> dear; desired; missed \\ \hline\\[-1em]
似合う & にあう &--> to suit; to match; to become; to be like \\ \hline\\[-1em]
形 & かたち &--> form; shape; figure \\ \hline\\[-1em]
嘘つき & うそつき &--> liar \\ \hline\\[-1em]
---- & びっくり &--> surprised /shocked \\ \hline\\[-1em]
工事 & こうじ &--> construction work \\ \hline\\[-1em]
最初 & さいしょ &--> beginning; outset; first; onset \\ \hline\\[-1em]
\end{tabular}\\
\begin{tabular}{ l | l p{14cm}  }
太陽 & たいよう &--> sun, solar \\ \hline\\[-1em]
昨今 & さっこん &--> recently /nowadays \\ \hline\\[-1em]
洒落る & しゃれる &--> stylish, used as cool in te-iru form しゃれてる \\ \hline\\[-1em]
湿気る & しける &--> moist/damp, used as uncool in te-iru slang form --> しけてる \\ \hline\\[-1em]
---- & いかれる &--> to break down, used in slang te-iru form as 'you are crazy' --> いかれてる \\ \hline\\[-1em]
---- & ウケる &--> hilarious \\ \hline\\[-1em]
超 & ちょう &--> super-; ultra-; hyper- \\ \hline\\[-1em]
---- & マジで &--> seriously, honestly, really \\ \hline\\[-1em]
---- & ガチ &--> seriously, honestly \\ \hline\\[-1em]
---- & ヤバい &--> crap/cool \\ \hline\\[-1em]
---- & ウザい &--> annoying \\ \hline\\[-1em]
---- & キモい &--> gross \\ \hline\\[-1em]
---- & ダサい &--> lame \\ \hline\\[-1em]
---- & ブス &--> ugly (very harsh) \\ \hline\\[-1em]
---- & ズルい &--> sneaky /not fair /you suck \\ \hline\\[-1em]
---- & びみょう &--> kinda sucks /not good (used to describe taste) \\ \hline\\[-1em]
---- & さいてい &--> the worst (for humans) \\ \hline\\[-1em]
神経質 & しんけいしつ &--> sensitive fussy /worried about petty things \\ \hline\\[-1em]
相槌 & あいずち &--> back-channeling; interjections indicating that one is paying attention \\ \hline\\[-1em]
案内人 & あんないにん &--> guide \\ \hline\\[-1em]
お辞儀 & おじぎ &--> bow /bowing \\ \hline\\[-1em]
勿論 & もちろん &--> of course /certainly /naturally \\ \hline\\[-1em]
看板 & かんばん &--> sign \\ \hline\\[-1em]
塩 & しお &--> salt \\ \hline\\[-1em]
主催者 & しゅさいしゃ &--> sponsor; promoter; impresario; organizer; organiser \\ \hline\\[-1em]
招待 & しょうたい &--> invitation \\ \hline\\[-1em]
散らかる & ちらかる &--> to be in disorder; to lie scattered around \\ \hline\\[-1em]
緊張 & きんちょう &--> tension; mental strain; nervousness \\ \hline\\[-1em]
色々 & いろいろ &--> various \\ \hline\\[-1em]
---- & むちゃぶり &--> unreasonable request \\ \hline\\[-1em]
ほろ酔い & ほろよい &--> slight intoxication; being tipsy \\ \hline\\[-1em]
覚ます & さます &--> to awaken \\ \hline\\[-1em]
有用 & ゆうよう &--> useful /helpful \\ \hline\\[-1em]
結構 & けっこう &--> fine /splendid \\ \hline\\[-1em]
偉い & えらい &--> admirable; remarkable \\ \hline\\[-1em]
貧弱 & ひんじゃく &--> poor; meagre; meager; insubstantial \\ \hline\\[-1em]
希望 & きぼう &--> hope; wish; aspiration \\ \hline\\[-1em]
---- & ウザイ &--> annoying /noisy \\ \hline\\[-1em]
雪だるま & ゆきだるま &--> snowman \\ \hline\\[-1em]
誘う & さそう &--> to invite; to ask; to call (for) \\ \hline\\[-1em]
---- & さすが &--> as one would expect \\ \hline\\[-1em]
見つける & みつける &--> to discover; to find (e.g. an error in a book); to come across; to detect; to spot \\ \hline\\[-1em]
素晴らしい & すばらしい &--> wonderful; splendid; magnificent \\ \hline\\[-1em]
見た目 & みため &--> appearance \\ \hline\\[-1em]
心配 & しんぱい &--> worry, concern; anxiety \\ \hline\\[-1em]
知識 & ちしき &--> knowledge, information \\ \hline\\[-1em]
特技 & とくぎ &--> special skill \\ \hline\\[-1em]
お互いに & おたがいに &--> mutually; with each other; reciprocally \\ \hline\\[-1em]
過労死 & かろうし &--> death from overwork \\ \hline\\[-1em]
不味い & まずい &--> unappetising; unappetizing; unpleasant \\ \hline\\[-1em]
感覚器官 & かんかくきかん &--> Sinnesorgan \\ \hline\\[-1em]
記憶の手がかり & きおくのてがかり &--> Eselsbrücke \\ \hline\\[-1em]
\end{tabular}\\
\begin{tabular}{ l | l p{14cm}  }
決まり文句 & きまりもんく &--> cliche \\ \hline\\[-1em]
講義室 & こうぎしつ &--> Hörsaal \\ \hline\\[-1em]
講義 & こうぎ &--> lecture \\ \hline\\[-1em]
思いつき & おもいつき &--> Das ist eine gute Idee! \\ \hline\\[-1em]
---- & かっこいい &--> attractive; good-looking; stylish; cool; smooth; neat; with-it; groovy \\ \hline\\[-1em]
格好 & かっこう &--> figure /shape /form \\ \hline\\[-1em]
展覧会 & てんらんかい &--> (art-) exhibition \\ \hline\\[-1em]
愉快 & ゆかい &--> pleasant /happy \\ \hline\\[-1em]
地味 & じみ &--> plain /simple /conservative \\ \hline\\[-1em]
異文化 & いぶんか &--> foreign culture \\ \hline\\[-1em]
---- & ぶさいく &--> clumsy (work) /ugly /plain \\ \hline\\[-1em]
主観的 & しゅかんてき &--> subjective \\ \hline\\[-1em]
客観的 & きゃっかんてき &--> objective \\ \hline\\[-1em]
不遜 & ふそん &--> arrogance /insolence /disrespect \\ \hline\\[-1em]
誇張 & こちょう &--> exaggeration \\ \hline\\[-1em]
誇張法 & こちょうほう &--> hyperbole /hybris \\ \hline\\[-1em]
余裕 & よゆう &--> leeway /flexibility \\ \hline\\[-1em]
表現 & ひょうげん &--> expression /presentation \\ \hline\\[-1em]
示す & しめす &--> to tell; to exemplify; to make apparent \\ \hline\\[-1em]
小惑星 & しょうわくせい &--> asteroid \\ \hline\\[-1em]
預ける & あずける &--> to give into custody; to leave (a child) in the care of; to entrust; to deposit \\ \hline\\[-1em]
余計 & よけい &--> too many; too much; excessive /unnecessary; needless; uncalled-for; otiose /abundantly; excessively; needlessly \\ \hline\\[-1em]
瓦 & かわら &--> roof tile \\ \hline\\[-1em]
脇 & わき &--> armpit \\ \hline\\[-1em]
愚か & おろか &--> foolish \\ \hline\\[-1em]
隕石 & いんせき &--> meteorite \\ \hline\\[-1em]
望遠鏡 & ぼうえんきょう &--> telescope \\ \hline\\[-1em]
悩む & なやむ &--> to be worried; to be troubled \\ \hline\\[-1em]
突く & つく &--> to prick /to stab \\ \hline\\[-1em]
突き刺さる & つきささる &--> to stick into; to pierce; to run into \\ \hline\\[-1em]
突き刺す & つきさす &--> to stab; to pierce; to thrust \\ \hline\\[-1em]
杭 & くい &--> stake /post /stump \\ \hline\\[-1em]
調べる & しらべる &--> to examine /to investigate /to study \\ \hline\\[-1em]
本番 & ほんばん &--> performance \\ \hline\\[-1em]
成る & なる &--> to become; to get; to grow; to be; to reach; to attain \\ \hline\\[-1em]
焼ける & やける &--> to burn; to be roasted; to be heated \\ \hline\\[-1em]
不可 & ふか &--> wrong; bad; improper; unjustifiable; inadvisable /not allowed; not possible /failing grade \\ \hline\\[-1em]
不合格 & ふごうかく &--> failure; rejection; disqualification \\ \hline\\[-1em]
毎月 & まいつき &--> every month \\ \hline\\[-1em]
盗む & ぬすむ &--> to steal \\ \hline\\[-1em]
黙る & だまる &--> to be silent \\ \hline\\[-1em]
---- & したっけ &--> good-bye; see you later; bye-bye \\ \hline\\[-1em]
---- & つらい &--> painful; bitter; heart-breaking; /cruel /cold /harsh \\ \hline\\[-1em]
専門用語 & せんもんようご &--> technical term \\ \hline\\[-1em]
気分 & きぶん &--> feeling /mood \\ \hline\\[-1em]
毛皮 & けがわ &--> fur /skin /pelt \\ \hline\\[-1em]
警告 & けいこく&--> warning, advice \\ \hline\\[-1em]
---- & ばかり &--> only /merely /nothing but \\ \hline\\[-1em]
---- & そうやって &--> like that /that way /thus \\ \hline\\[-1em]
見分ける & みわける &--> to distinguish; to recognize; to recognise; to tell apart; to differentiate (both figuratively and visually) \\ \hline\\[-1em]
役に立つ & やくにたつ &--> to be helpful /to be useful \\ \hline\\[-1em]
不評 & ふひょう &--> bad reputation /disgrace /unpopularity \\ \hline\\[-1em]
\end{tabular}\\
\begin{tabular}{ l | l p{14cm}  }
---- & らしい &--> seemingly /-ish \\ \hline\\[-1em]
結果 & けっか &--> result /consequence /outcome /effect \\ \hline\\[-1em]
決まり & きまり &--> rule; regulation \\ \hline\\[-1em]
---- & うっとうしい &--> irritating /troublesome, gloomy /depressing \\ \hline\\[-1em]
熱心 & ねっしん &--> zeal /enthusiasm \\ \hline\\[-1em]
---- & めちゃくちゃ &--> absurd /preposterous /incoherent /extreme /reckless /disorder /confussion \\ \hline\\[-1em]
---- & バカバカしい &--> absurd /ridiculous /silly \\ \hline\\[-1em]
相談 & そうだん &--> discussion /consultation \\ \hline\\[-1em]
力になる & ちからになる &--> to be helpful; to be dependable \\ \hline\\[-1em]
冷静 & れいせい &--> calm; composure; coolness; serenity \\ \hline\\[-1em]
比較的 & ひかくてき &--> comparative; relative \\ \hline\\[-1em]
重宝 & ちょうほう &--> convenient; useful; handy; helpful \\ \hline\\[-1em]
関係 & かんけい &--> relation(ship) /connection \\ \hline\\[-1em]
活気 & かっき &--> energy /liveliness /activity \\ \hline\\[-1em]
---- & ググレカス &--> google it, scum; just fucking google it; JFGI \\ \hline\\[-1em]
法則 & ほうそく &--> law /rule \\ \hline\\[-1em]
破る & やぶる &--> to tear; to break; to destroy \\ \hline\\[-1em]
秘密 & ひみつ &--> secret; secrecy; confidentiality; privacy \\ \hline\\[-1em]
勤勉 & きんべん &--> diligent; industrious \\ \hline\\[-1em]
---- & なんとかなる &--> to be able to manage somehow or another \\ \hline\\[-1em]
褒める & ほめる &--> to praise; to admire; to speak well of someone \\ \hline\\[-1em]
家政婦 & かせいふ &--> housekeeper \\ \hline\\[-1em]
---- & とうとう &--> finally /at last /ultimately /in the end \\ \hline\\[-1em]
---- & ださい &--> uncool \\ \hline\\[-1em]
ご褒美 & ごほうび &--> reward /prize \\ \hline\\[-1em]
痒がる & かゆがる &--> to feel itchy /to itch \\ \hline\\[-1em]
---- & くすぐる &--> to tickle \\ \hline\\[-1em]
---- & そんなもの &--> that's the way it is; such is life \\ \hline\\[-1em]
---- & ヨット &--> yacht \\ \hline\\[-1em]
最悪 & さいあく &--> the worst \\ \hline\\[-1em]
流行 & はやり &--> fashion; fad; vogue; craze \\ \hline\\[-1em]
---- & いいな &--> I'm jealous \\ \hline\\[-1em]
前提 & ぜんてい &--> preamble; premise; reason; prerequisite; condition; assumption; hypothesis; given \\ \hline\\[-1em]
寄り添う & よりそう &--> to get close; to cuddle close together; to nestle close to; to snuggle up to \\ \hline\\[-1em]
---- & せっかく &--> with trouble; at great pains /rare; valuable; precious; long-awaited \\ \hline\\[-1em]
膨らむ & ふくらむ &--> to expand; to swell (out); to get big; to become inflated \\ \hline\\[-1em]
過保護 & かほご &--> overprotective; excessive care; overprotection; overprotectiveness \\ \hline\\[-1em]
 & &--> \\ \hline\\[-1em]
 & &--> \\ \hline\\[-1em]
\end{tabular}

%\part*{Phrases and Idioms}
%\begin{tabular}{ l | l p{10cm} }
%お邪魔します  &おじゃまします     &--> excuse me for disturbing (interrupting) you; \\ \hline\\[-1em]
%触らぬ神に祟りなし。    &さわらぬかみにたたりなし &--> Let sleeping dogs lie. if it ain't broke, don't fix it \\ \hline\\[-1em]
%乞うご期待 &こうごきたい &--> don't miss it; stay tuned; coming soon; look forward to it \\ \hline\\[-1em]
%失礼します &しつれいします &--> excuse me \\ \hline\\[-1em]
%をつけて、もう限界  & きをつけて、もうげんかい &--> to walk on thin ice \\ \hline\\[-1em]
%郷に入っては郷に従え & ごうにいってはごうにしたがえ &--> when in rome do as the romans do\\ \hline\\[-1em]
%遠回しに言わないでください & とおまわしにいわないでください &--> please don't beat about the bush \\ \hline\\[-1em]
% & &--> \\ \hline\\[-1em]
% & &--> \\ \hline\\[-1em]
% & &--> \\ \hline\\[-1em]
% & &--> \\ \hline\\[-1em]
% & &--> \\ \hline\\[-1em]
% & &--> \\ \hline\\[-1em]
% & &--> \\ \hline\\[-1em]
% & &--> \\ \hline\\[-1em]
% \end{tabular}

%好きでも嫌いでもないです。 & to neither like nor dislike.

\end{document}

