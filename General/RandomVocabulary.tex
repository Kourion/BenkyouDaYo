\documentclass{article}
\usepackage{xeCJK}
\usepackage{enumitem}
\usepackage{titlesec}
\usepackage[dvipsnames]{xcolor}
\usepackage{fancyhdr}
\usepackage{geometry}
\usepackage[CJK,overlap]{ruby}
 \geometry{
 a4paper,
 %total={170mm,257mm},
 left=5mm,
 top=12mm,
 right=5mm,
 bottom=12mm,
 }
 
\renewcommand{\rubysep}{0.1ex}%3.9ex}
\makeatletter
\renewcommand\paragraph{\@startsection{paragraph}{4}{\z@}%
            {-0.5ex\@plus -.1ex \@minus -.25ex}%
            {-1em}%
            {\normalfont\normalsize\bfseries}}
\renewcommand\subparagraph{\@startsection{subparagraph}{5}{\z@}%
            {-0.1ex\@plus -.1ex \@minus -.25ex}%
            {-1em}%
            {\normalfont\normalsize\bfseries}}
\makeatother

\pagestyle{fancy}
\lhead{}
\chead{}
\rhead{}
\lfoot{}
\cfoot{\thepage}
\rfoot{v0.3}

\newcommand\tab[1][1cm]{\hspace*{#1}} 
\newcommand\tabi[1][0.05cm]{\hspace*{#1}}
\newcommand\tabni[1][0.2cm]{\hspace*{#1}}
\newcommand\tabyon[1][0.4cm]{\hspace*{#1}}
\newcommand\tabroku[1][0.6cm]{\hspace*{#1}}

\XeTeXlinebreaklocale "ja"
\XeTeXlinebreakskip=0em plus 0.1em minus 0.01em

\setlist{nolistsep}
\titlespacing*{\section}
{0pt}{1.5ex plus 1ex minus .2ex}{1.5ex plus .2ex}
\titlespacing*{\subsection}
{0pt}{0.5ex plus 1ex minus .2ex}{0.5ex plus .2ex}

\setlength{\parindent}{0pt}
\setCJKmainfont{ipaexm.ttf}
\setcounter{tocdepth}{5}

\begin{document}
\part*{Random Vocabulary}
\begin{tabular}{ l | l p{14cm} }
Kanji&Kana&Translation\\ \hline\\[-1em]
予定 &よてい &--> plan /schedule \\ \hline\\[-1em]
発音 & はつおん	&--> pronunciation 	\\ \hline\\[-1em]
惜しい &おしい &--> regrettable 							\\ \hline\\[-1em]
---- &きざ &--> smug /pompous /snobby					\\ \hline\\[-1em]
---- &こぼす &--> to spill /to drop /to shed (tears)		\\ \hline\\[-1em]
見せる &みせる &--> to show /to display					\\ \hline\\[-1em]
込む &こむ &--> to be crowded							\\ \hline\\[-1em]
頑張れ &がんばれ &--> go for it /keep at it /do your best	\\ \hline\\[-1em]
頑張る &がんばる &--> to keep at it /to do one's best		\\ \hline\\[-1em]
手袋 &てぶくろ &--> gloves /mittens						\\ \hline\\[-1em]
あだ名 &あだな &--> nickname								\\ \hline\\[-1em]
真面目 &まじめ &--> serious /honest /sober /grave			\\ \hline\\[-1em]
一杯 &いっぱい &--> full									\\ \hline\\[-1em]
酔っ払い &よっぱらい &--> drunkard								\\ \hline\\[-1em]
酔っ払う &よっぱらう &--> to get drunk						\\ \hline\\[-1em]
未定 &みてい &--> not yet fixed; undecided; pending		\\ \hline\\[-1em]
%未定 &みてい &--> not yet fixed; undecided; pending   \\ \hline\\[-1em]
香辛料 &こうしんりょう &--> spices /condiment						\\ \hline\\[-1em]
親友 &しんゆう &--> close friend							\\ \hline\\[-1em]
他 &ほか &--> other (place, thing) /the rest			\\ \hline\\[-1em]
助ける &たすける &--> to help /to save /to rescue        \\ \hline\\[-1em]
法学 &ほうがく &--> law; jurisprudence  \\ \hline\\[-1em]
航空工学 &こうくうこうがく &--> aeronautical engineering   \\ \hline\\[-1em]
情報学 &じょうほうがく &--> informatics   \\ \hline\\[-1em]
煩わしい &わずらわしい &--> troublesome; annoying; complicated   \\ \hline\\[-1em]
---- &めんどくさい &--> bothersome; tiresome; bother to do   \\ \hline\\[-1em]
%じゃが芋 &じゃがいも &--> potato   \\ \hline\\[-1em] //Now as Katakana
性行為 &せいこうい &--> sex act; intercourse   \\ \hline\\[-1em]
駒 &こま &--> piece (in shogi, chess, etc.)   \\ \hline\\[-1em]
詩人 &しじん &--> poet   \\ \hline\\[-1em]
吸血鬼 &きゅうけつき &--> vampire; bloodsucker   \\ \hline\\[-1em]
血液 &けつえき &--> blood   \\ \hline\\[-1em]
学ぶ &まなぶ &--> to study (in depth); to learn; to take lessons in   \\ \hline\\[-1em]
単語 &たんご &--> word; vocabulary;   \\ \hline\\[-1em]
---- &しょうがない &--> it can't be helped; it is inevitable; c'est la vie   \\ \hline\\[-1em]
理由 &りゆう &--> reason; pretext; motive   \\ \hline\\[-1em]
混ぜる &まぜる &--> to mix; to stir; to blend   \\ \hline\\[-1em]
---- &ませる &--> to be mature for one's age; to seem grown-up; to seem mature   \\ \hline\\[-1em]
侮辱 &ぶじょく &--> insult; contempt; slight   \\ \hline\\[-1em]
---- &ばいた &--> prostitute; whore   \\ \hline\\[-1em]
---- &なるほど &--> I see; That's right!; Indeed   \\ \hline\\[-1em]
最悪 &さいあく &--> the worst   \\ \hline\\[-1em]
雑 &ざつ &--> rough; crude; sloppy; messy   \\ \hline\\[-1em]
気持ちいい &きもちいい &--> good feeling; feeling good \\ \hline\\[-1em]
気持ち &きもち &--> feeling; sensation; mood \\ \hline\\[-1em]
辞める &やめる &--> to resign; to retire; to quit \\ \hline\\[-1em]
難い &にくい &--> difficult to ...; hard to ... \\ \hline\\[-1em]
---- &しにくい &--> hard to do \\ \hline\\[-1em]
邪魔 &じゃま &--> hindrance; obstacle; nuisance \\ \hline\\[-1em]
真っ只中 &まっただなか &--> right in the midst of; right at the height of \\ \hline\\[-1em]
---- &そがれる &--> to be dampened; to be discouraged; to be weakened; to be reduced   \\ \hline\\[-1em]
\end{tabular} \\
\begin{tabular}{ l | l p{14cm}  }
喫茶店 &きっさてん &--> coffee shop; tearoom; coffee lounge; coffeehouse; cafe \\ \hline\\[-1em]
やる気 &やるき &--> willingness (e.g. to do something); eagerness; motivation; inspiration; determination; high aspirations \\ \hline\\[-1em]
触る &さわる &--> to touch; to feel \\ \hline\\[-1em]
会計 &かいけい &--> account; finance; accountant; treasurer; paymaster; reckoning; bill \\ \hline\\[-1em]
飽きる &あきる &--> to get tired of; to lose interest in; to have enough \\ \hline\\[-1em]
---- &なんだっけ &--> what is it? \\ \hline\\[-1em]
帰国 &きこく &--> return to one's country \\ \hline\\[-1em]
請う &こう &--> to beg; to ask; to request; to invite \\ \hline\\[-1em]
---- &けど &--> but; however; although \\ \hline\\[-1em]
大事 &だいじ &--> important; valuable; serious matter \\ \hline\\[-1em]
世話 &せわ &--> looking after; help; aid; assistance \\ \hline\\[-1em]
確かに &たしかに &--> surely; certainly \\ \hline\\[-1em]
単行本 &たんこうぼん &--> special book; separate volume; book of lectures \\ \hline\\[-1em]
---- &やっぱり &--> too; also; likewise; either; still; as before; even so; either way; nonetheless; in any event; all the same \\ \hline\\[-1em]
辞める &やめる &--> to resign; to retire; to quit; to leave (one's job, etc.) \\ \hline\\[-1em]
---- &やる &--> to do; to undertake; to perform; to play (a game); to study \\ \hline\\[-1em]
謎 &なぞ &--> riddle; puzzle; enigma; mystery \\ \hline\\[-1em]
無神論者 &むしんろんしゃ &--> atheist \\ \hline\\[-1em]
怠惰 &たいだ &--> laziness; idleness; sloth \\ \hline\\[-1em]
熟語 &じゅくご &--> idiom; idiomatic phrase \\ \hline\\[-1em]
---- &だけ &--> as (e.g. as much as /as soon as); only; just; merely; \\ \hline\\[-1em]
言葉 &ことば &--> language; dialect \\ \hline\\[-1em]
---- &しゃべる &--> to talk; to chat; to chatter \\ \hline\\[-1em]
 & &--> \\ \hline\\[-1em]
 & &--> \\ \hline\\[-1em]
 & &--> \\ \hline\\[-1em]
 & &--> \\ \hline\\[-1em]
 & &--> \\ \hline\\[-1em]
 & &--> \\ \hline\\[-1em]
 & &--> \\ \hline\\[-1em]
 & &--> \\ \hline\\[-1em]
 & &--> \\ \hline\\[-1em]
 & &--> \\ \hline\\[-1em]
 & &--> \\ \hline\\[-1em]
 & &--> \\ \hline\\[-1em]
 & &--> \\ \hline\\[-1em]
 & &--> \\ \hline\\[-1em]
\end{tabular}

\part*{カタカナ /ひらがな}
\begin{tabular}{ p{3cm} | l l }
サバ    &--> makrele \\ \hline\\[-1em]
やった  &--> hooray /yowzer! /whoopee! /yes! \\ \hline\\[-1em]
ジャガイモ &--> potato \\ \hline\\[-1em]
\end{tabular}

\part*{Phrases and Idioms}
\begin{tabular}{ l | l p{10cm} }
お邪魔します        &おじゃまします     &--> excuse me for disturbing (interrupting) you; \\ \hline\\[-1em]
触らぬ神に祟りなし。&さわらぬかみにたたりなし &--> Let sleeping dogs lie. if it ain't broke, don't fix it \\ \hline\\[-1em]
乞うご期待 &こうごきたい &--> don't miss it; stay tuned; coming soon; look forward to it \\ \hline\\[-1em]
 失礼します &しつれいします &--> excuse me \\ \hline\\[-1em]
 & &--> \\ \hline\\[-1em]
 & &--> \\ \hline\\[-1em]
 & &--> \\ \hline\\[-1em]
 & &--> \\ \hline\\[-1em]
 & &--> \\ \hline\\[-1em]
 & &--> \\ \hline\\[-1em]
 & &--> \\ \hline\\[-1em]
 & &--> \\ \hline\\[-1em]
 & &--> \\ \hline\\[-1em]
 & &--> \\ \hline\\[-1em]
 & &--> \\ \hline\\[-1em]
\end{tabular}

\end{document}
