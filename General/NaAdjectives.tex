\documentclass{article}
\usepackage{xeCJK}
\usepackage{enumitem}
\usepackage{titlesec}
\usepackage[dvipsnames]{xcolor}
\usepackage{geometry}
\usepackage[CJK,overlap]{ruby}
 \geometry{
 a4paper,
 %total={170mm,257mm},
 left=10mm,
 top=12mm,
 right=10mm,
 bottom=12mm,
 }
 
\renewcommand{\rubysep}{0.1ex}%3.9ex}
\makeatletter
\renewcommand\paragraph{\@startsection{paragraph}{4}{\z@}%
            {-0.5ex\@plus -.1ex \@minus -.25ex}%
            {-1em}%
            {\normalfont\normalsize\bfseries}}
\renewcommand\subparagraph{\@startsection{subparagraph}{5}{\z@}%
            {-0.1ex\@plus -.1ex \@minus -.25ex}%
            {-1em}%
            {\normalfont\normalsize\bfseries}}
\makeatother

\newcommand\tab[1][1cm]{\hspace*{#1}} 
\newcommand\tabi[1][0.05cm]{\hspace*{#1}}
\newcommand\tabni[1][0.2cm]{\hspace*{#1}}
\newcommand\tabyon[1][0.4cm]{\hspace*{#1}}
\newcommand\tabroku[1][0.6cm]{\hspace*{#1}}

\XeTeXlinebreaklocale "ja"
\XeTeXlinebreakskip=0em plus 0.1em minus 0.01em

\setlist{nolistsep}
\titlespacing*{\section}
{0pt}{1.5ex plus 1ex minus .2ex}{1.5ex plus .2ex}
\titlespacing*{\subsection}
{0pt}{0.5ex plus 1ex minus .2ex}{0.5ex plus .2ex}

\setlength{\parindent}{0pt}
\setCJKmainfont{ipaexm.ttf}
\setcounter{tocdepth}{5}

\begin{document}	%\large {1.5ex \@plus .2ex}%
%\tableofcontents
%\newpage

\section{All な-Adjectives (These lists are not at all complete)}
\begin{itemize}
\item Adjective ends with い: \\
\end{itemize}
\begin{tabular}{ p{1.41cm} | p{3.4cm} l }
得意&とくい&--> good at and likes \\ \hline\\[-1em]
大勢 &おおぜい&--> lots of people, crowded\\ \hline\\[-1em]
有名 &ゆうめい&--> famous\\ \hline\\[-1em]
きれい &きれい&--> pretty, beautiful \\ \hline\\[-1em]
嫌い&きらい&--> to not like\\ \hline\\[-1em]
丁寧 &ていねい &--> polite\\ \hline\\[-1em]
一生懸命&いっしょうけんめい&--> to one’s fullest possibility\\ \hline\\[-1em]
心配&しんぱい&--> care, worry %\\ \hline\\[-1em]
\end{tabular} \\ \\ \\
%\end{table}
%\begin{table}[h] %[ht]
\begin{itemize}
\item Adjective ends with either of: \'き,し,ち,に,ひ,み,り,ぎ,じ,ぢ,び,ぴ\' \\
\end{itemize}
\begin{tabular}{ p{1.41cm} | p{3.4cm} l }
好き&すき&--> like /love\\ \hline\\[-1em]
便利&べんり&--> useful, convenient\\ \hline\\[-1em]
元気&げんき&--> healthy\\ \hline\\[-1em]
無理 	&むり 	&--> unreasonable %\\ \hline\\[-1em]
\end{tabular} \\ \\ \\
%\end{table}
%\begin{table}[ht]
\begin{itemize}
\item Adjective ends with something else: \\
\end{itemize}
\begin{tabular}{ p{1.41cm} | p{3.4cm} l }
変&へん&--> strange \\ \hline\\[-1em]
上手&じょうず&--> good at \\ \hline\\[-1em]
下手&へた&--> bad at \\ \hline\\[-1em]
苦手&にがて&--> bad at and dislikes \\ \hline\\[-1em]
大切 &たいせつ&--> important, great\\ \hline\\[-1em]
結構 &けっこう&--> wonderful,enviable\\ \hline\\[-1em]
簡単 &かんたん&--> easy\\ \hline\\[-1em]
静か&しずか&--> quiet\\ \hline\\[-1em]
暇&ひま&--> free (time)\\ \hline\\[-1em]
賑やか&にぎやか&--> lively\\ \hline\\[-1em]
色々&いろいろ&--> various\\ \hline\\[-1em]
大丈夫&だいじょうぶ&--> OK,fine\\ \hline\\[-1em]
丈夫&じょうぶ&--> healthy, robust\\ \hline\\[-1em]
大変&たいへん&--> very /terrible\\ \hline\\[-1em]
楽 &らく&--> comfortable, easy\\ \hline\\[-1em]
嫌 &いや&--> unpleasant\\ \hline\\[-1em]
大切 &たいせつ&--> important,precious\\ \hline\\[-1em]
上手&じょうず&--> good, skilled\\ \hline\\[-1em]
下手&へた&--> bad at, unskilled\\ \hline\\[-1em]
危険&きけん&--> dangerous\\ \hline\\[-1em]
残念&ざんねん&--> regrettable, disappointing\\ \hline\\[-1em]
自由&じゆう&--> free, unrestrained\\ \hline\\[-1em]
十分&じゅうぶん&--> sufficient\\ \hline\\[-1em]
適当 	&てきとう&--> proper, suitable\\ \hline\\[-1em]
特別 	&とくべつ &--> special\\ \hline\\[-1em]
熱心 	&ねっしん &--> eager, enthusiastic\\ \hline\\[-1em]
必要 	&ひつよう &--> necessary\\ \hline\\[-1em]
真面目 	&まじめ &--> serious, earnest\\ \hline\\[-1em]
真直ぐ 	&まっすぐ&--> straight\\ \hline\\[-1em]
立派 	&りっぱ &--> excellent, splendid,fine\\ \hline\\[-1em]
不便 	&ふべん &--> inconvenient   %\\ \hline\\[-1em]
\end{tabular}
%\end{table}
\end{document}
