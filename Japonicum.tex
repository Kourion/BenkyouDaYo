\documentclass{article}
\usepackage{xeCJK}
\usepackage{enumitem}
\usepackage{titlesec}
\usepackage{geometry}
\usepackage[CJK,overlap]{ruby}
 \geometry{
 a4paper,
 %total={170mm,257mm},
 left=10mm,
 top=12mm,
 right=10mm,
 bottom=12mm,
 }
 
\renewcommand{\rubysep}{0.1ex}%3.9ex}
\makeatletter
\renewcommand\paragraph{\@startsection{paragraph}{4}{\z@}%
            {-0.5ex\@plus -.1ex \@minus -.25ex}%
            {-1em}%
            {\normalfont\normalsize\bfseries}}
\renewcommand\subparagraph{\@startsection{subparagraph}{5}{\z@}%
            {-0.1ex\@plus -.1ex \@minus -.25ex}%
            {-1em}%
            {\normalfont\normalsize\bfseries}}
\makeatother
\newcommand\tab[1][1cm]{\hspace*{#1}} 
\newcommand\tabni[1][0.2cm]{\hspace*{#1}}
\newcommand\tabyon[1][0.4cm]{\hspace*{#1}}
\newcommand\tabroku[1][0.6cm]{\hspace*{#1}}

\XeTeXlinebreaklocale "ja"
\XeTeXlinebreakskip=0em plus 0.1em minus 0.01em

\setlist{nolistsep}
\titlespacing*{\section}
{0pt}{1.5ex plus 1ex minus .2ex}{1.5ex plus .2ex}
\titlespacing*{\subsection}
{0pt}{0.5ex plus 1ex minus .2ex}{0.5ex plus .2ex}

\setlength{\parindent}{0pt}
\setCJKmainfont{ipaexm.ttf}
\setcounter{tocdepth}{5}

\begin{document}	%\large {1.5ex \@plus .2ex}%

\tableofcontents
\newpage

\section{Verb Conjugation}
\subsection{The 3 Verb Groups}
\begin{enumerate}
\item Group 3: \ruby{来}{く}る and する make up the group 3 verbs.
\item Group 2: If there is a え or い ending syllable before the
 る-ending of the verb then it is a group 2 verb.
\item Exceptions: There are some exceptions to group 2: \ruby{帰}{かえ}る,\ruby{切}{き}る,
\ruby{知}{し}る,\ruby{入}{はい}る,\ruby{走}{はし}る these are group 1 verbs.
%More such excpetions for future reference: しる (to know), はいる (to enter), いる (to need), すべる (to slide), へる (to decrease), ねる (to knead), ける (to kick), かじる (to bite), まじる (to be mixed), しゃべる (to talk), せびる (to extort), あせる (to be in a hurry), はしる (to run), ちる (to scatter), てる (to shine)
\item Group 1: Everything else. Including る ending verbs which are not in group 2.
\end{enumerate}
\subsection{The Polite-Style, ます-Form}
\begin{enumerate}
\item Group 1: Change the う-ending to a い-ending and add ます.
\item Group 2: Remove the る and add ます instead.
\item Group 3: する --> します, \tabyon \ruby{来}{く}る --> \ruby{来}{き}ます
\end{enumerate}
\subsubsection{The Forms ません, ました, ませんでした, ましょう}
\begin{enumerate} \item Remove the ます and add ません for the negative-form, ました for the past-form,\\
ませんでした for the past-negative-form and finally ましょう for the 'lets-do'-form.
\end{enumerate}
\subsection{The Plain-Style, ない-Form }
\begin{enumerate}
\item Group 1: Change the う-ending to an あ-ending, if the verb ends in う itself, 
change it to わ instead.
Add ない. \\ \tab ある is an exception, it becomes simply ない itself.
\item Group 2: Remove the る and add ない. 
\item Group 3: する --> しない, \tabyon \ruby{来}{く}る --> \ruby{来}{こ}ない
\end{enumerate}
\subsubsection{The negative Plain-Style, なかった-Form}
\begin{enumerate}
\item In order to negate any verb in plain style simply remove the ない and add なかった instead.
\end{enumerate}
\subsection{The て-Form}
\begin{enumerate} \item Group 1:
\begin{enumerate} \item Ending with: る, つ, う: Remove the ending and replace it with って.
\item Ending with: む, ぶ, ぬ Remove the ending and replace it with んで.
\item Ending with: く, ぐ: The く becomes いて and the ぐ becomes いで. Exception: いく becomes いって.
\item Ending with: す: Remove the ending and replace it with して.
\end{enumerate}
\item Group 2: Replace the ending る with て.
\item Group 3: する --> して, \tabyon \ruby{来}{く}る --> きて
\end{enumerate}
\subsubsection{The た-Form}
\begin{enumerate}
\item For the た-Form simply replace the て with た and で with だ.
\end{enumerate}
\subsection{The Casual Volitional-Form ¥}
\begin{enumerate}
\item This form expresses ones desire to do something akin to 'Let's...',
 it is the plain-style of ましょう.
\item Group 1: Change the ending from its う-form to the お-form and add う.
This then forms essentially a long o.
\item Group 2: Take out the ending る and replace it with よう.
\item Group 3: する --> しよう, \tabyon \ruby{来}{く}る --> \ruby{来}{こ}よう.
\end{enumerate}
\subsection{The Potential-Form ¥}
\begin{enumerate}
\item This form expresses ones ability to know how to, to be able to, do something.
\item Group 1: Change the ending from its う-form to the え-form and add る.
\item Group 2: Take out the ending る and replace it with られる.
\item Group 3: する --> できる, \tabyon \ruby{来}{く}る --> \ruby{来}{こ}られる.
\item Warning: All Verbs in potential-form are group 2 verbs when conjugating further.
\end{enumerate}
\subsection{The Causative-Form ¥}	
\begin{enumerate}
\item This form expresses one to let or make someone else do something.
\item Group 1: Change the ending from its う-form to the あ-form and add せる.
\item Group 2: Take out the ending る and replace it with させる.
\item Group 3: する --> させる, \tabyon \ruby{来}{く}る --> \ruby{来}{こ}させる.
\item Warning: All Verbs in potential-form are group 2 verbs when conjugating further.
\end{enumerate}
\section{Vocabulary}
\begin{itemize}
\item おぼえています --> to remember
\item おぼえる --> to memorize
\item わすれました --> to have forgotten
\item わすれる --> to forget
\item よとう --> ruling party
\item やとう --> opposition party
\end{itemize}
\section{To want something...}
\subsection{Desiring things: ほしい}
\begin{itemize}
\item ほしい is used when the thing you desire is an object and not an action.
\item First Person:
\begin{itemize}
\item Because ほしい (hoshii) is an adjective and not a verb, the object of the sentence is marked by が (ga) and not を (o).
\item Example: 私はお金がほしいです。 --> I want money.
\end{itemize}
\item Third Person:
\begin{itemize}
\item: When expressing other’s desires using ほしがっています you should use the particle を instead of が (ga).
\item Example: 彼女はワインをほしがっています。	 --> She seems to want wine.	(Wordstem: ほしがる)
\item An equivalent would be: 彼女はワインがほしいと言っています。 --> She said she wants wine.
\end{itemize}
\end{itemize}
\subsection{Desiring actions: たい}
\begin{itemize}
\item たい works again like an adjective thus が is used and the negative can be formed with (た)くない。
\item Positive first person: 私はパンが食べたいです。 --> I want to eat bread.
\item Negative first person: 私はパンが食べたくないです。 --> I do not want to eat bread. \\
\item Again third person is formed with がって thus we are back to using を.
\item Third Person: かれはコーヒーを飲みたがっています。 --> He wants to drink coffee.
\item Equivalent: コーヒーがのみたいといっています。
\end{itemize}
\subsection{Comparing たい und ほしい}
\begin{itemize}
\item あなたはなにがほしいですか。 --> What do you want?
\item 彼女はなにをほしいがっていますか。 --> What does she want?
\item あなた\ruby{何}{なに}がしたいですか。 --> What do you want to do?
\item 彼は\ruby{何}{なに}をしたがっています。 --> What does she want to do?
\end{itemize}
\section{Expressions of Desire or Hope}
\subsubsection{こわい --> scary/awful }
\begin{itemize}
\item 彼女はこわいです。 --> She is awful.
\item 彼女は\ruby{犬}{いぬ}をこわがっています。 --> It seems she is afraid of dogs.
\item 彼女は\ruby{犬}{いぬ}をこわいと言ってます。 --> She said she is afraid of dogs.
\item All statements are to be taken as facts, the uncertainty the middle one contains comes from politeness.
\end{itemize}
\subsubsection{\ruby{嬉}{うれ}しい / \ruby{喜}{よろこ}んで --> happy}
うれしい and よろこんで both mean happy. But うれしい is used in first person and よろこんで in third person.
よろこんで comes from よろこぶ.
\begin{itemize}
\item 私はうれしいです。 --> I am happy.
\item 彼はよろこんでいます。 --> He is happy.
\end{itemize}
\subsubsection{\ruby{悲}{かな}しい --> sad }
\begin{itemize}
\item The first person uses \ruby{悲}{かな}しい as い-adjective the third person however derives \ruby{悲}{かな}しんで from \ruby{悲}{かな}しむ.
\item 私はかなしいです。 --> I am sad.
\item 彼女は\ruby{悲}{かな}しんでいます。 --> She is sad.
\end{itemize}
\subsubsection{\ruby{寂}{さび}しい --> lonely }
\begin{itemize}
\item The first person uses \ruby{寂}{さび}しい as い-adjective the third person however again uses it in connection with がって forming さびしがっています.
\item 私は\ruby{寂}{さび}しいです。 --> I am loney.
\item 彼はさびしがっています。 --> He is lonely.
\end{itemize}
\subsubsection{\ruby{寂}{さび}しい --> ashamed }
\begin{itemize}
\item The first person uses \ruby{寂}{さび}しい as い-adjective the third person however again uses it in connection with がって forming さびしがっています.
\item 私は\ruby{寂}{さび}しいです。 --> I am loney.
\item 彼はさびしがっています。 --> He is lonely.
\end{itemize}
\end{document}