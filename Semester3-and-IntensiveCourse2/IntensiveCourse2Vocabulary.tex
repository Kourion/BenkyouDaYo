\documentclass{article}
\usepackage{xeCJK}
\usepackage{enumitem}
\usepackage{titlesec}
\usepackage[dvipsnames]{xcolor}
\usepackage{fancyhdr}
\usepackage{geometry}
\usepackage[CJK,overlap]{ruby}
 \geometry{
 a4paper,
 %total={170mm,257mm},
 left=10mm,
 top=10mm,
 right=10mm,
 bottom=12mm,
 }
 
\renewcommand{\rubysep}{0.1ex}%3.9ex}
\makeatletter
\renewcommand\paragraph{\@startsection{paragraph}{4}{\z@}%
            {-0.5ex\@plus -.1ex \@minus -.25ex}%
            {-1em}%
            {\normalfont\normalsize\bfseries}}
\renewcommand\subparagraph{\@startsection{subparagraph}{5}{\z@}%
            {-0.1ex\@plus -.1ex \@minus -.25ex}%
            {-1em}%
            {\normalfont\normalsize\bfseries}}
\makeatother

\newcommand\tab[1][1cm]{\hspace*{#1}} 
\newcommand\tabi[1][0.05cm]{\hspace*{#1}}
\newcommand\tabni[1][0.2cm]{\hspace*{#1}}
\newcommand\tabyon[1][0.4cm]{\hspace*{#1}}
\newcommand\tabroku[1][0.6cm]{\hspace*{#1}}

\pagestyle{fancy}
\lhead{}
\chead{}
\rhead{}
\lfoot{}
\cfoot{\thepage}
\rfoot{v0.3}
\renewcommand{\headrulewidth}{0pt}
\renewcommand{\footrulewidth}{0pt}

\XeTeXlinebreaklocale "ja"
\XeTeXlinebreakskip=0em plus 0.1em minus 0.01em

\setlist{nolistsep}
\titlespacing*{\section}
{0pt}{1.5ex plus 1ex minus .2ex}{1.5ex plus .2ex}
\titlespacing*{\subsection}
{0pt}{0.5ex plus 1ex minus .2ex}{0.5ex plus .2ex}

\setlength{\parindent}{0pt}
\setCJKmainfont{ipaexm.ttf}
\setcounter{tocdepth}{5}

\begin{document}
\part*{New Vocabulary}
\begin{tabular}{ l | l p{13cm} }
Kanji&Kana&Translation\\ \hline\\[-1em]
*返す&かえす&--> to return (something)\\ \hline\\[-1em]
*積もり&つもり&--> intention; plan \\ \hline\\[-1em]
*恐らく&おそらく&--> perhaps; likely; probably; I dare say \\ \hline\\[-1em]
*----&だろう&--> seems; I think; I guess; I wonder; I hope /don't you agree?; I thought you'd say that! \\ \hline\\[-1em]
*----&でしょう&--> seems; I think; I guess; I wonder; I hope /don't you agree?; I thought you'd say that! \\ \hline\\[-1em]
*----&ひょっとしたら&--> possibly; by some possibility; perhaps \\ \hline\\[-1em]
*----&ひょっとすると&--> perhaps; maybe; possibly \\ \hline\\[-1em]
*若しかしたら&もしかしたら&--> perhaps; maybe; perchance; by some chance; by any chance \\ \hline\\[-1em]
*若しかすると&もしかすると&--> perhaps; maybe; by some chance \\ \hline\\[-1em]
*かも知れない&かもしれない&--> may; might; perhaps; may be; possibly \\ \hline\\[-1em]
*----&こんな&--> such /like this \\ \hline\\[-1em]
*----&そんな&--> such /like that; that sort of \\ \hline\\[-1em]
*----&あんな&--> such /so; that; sort of \\ \hline\\[-1em]
*----&どんな&--> what; what kind of \\ \hline\\[-1em]
*違い&ちがい&--> difference; discrepancy \\ \hline\\[-1em]
*違いない&ちがいない&--> (phrase) sure; no mistaking it; for certain \\ \hline\\[-1em]
*屹度&きっと&--> surely; undoubtedly; almost certainly; most likely (e.g. 90 percent) \\ \hline\\[-1em]
*音&おと&--> sound; noise; report \\ \hline\\[-1em]
*声&こえ&--> voice \\ \hline\\[-1em]
*聞こえる&きこえる&--> to be heard; to be audible \\ \hline\\[-1em]
*見える&みえる&--> to be seen; to be in sight \\ \hline\\[-1em]
*開く&あく&--> to open (e.g. doors, business, etc.) \\ \hline\\[-1em]
*開ける&あける&--> to open (a door, for business, etc.); to unwrap (e.g. parcel, package); to unlock \\ \hline\\[-1em]
*閉める&しめる&--> to close; to shut \\ \hline\\[-1em]
*閉まる&しまる&--> to be shut; to close; to be closed \\ \hline\\[-1em]
*開く&ひらく&--> to open; to undo; to unseal; to unpack \\ \hline\\[-1em]
*閉じる&とじる&--> to close (e.g. book, eyes, meeting, etc.); to shut \\ \hline\\[-1em]
*掛ける&かける&--> to hang (e.g. picture, cutain); to hoist (e.g. sail, flag) /to make (a call) /to put on (glasses, etc.) /to secure (e.g. lock) \\ \hline\\[-1em]
*掛かる&かかる&--> to hang(e.g. picture, curtain) /to take (a resource, e.g. time or money) \\ \hline\\[-1em]
*鍵&かぎ&--> key, lock \\ \hline\\[-1em]
*電気&でんき&--> electricity /(electric) light \\ \hline\\[-1em]
*点ける&つける&--> to turn on; to switch on; to light up \\ \hline\\[-1em]
*点く&つく&--> to be lit (e.g. electricity comes on); to be lighted \\ \hline\\[-1em]
*消す&けす&--> to turn off power /to extinguish; to put out /to erase; to delete; to cross out \\ \hline\\[-1em]
*消える&きえる&--> to go out; to vanish; to disappear \\ \hline\\[-1em]
*卒業&そつぎょう&--> graduation; completion (e.g. of a course) \\ \hline\\[-1em]
*置く&おく&--> to put; to place /to do something in advance \\ \hline\\[-1em]
*予習&よしゅう&--> preparation for a lesson \\ \hline\\[-1em]
*復習&ふくしゅう&--> review; revision \\ \hline\\[-1em]
*準備&じゅんび&--> preparation; setup; arrangements; provision; reserve \\ \hline\\[-1em]
*予約&よやく&--> reservation; appointment; booking; advance order \\ \hline\\[-1em]
*上がる&あがる&--> to rise; to go up; to come up; to ascend; to be raised \\ \hline\\[-1em]
*下がる&さがる&--> to come down; to go down; to fall; to drop; to sink; to get lower \\ \hline\\[-1em]
*下りる&おりる&--> to descend (e.g. a mountain); to go down; to come down \\ \hline\\[-1em]
*上る&のぼる&--> to ascend; to go up; to climb \\ \hline\\[-1em]
\end{tabular}
\begin{tabular}{ l | l p{13cm} }
*取る&とる&--> to take; to pick up; to harvest; to earn; to win; to choose \\ \hline\\[-1em]
*迎える&むかえる&--> to go out to meet /to receive; to welcome; to greet; \\ \hline\\[-1em]
*仕舞う&しまう&--> to finish; to stop; to end; to put an end to; to bring to a close \\ \hline\\[-1em]
*迄に&までに&--> by; not later than; before \\ \hline\\[-1em]
*困る&こまる&--> to be troubled; to be worried; to be bothered; to be embarrassed; to be stumped \\ \hline\\[-1em]
*半年&はんとし&--> half a year /6 months \\ \hline\\[-1em]
*成る&なる&--> to become; to get; to grow; to be; to reach; to attain \\ \hline\\[-1em]
*砂糖&さとう&--> sugar \\ \hline\\[-1em]
*塩&しお&--> salt \\ \hline\\[-1em]
*入れる&いれる&--> to put in; to let in; to take in; to bring in; to insert; \\ \hline\\[-1em]
*塩辛い&しおからい&--> salty (taste) \\ \hline\\[-1em]
*酸っぱい&すっぱい&--> sour \\ \hline\\[-1em]
*酢&す&--> vinegar \\ \hline\\[-1em]
*掃除&そうじ&--> cleaning; sweeping; dusting; scrubbing \\ \hline\\[-1em]
*洗濯&せんたく&--> washing; laundry \\ \hline\\[-1em]
*機械&きかい&--> machine; mechanism \\ \hline\\[-1em]
*洗濯機&せんたくき&--> washing machine (laundry) \\ \hline\\[-1em]
*----&はっきり(と)&--> clearly; plainly; distinctly \\ \hline\\[-1em]
*----&どんどん&--> rapidly \\ \hline\\[-1em]
*段々(と)&だんだん(と)&--> gradually \\ \hline\\[-1em]
*確り(と)&しっかり(と)&--> properly; well; strongly; tightly \\ \hline\\[-1em]
*----&ちゃんと&--> diligently; seriously; earnestly; perfectly \\ \hline\\[-1em]
*----&もう少し&--> a bit more \\ \hline\\[-1em]
*真っ直ぐ&まっすぐ&--> straight (ahead); direct; \\ \hline\\[-1em]
*----&ずっと&--> continuously /all the way \\ \hline\\[-1em]
*曲がる&まがる&--> to turn \\ \hline\\[-1em]
*
\end{tabular}

%\newpage




\part*{カタカナ}
\begin{tabular}{ p{3cm} | l l }
*カーテン&--> curtain; curtains \\ \hline\\[-1em]
*スイッチ&--> switch \\ \hline\\[-1em]
*オン&--> on \\ \hline\\[-1em]
*オフ&--> off \\ \hline\\[-1em]
*ブラインド&--> blinds /window blind \\ \hline\\[-1em]
*コショウ&(胡椒/こしょう) --> pepper
*マイルド&--> mild
\end{tabular}

%\part*{Vocabulary that we already learned}
%\begin{tabular}{ p{3cm} | l l }
%*違い&ちがい&--> difference; discrepancy \\ \hline\\[-1em]
%\end{tabular}
%\footnotetext{する-verb}
\end{document}
