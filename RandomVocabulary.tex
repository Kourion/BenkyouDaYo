\documentclass{article}
\usepackage{xeCJK}
\usepackage{enumitem}
\usepackage{titlesec}
\usepackage[dvipsnames]{xcolor}
\usepackage{geometry}
\usepackage[CJK,overlap]{ruby}
 \geometry{
 a4paper,
 %total={170mm,257mm},
 left=10mm,
 top=12mm,
 right=10mm,
 bottom=12mm,
 }
 
\renewcommand{\rubysep}{0.1ex}%3.9ex}
\makeatletter
\renewcommand\paragraph{\@startsection{paragraph}{4}{\z@}%
            {-0.5ex\@plus -.1ex \@minus -.25ex}%
            {-1em}%
            {\normalfont\normalsize\bfseries}}
\renewcommand\subparagraph{\@startsection{subparagraph}{5}{\z@}%
            {-0.1ex\@plus -.1ex \@minus -.25ex}%
            {-1em}%
            {\normalfont\normalsize\bfseries}}
\makeatother

\newcommand\tab[1][1cm]{\hspace*{#1}} 
\newcommand\tabi[1][0.05cm]{\hspace*{#1}}
\newcommand\tabni[1][0.2cm]{\hspace*{#1}}
\newcommand\tabyon[1][0.4cm]{\hspace*{#1}}
\newcommand\tabroku[1][0.6cm]{\hspace*{#1}}

\XeTeXlinebreaklocale "ja"
\XeTeXlinebreakskip=0em plus 0.1em minus 0.01em

\setlist{nolistsep}
\titlespacing*{\section}
{0pt}{1.5ex plus 1ex minus .2ex}{1.5ex plus .2ex}
\titlespacing*{\subsection}
{0pt}{0.5ex plus 1ex minus .2ex}{0.5ex plus .2ex}

\setlength{\parindent}{0pt}
\setCJKmainfont{ipaexm.ttf}
\setcounter{tocdepth}{5}

\begin{document}
\part*{Random Vocabulary}
\begin{tabular}{ l | l l }
Kanji&Kana&Translation\\ \hline\\[-1em]
予定				& よてい					&--> plan /schedule 						\\ \hline\\[-1em]
発音				& はつおん					&--> pronunciation 							\\ \hline\\[-1em]
惜しい			& おしい					&--> regrettable 							\\ \hline\\[-1em]
気障				& きざ 					&--> smug /pompous /snobby					\\ \hline\\[-1em]
零す				& こぼす					&--> to spill /to drop /to shed (tears)		\\ \hline\\[-1em]
見せる			& みせる					&--> to show /to display					\\ \hline\\[-1em]
込む				& こむ					&--> to be crowded							\\ \hline\\[-1em]
*				& やった					&--> hooray									\\ \hline\\[-1em]
頑張れ			& がんばれ					&--> go for it /keep at it /do your best	\\ \hline\\[-1em]
頑張る			& がんばる 				&--> to keep at it /to do one's best		\\ \hline\\[-1em]
手袋				& てぶくろ					&--> gloves /mittens						\\ \hline\\[-1em]
あだ名			& あだな					&--> nickname								\\ \hline\\[-1em]
真面目			& まじめ					&--> serious /honest /sober /grave			\\ \hline\\[-1em]
一杯				& いっぱい					&--> full									\\ \hline\\[-1em]
酔っ払い			& よっぱらい				&--> drunkard								\\ \hline\\[-1em]
酔っ払う			& よっぱらう					&--> to get drunk							\\ \hline\\[-1em]
未定				& みてい					&--> not yet fixed; undecided; pending		\\ \hline\\[-1em]
香辛料			& こうしんりょう 				&--> spices /condiment						\\ \hline\\[-1em]
親友				& しんゆう					&--> close friend							\\ \hline\\[-1em]
他				& ほか					&--> other (place, thing) /the rest			\\ \hline\\[-1em]
助ける			& たすける					&--> to help /to save /to rescue

\end{tabular}
\end{document}