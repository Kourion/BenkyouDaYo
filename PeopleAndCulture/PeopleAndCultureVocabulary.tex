\documentclass{article}
\usepackage{xeCJK}
\usepackage{enumitem}
\usepackage{titlesec}
\usepackage[dvipsnames]{xcolor}
\usepackage{fancyhdr}
\usepackage{geometry}
\usepackage[CJK,overlap]{ruby}
\geometry{ a4paper, %total={170mm,257mm},
	left=10mm, top=10mm, right=10mm, bottom=12mm, } \pagestyle{fancy}
\lhead{}
\chead{}
\rhead{}
\lfoot{}
\cfoot{\thepage}
\rfoot{v1.0}


\renewcommand{\rubysep}{0.1ex}%3.9ex}
\makeatletter\renewcommand\paragraph{\@startsection{paragraph}{4}{\z@}{-0.5ex\@plus -.1ex \@minus -.25ex}{-1em}{\normalfont\normalsize\bfseries}}
\renewcommand\subparagraph{\@startsection{subparagraph}{5}{\z@}{-0.1ex\@plus -.1ex \@minus -.25ex}{-1em}{\normalfont\normalsize\bfseries}}\makeatother
\renewcommand{\headrulewidth}{0pt} \renewcommand{\footrulewidth}{0pt}
\newcommand\tab[1][1cm]{\hspace*{#1}} \newcommand\tabi[1][0.05cm]{\hspace*{#1}}\newcommand\tabni[1][0.2cm]{\hspace*{#1}}
\newcommand\tabyon[1][0.4cm]{\hspace*{#1}}\newcommand\tabroku[1][0.6cm]{\hspace*{#1}}
\XeTeXlinebreaklocale "ja" \XeTeXlinebreakskip=0em plus 0.1em minus 0.01em \setlist{nolistsep}
\titlespacing*{\section}{0pt}{1.5ex plus 1ex minus .2ex}{1.5ex plus .2ex}
\titlespacing*{\subsection}{0pt}{0.5ex plus 1ex minus .2ex}{0.5ex plus .2ex}
\setlength{\parindent}{0pt}\setCJKmainfont{ipaexm.ttf}\setcounter{tocdepth}{5}

\begin{document} \large

\section{地理 (ちり) geography	| 自然 (しぜん) nature}
\begin{tabular}{ l | l | p{7.5cm} }
\ruby{地理}{ちり} &  ちり  & geography \\ \hline \\[-1em]
\ruby{自然}{しぜん} &  しぜん & nature \\ \hline \\[-1em]
\ruby{気候}{ きこう } &  きこう  & climate \\ \hline \\[-1em]
\ruby{琵琶湖}{ びわこ } & びわこ/びわみずうみ & lake biwa(ko)   \\ \hline \\[-1em]
\ruby{北海道}{ほっかいどう} & ほっかいどう & Hokkaido  \\ \hline \\[-1em]
\ruby{本州}{ ほんしゅう } &  ほんしゅう  & Honshuu  \\ \hline \\[-1em]
\ruby{四国}{ しこく } &  しこく & Shikoku  \\ \hline \\[-1em]
\ruby{九州}{ きゅうしゅう } &  きゅうしゅう  & Kyushu  \\ \hline \\[-1em]
\ruby{沖縄}{ おきなわ } &  おきなわ  & Okinawa  \\ \hline \\[-1em]
\ruby{都道府県}{ とどうふけん } &  とどうふけん  & administrative divisions of Japan \\ \hline \\[-1em]
\ruby{札幌}{ さっぽろ } &  さっぽろ & Sapporo  \\ \hline \\[-1em]
\ruby{仙台}{ せんだい } &  せんだい  & Sendai  \\ \hline \\[-1em]
\ruby{東京}{ とうきょう } &  とうきょう  & Tokyo  \\ \hline \\[-1em]
\ruby{名古屋}{ なごや } &  なごや & Nagoya  \\ \hline \\[-1em]
\ruby{京都}{ きょうと } &  きょうと  & Kyoto  \\ \hline \\[-1em]
\ruby{大阪}{ おおさか } &  おおさか  & Osaka  \\ \hline \\[-1em]
\ruby{福岡}{ ふくおか } &  ふくおか  & Fukuoka  \\ \hline \\[-1em]
\ruby{広島}{ ひろしま } &  ひろしま  & Hiroshima  \\ \hline \\[-1em]
\ruby{長崎}{ ながさき } &  ながさき  & Nagasaki  \\ \hline \\[-1em]
\ruby{福島}{ふくしま} & ふくしま &Fukushima  \\ \hline \\[-1em]
\ruby{人口}{ じんこう } &  じんこう  & population  \\ \hline \\[-1em]
\ruby{島}{しま} & しま & island  \\ \hline \\[-1em]
\ruby{富士山}{ ふじさん} & ふじさん & Fuji-San  \\ \hline \\[-1em]
\ruby{火山}{ かざん } &  かざん & volcano  \\ \hline \\[-1em]
\ruby{温泉}{ おんせん } &  おんせん  & onsen  \\ \hline \\[-1em]
\ruby{地震}{ じしん } &  じしん  & earthquake  \\ \hline \\[-1em]
\ruby{津波}{ つなみ } &  つなみ  & tsunami  \\ \hline \\[-1em]
\ruby{気温}{ きおん } &  きおん & atmospheric temperature  \\ \hline \\[-1em]
\ruby{梅雨}{ばいう} & ばいう/つゆ & rainy season  \\ \hline \\[-1em]
\ruby{台風}{ たいふう } &  たいふう  & typhoon  \\ \hline \\[-1em]
\ruby{動物}{ どうぶつ } &  どうぶつ  & animal  \\ \hline \\[-1em]
\ruby{鳥}{とり} & とり & bird   \\ \hline \\[-1em]
\ruby{植物}{ しょくぶつ } &  しょくぶつ  & plant /vegetation   %\\ \hline \\[-1em]
\end{tabular}
\section{政治 (せいじ) politics}
\begin{tabular}{ l | l | p{7.5cm} }
憲法 & けんぽう & constitution \\ \hline \\[-1em]
天皇 & てんのう & Emperor of Japan \\ \hline \\[-1em]
政治 & せいじ & politics; government \\ \hline \\[-1em]
三権分立 & さんけんぶんりつ & separation of powers \\ \hline \\[-1em]
立法 & りっぽう & legislation \\ \hline \\[-1em]
国会 & こっかい & National Diet; parliament; congress \\ \hline \\[-1em]
衆議院 & しゅうぎいん & lower house; House of Representatives \\ \hline \\[-1em]
参議院 & さんぎいん & House of Councillors \\ \hline \\[-1em]
選挙	& せんきょ	& election \\ \hline \\[-1em]
選挙権 & せんきょけん & suffrage; the right to vote \\ \hline \\[-1em]
政党 & せいとう & political party \\ \hline \\[-1em]
自由民主党 & じゆうみんしゅとう & Liberal Democratic Party; LDP\\ \hline \\[-1em]
公明党 & こうめいとう & Komeito /Justice Party \\ \hline \\[-1em]
民進党 & みんしんとう	& Democratic Party \\ \hline \\[-1em]
社会民主党 & しゃかいみんしゅとう & Social Democratic Party \\ \hline \\[-1em]
日本共産党 & にほんきょうさんとう & Japanese Communist Party \\ \hline \\[-1em]
ドイツキリスト\ruby{教民主同盟}{きょうみんしゅどうめい} &  & (CDU) \\ \hline \\[-1em]
ドイツ\ruby{社会民主党}{しゃかいみんしゅとう} &  & (SPD) \\ \hline \\[-1em]
バイエルン &  & Bayern \\ \hline \\[-1em]
キリスト\ruby{教社会同盟}{きょうしゃかいどうめい} &  & (CSU) \\ \hline \\[-1em]
\ruby{同盟90/緑の党}{どうめいきゅうぜろ-みどりのとう} &  & (Bündnis 90 die Grünen) \\ \hline \\[-1em]
\ruby{左翼党}{さよくとう} & & Die Linke \\ \hline \\[-1em]
\ruby{自由民主党}{じゆうみんしゅとう} &  & Liberal Democratic Party; LDP\\ \hline \\[-1em]
行政 & ぎょうせい & administration / executive \\ \hline \\[-1em]
内閣	& ないかく & cabinet(government) \\ \hline \\[-1em]
内閣総理大臣 & ないかくそうりだいじん & primeminister \\ \hline \\[-1em]
(首相) & しゅしょう & chancellor \\ \hline \\[-1em]
大統領 & だいとうりょう & president \\ \hline \\[-1em]
司法 & しほう & administration of justice; Judiciary \\ \hline \\[-1em]
最高裁判所 & さいこうさいばんしょ & Supreme Court \\ \hline \\[-1em]
裁判所 & さいばんしょ & court; courthouse \\ \hline \\[-1em]
天皇明仁 & てんのうあきひと & Akihito Emperor of Japan \\ \hline \\[-1em]
今上天皇 & きんじょうてんのう & the present Emperor \\ \hline \\[-1em]
皇后 & こうごう & (Japanese) empress; queen \\ \hline \\[-1em]
陛下 & へいか & your Majesty; his Majesty; her Majesty \\ \hline \\[-1em]
皇太子徳仁親王 & こうたいしなるひと & Crown Princess Masako \\ \hline \\[-1em]
秋篠宮文仁 & あきしののみやふみひと & Akishino-no-miya Fumihito \\ \hline \\[-1em]
親王 & しんのう & prince of royal blood; Imperial prince \\ \hline \\[-1em]
悠仁親王 & はるかひとししんのう & Prince Hisahito of Akishino \\ \hline \\[-1em]
殿下 & でんか & your/his/her Highness \\ \hline \\[-1em]
宮内庁 & くないちょう & Imperial Household Agency %\\ \hline \\[-1em]
\end{tabular}
\section{外交 (がいこう) diplomacy}
%\begin{center}
\begin{tabular}{  l | l | p{7.5cm} }
%Kanji 					& Hiragana 			& Translation 					\\[5pt] \hline	\\[-1em]
 外交 					& がいこう 				& diplomacy   							\\ \hline \\[-1em]
 外務省					& がいむしょう			& Ministry of Foreign Affairs 			\\ \hline \\[-1em]
 国交問題					& こっこうもんだい			& diplomatic relations problem 			\\ \hline \\[-1em]
 サンフランシスコ講和条約 		& SFRこうわじょうやく 		& San Francisco Treaty (Sept. 8, 1951)	\\ \hline \\[-1em]
 日華平和条約				& にっかへいわじょうやく  	& peace treaty between CHI and JAP 1978	\\ \hline \\[-1em]
 日ソ共同宣言				& にっきょうどうせんげん 	& JAP-SOV joint declaration 			\\ \hline \\[-1em]
 日韓基本条約				& にっかんきほんじょうやく	& JAP-SKOR basic treaty					\\ \hline \\[-1em]
 日中共同声明				& にちゅきょうどうせいめい	& CHI-JAP joint declaration 			\\ \hline \\[-1em]
 平和条約					& へいわじょうやく 		& peace treaty 							\\ \hline \\[-1em]
 領土問題					& りょうどもんだい			& territorial dispute 					\\ \hline \\[-1em]
 ロシア					& ろしあ				& RUS									\\ \hline \\[-1em]
 北方領土					& ほっぽうりょうど 		& the northern territories				\\ \hline \\[-1em]
 日露和親条約				& にちろわしんじょうやく		& RUS-JAP peace treaty					\\ \hline \\[-1em]
 中国					& ちゅうごく				& CHI 									\\ \hline \\[-1em]
 尖閣諸島					& せんかくしょとう			& senkaku islands 						\\ \hline \\[-1em]
 台湾					& たいわん				& Taiwan 								\\ \hline \\[-1em]
 韓国					& かんこく 				& South Korea							\\ \hline \\[-1em]
 竹島					& たけしま				& Liancourt Rocks 						\\ \hline \\[-1em]
 慰安婦問題				& いあんふもんだい		& comfort women problem					\\ \hline \\[-1em]
 北朝鮮					& きたちょうせん			& North Korea 							\\ \hline \\[-1em]
 防衛問題					& ぼうえいもんだい		& Defence Problem						\\ \hline \\[-1em]
 アメリカ					& あめりか				& USA	 								\\ \hline \\[-1em]
 貿易問題					& ぼうえきもんだい		& international trade problem			\\ \hline \\[-1em]
 ODA					& せいふかいはつえんじょ 	& Official Development Assistance 		\\ \hline \\[-1em]
 国際連合					& こくさいれんごう 		& United Nations 						%\\ \hline \\[-1em]
\end{tabular}
%\end{center}
\section{\ruby{宗教}{しゅうきょう} \tabni religion | \ruby{社会思想}{しゃかいしそう} \tabni societal-thought }
\begin{tabular}{ l | p{6.7cm} | p{7.5cm} }
\ruby{社会}{しゃかい} & しゃかい 				& society								\\ \hline \\[-1em]
宗教 			& しゅうきょう 				& religion								\\ \hline \\[-1em]
思想	 			& しそう 					& thought; idea; ideology				\\ \hline \\[-1em]
恩 				& おん 					& favour								\\ \hline \\[-1em]
義理				& ぎり 					& duty; honour							\\ \hline \\[-1em]
義理チョコ			& ぎりちょこ 				& obligatory-gift chocolate				\\ \hline \\[-1em]
人情				& にんじょう					& humanity								\\ \hline \\[-1em]
恥 				& はじ 					& shame									\\ \hline \\[-1em]
根回し 			& ねまわし 				& making necessary arrangements			\\ \hline \\[-1em]
本音				& ほんね 					& real intention; motive				\\ \hline \\[-1em]
建前				& たてまえ 				& face; official stance					\\ \hline \\[-1em]
年功序列 			& ねんこうじょれつ 			& seniority by length of service		\\ \hline \\[-1em]
村八分 			& むらはちぶ 				& ostracism								\\ \hline \\[-1em]
賄賂、袖の下		& わいろ, そでのした 			& bribe									\\ \hline \\[-1em]
心付け 			& こころづ 					& gratuity; tip							\\ \hline \\[-1em]
祝儀、ご祝儀 		& しゅうぎ, ごしゅうぎ 			& congratulations						\\ \hline \\[-1em]
お祝い			& おいわい 				& congratulation						\\ \hline \\[-1em]
お祝いの贈り物 		& おいわいのおくりもの 			& congratulations-present				\\ \hline \\[-1em]
お祝いの品 		& おいわいのひん				& congratulations-gift					\\ \hline \\[-1em]
(お)中元			& (お)ちゅうげん 			& Bon Festival, Bon Festival gifts		\\ \hline \\[-1em]
(お)歳暮 			& (お)せいぼ 				& end of the year, year-end gift		\\ \hline \\[-1em]
お返し 			& おかえし 				& return gift							\\ \hline \\[-1em]
年賀状 			& ねんがじょう 				& New Year's card						\\ \hline \\[-1em]
名刺				& めいし 					& business card 						\\ \hline \\[-1em]
神道 			& しんとう 					& Shinto								\\ \hline \\[-1em]
神社 			& じんじゃ 					& Shinto shrine							\\ \hline \\[-1em]
鳥居 			& とりい 					& Shinto shrine archway					\\ \hline \\[-1em]
仏教 			& ぶっきょう 				& Buddhism								\\ \hline \\[-1em]
(お)寺、寺院、仏閣	& (お)てら, じいん, ぶっかく 	& temple								\\ \hline \\[-1em]
三門				& さんもん 				& large triple gate to temple			\\ \hline \\[-1em]
神仏習合			& しんぶつしゅうごう			& syncretism of Shinto and Buddhism		\\ \hline \\[-1em]
神仏分離			& しんぶつぶんり 				& separation of Buddhism and Shintoism	\\ \hline \\[-1em]
キリスト教 			& キリストきょう 				& Christianity							\\ \hline \\[-1em]
教会 			& きょうかい 				& church								\\ \hline \\[-1em]
右翼				& うよく 					& right-wing (politics)					\\ \hline \\[-1em]
神風				& かみかぜ 				& divine wind							\\ \hline \\[-1em]
切腹 			& せっぷく 					& ritual suicide 					   %\\ \hline \\[-1em]
\end{tabular}
\section{\ruby{教育}{きょういく} \tabni education }
\begin{tabular}{ l | p{6.7cm} | p{7.5cm} }
教育 &  きょういく  & education \\ \hline \\[-1em]
文部科学省、文科省 &  もんぶかがくしょう, もんかしょう  & Ministry of Education, Culture, Sports, Science and Technology; MEXT \\ \hline \\[-1em]
学校制度 &  がっこう せいど   & school-system \\ \hline \\[-1em]
義務教育 &  ぎむきょういく  & compulsory education \\ \hline \\[-1em]
小学校 & しょうがっこう & primary/elementary/grade - school \\ \hline \\[-1em]
中学校 & ちゅうがっこう & middle school \\ \hline \\[-1em]
高等学校、高校 &  こうとうがっこう /  こうこう  & high school \\ \hline \\[-1em]
大学 &  だいがく  & university/college\\ \hline \\[-1em]
大学院  & だいがくいん  & graduate school \\ \hline \\[-1em]
修士課程  &  しゅうしかてい  & master's course \\ \hline \\[-1em]
博士課程  &  はくしかてい  & doctoral course \\ \hline \\[-1em]
国立大学 &  こくりつだいがく  & national university \\ \hline \\[-1em]
公立大学 &  こうりつだいがく  & public university\\ \hline \\[-1em]
私立大学 &  しりつだいがく & private university\\ \hline \\[-1em]
帝国大学、旧帝大 &  ていこくだいがく/きゅうていだい  & (former) imperial university \\ \hline \\[-1em]
七帝大 &   ななていだい & national university  \\ \hline \\[-1em]
北海道大学 & ほっかいどうだいがく & Hokkaido University (Baa)\\ \hline \\[-1em]
東北大学 &  とうほくだいがく  & Northeastern University (A)\\ \hline \\[-1em]
東京大学 & とうきょうだいがく & Tokyo University (Aaa 1st)\\ \hline \\[-1em]
名古屋大学 &  なごやだいがく & Nagoya University (A)\\ \hline \\[-1em]
京都大学 &  きょうと だいがく& Kyoto University (Aaa 2nd)\\ \hline \\[-1em]
大阪大学 &  おおさか だいがく& Osaka University (Aa)\\ \hline \\[-1em]
九州大学 &  きゅうしゅうだいがく& Kyuushuu University (A)\\ \hline \\[-1em]
早稲田大学 & わせだだいがく& Waseda University (A 8th)\\ \hline \\[-1em]
慶応大学 &  けいおう だいがく& Keio University \\ \hline \\[-1em]
上智大学 &  じょうちだいがく& Sophia University (Baa {Same as Würzburg})\\ \hline \\[-1em]
進学率 &  しんがくりつ  & rate of advancement to the next higher level of education\\ \hline \\[-1em]
塾 & じゅく &  cram(ming) school \\ \hline \\[-1em]
家庭教師 &  かていきょうし  & private tutor\\ \hline \\[-1em]
浪人 &  ろうにん & ronin \\ \hline \\[-1em]
予備校 &  よびこう & prep school (ronin year) \\ \hline \\[-1em]
教育費 &  きょういくひ & education or school expenses\\ \hline \\[-1em]
学費  &  がくひ & tuition; school expenses\\ \hline \\[-1em]
授業料 &  じゅぎょうりょう & tuition fee; course fee\\ \hline \\[-1em]
生活費 &  せいかつひ & living expenses \\ \hline \\[-1em]
学閥 &  がくばつ & alma mater clique; old school tie %\\ \hline \\[-1em]
\end{tabular}
\section{\ruby{経済}{けいざい} \tabni economy }
\begin{tabular}{ l | l | p{7.5cm} }
企業 & きぎょう & business \\ \hline \\[-1em]
大企業 & だいきぎょう & large company \\ \hline \\[-1em]
中小企業 & ちゅうしょうきぎょう & Small and medium enterprises \\ \hline \\[-1em]
豊田自動車 & とよだじどうしゃ & Toyota automobiles \\ \hline \\[-1em]
任天堂 & にんてんどう & Nintendo \\ \hline \\[-1em]
ソニー & そにい & Sony \\ \hline \\[-1em]
財閥 & ざいばつ & financial conglomerate \\ \hline \\[-1em]
三菱 & みつびし & Mitsubishi \\ \hline \\[-1em]
三井 & みつい & Mitsui \\ \hline \\[-1em]
住友 & すみとも & Sumitomo \\ \hline \\[-1em]
財閥解体 & ざいばつかいたい & dissolution of zaibatsu \\ \hline \\[-1em]
六大グループ & ろくだい-ぐるうぷ & the six elements group \\ \hline \\[-1em]
三菱グループ & みつびし-ぐるうぷ & Mitsubishi Group \\ \hline \\[-1em]
三井グループ & みつい-ぐるうぷ & Mitsui group \\ \hline \\[-1em]
住友グループ & すみとも-ぐるうぷ & Sumitomo Group \\ \hline \\[-1em]
バブル経済 & バブルけいざい & Economic bubble \\ \hline \\[-1em]
三大メガバンク & さんだいめがばんく & the big three banks \\ \hline \\[-1em]
三菱UFJフィナンシャル && Mitsubishi UFJ Financial (Group) \\ \hline \\[-1em]
三井住友フィナンシャル && Sumitomo Mitsui Financial (Group) \\ \hline \\[-1em]
みずほフィナンシャル && Mizuho Financial (Group) \\ \hline \\[-1em]
ソフトバンク & そふとばんく & SoftBank (group) \\ \hline \\[-1em]
GDP & じー でぃーぴー & GDP \\ \hline \\[-1em]
経済成長率 & けいざいせいちょうりつ & rate of economic growth \\ \hline \\[-1em]
労働時間 & ろうどうじかん & working hours \\ \hline \\[-1em]
失業率 & しつぎょうりつ & unemployment rate \\ \hline \\[-1em]
非正社員 & ひせいしゃいん & non-fulltime/permanent employee\\ \hline \\[-1em]
為替レート & かわせれえと & exchange rate \\ \hline \\[-1em]
外貨準備高 & がいかじゅんびだか & total amount of foreign currency reserves \\ \hline \\[-1em]
貿易(相手国) & ぼうえき(あいてこく) & International trade (partner country)\\ \hline \\[-1em]
輸出 & ゆしゅつ & export \\ \hline \\[-1em]
輸入 & ゆにゅう & import \\ % \\[-1em] %\hline
\end{tabular}
\section{\ruby{文化}{ぶんか} \tabni culture }
\begin{tabular}{ l | l | p{11.5cm} }
衣 & い (ころも) & clothes \\ \hline \\[-1em]
体格 &  たいかく  & physique; constitution \\ \hline \\[-1em]
平均寿命&  へいきんじゅみょう & life expectancy \\ \hline \\[-1em]
死亡率 &  しぼうりつ  & death rate; mortality  \\ \hline \\[-1em]
着物 &  きもの & kimono jap-clothing; jap-clothes  \\ \hline \\[-1em]
和服 &  わふく & traditional Japanese clothes \\ \hline \\[-1em]
冠婚葬祭 &  かんこんそうさい & important ceremonial occasions in family relationships \\ \hline \\[-1em]
七五三 & しちごさん & festival (shrine visit) by children aged 7, 5 and 3 \\ \hline \\[-1em]
成人式 &  せいじんしき & coming-of-age ceremony \\ \hline \\[-1em]
卒業式 & そつぎょうしき & graduation ceremony \\ \hline \\[-1em]
結婚式 &  けっこんしき & marriage ceremony; wedding; nuptials \\ \hline \\[-1em]
披露宴 &   ひろうえん  & reception (wedding) \\ \hline \\[-1em]
帯 & おび & obi; kimono sash  \\ \hline \\[-1em]
紋付羽織袴 &もんつきはおりはかま& Japanese male formal attire decorated with one's family crest \\ \hline \\[-1em]
紋付& もんつき &clothing (e.g. kimono) decorated with one's family crest \\ \hline \\[-1em]
紋&もん&(family) crest; coat of arms \\ \hline \\[-1em]
羽織 & はおり &haori (Japanese formal coat) \\ \hline \\[-1em]
袴 & はかま & hakama; man's formal divided skirt \\ \hline \\[-1em]
足袋 & たび & tabi; Japanese socks (with split toe) \\ \hline \\[-1em]
雪駄 & せった & leather-soled sandals \\ \hline \\[-1em]
下駄 & げた & geta; Japanese wooden clogs \\ \hline \\[-1em]
草履 & ぞうり & zori; Japanese sandals \\ \hline \\[-1em]
振袖 & ふりそで & long-sleeved kimono \\ \hline \\[-1em]
留袖 & とめそで & formal, usually black, kimono with designs along the bottom of the skirt worn by married women on ceremonial occasions; \\ \hline \\[-1em]
日本髪 &にほんがみ&traditional Japanese women's hairstyle \\ \hline \\[-1em]
浴衣 & ゆかた & yukata (light cotton kimono worn in the summer or used as a bathrobe) \\ \hline \\[-1em]
手拭 & てぬぐい & japanese (hand) towel \\ \hline \\[-1em]
うちわ & うちわ &  uchiwa; type of traditional Japanese handheld fan  \\ \hline \\[-1em]
扇子 &  せんす  & folding fan \\ \hline \\[-1em]
懐紙 &  かいし & paper folded and tucked inside the front of one's kimono (esp. for use at the tea ceremony) %\\ \hline \\[-1em]
\end{tabular}
\section{\ruby{食}{しょく} \tabni food}
\begin{tabular}{ l | l | p{11.5cm} } %TODO continue ANKI
ご飯		&ごはん   	&cooked rice \\ \hline \\[-1em]
米		&こめ       	&uncooked rice \\ \hline \\[-1em]
稲		&いね      	&rice plant \\ \hline \\[-1em]
日本料理 	&にほんりょうり 	&japanese cuisine \\ \hline \\[-1em]
和食		&わしょく  		&japanese food \\ \hline \\[-1em]
どんぶり  	&    		&porcelain bowl \\ \hline \\[-1em]
丼物  	&どんぶりもの  	&bowl of rice with food on top \\ \hline \\[-1em]
なべ  	&    	  	&stew; hot pot \\ \hline \\[-1em]
鍋料理  	&なべりょうり 	&hot pot cooking \\ \hline \\[-1em]
しゃぶしゃぶ	&        	&thinly sliced meat boiled quickly with vegetables \\ \hline \\[-1em]
すき焼き	&すきやき    	&sukiyaki; thin slices of beef \\ \hline \\[-1em]
天ぷら 	&てんぷら   	&tempura; deep-fried fish and vegetables in a light batter \\ \hline \\[-1em]
刺身   	&さしみ       	&sashimi (raw sliced fish, shellfish or crustaceans) \\ \hline \\[-1em]
会席料理 	&かいせきりょうり 	&set of dishes served on an individual tray for entertaining guests; banquet \\ \hline \\[-1em]
寿司   	&すし       	&anything made with vinegared rice \\ \hline \\[-1em]
にぎり寿司 	&にぎりずし   	&nigirizushi /hand-formed sushi with a topping of seafood \\ \hline \\[-1em]
巻き寿司 	&まきずし     	&makizushi; sushi made rolled in nori seaweed with a core of filling \\ \hline \\[-1em]
手巻き寿司	&てまきずし   	&hand-rolled sushi (usually cone-shaped) \\ \hline \\[-1em]
稲荷寿司 	&いなりずし  	&sushi wrapped in fried tofu \\ \hline \\[-1em]	
シャリ    	&         	&sushi rice; rice seasoned with vinegar, sugar and salt \\ \hline \\[-1em]
ムラサキ  	&      		&murasaki /Purple Gromwell \\ \hline \\[-1em]
醤油   	&しょうゆ     	&soy sauce \\ \hline \\[-1em]
(サビ)わさび	&     	&(at a sushi shop) wasabi \\ \hline \\[-1em]
アガリ   	&       	& \\ \hline \\[-1em]
お茶  	&おちゃ  	 	&green tea \\ \hline \\[-1em]													%is voc
抹茶   	&まっちゃ   	&matcha; powdered green tea   \\ \hline \\[-1em]								%is voc
玉露   	&ぎょくろ   	&high-quality green tea  \\ \hline \\[-1em]										%is voc
煎茶   	&せんちゃ   	&green leaf tea; non-powdered tea (as opposed to matcha) \\ \hline \\[-1em]		%is voc
番茶 	&ばんちゃ   	&coarse tea \\ \hline \\[-1em]													%is voc
麺    	&めん     	&noodles \\ \hline \\[-1em]														%is voc
饂飩    	&うどん     	&udon; thick Japanese wheat noodles \\ \hline \\[-1em] 							%is voc
そうめん  	&         	&fine white noodles \\ \hline \\[-1em]							
そば     	&         	&soba; Japanese buckwheat noodles \\ \hline \\[-1em]
ラーメン   	&         	&ramen; Chinese-style noodles \\ \hline \\[-1em]
日本酒   	&にほんしゅ    	&sake; Japanese rice wine \\ \hline \\[-1em]									%is voc
焼酎   	&しょうちゅう   	&Japanese spirit distilled from sweet potatoes, rice, etc.   \\ \hline \\[-1em]	%is voc
ビール   	&        	&beer \\ \hline \\[-1em]
発泡酒  	&はっぽうしゅ          	&happoshu; low-malt beer-like beverag /sparkling wine \\ \hline \\[-1em]
第三のビール	&だいさんのビール   	&cheap, malt-free beer-like alcoholic beverage \\ \hline \\[-1em]
みりん       	&              	&mirin; type of sweet rice wine used in cooking \\ \hline \\[-1em]
酢          	&す           	&vinegar \\ \hline \\[-1em]												%is voc
味噌       	&みそ            	&miso; fermented condiment usu. made from soybeans  \\ \hline \\[-1em]
出汁        	&だし            	&dashi (Japanese soup stock made from fish and kelp) \\ \hline \\[-1em]
味の素       	&あじのもと       	&Ajinomoto, brand name of monosodium glutamate (MSG) \\ \hline \\[-1em]
梅干し       	&うめぼし         	&umeboshi; pickled ume; pickled plum  \\ \hline \\[-1em]				%is voc
漬け物       	&つけもの      	&tsukemono; Japanese pickled vegetables  \\ \hline \\[-1em]
\end{tabular}
\section{\ruby{住}{じゅう} \tabni housing}
\begin{tabular}{ l | l | p{10.5cm} }
玄関        	&げんかん   		&entry hall; vestibule; foyer; entryway;    \\ \hline \\[-1em]
畳         	&たたみ     		&tatami mat; Japanese straw floor coverings \\ \hline \\[-1em]
座布団     	&ざぶとん   		&zabuton (flat floor cushion for sitting or kneeling) \\ \hline \\[-1em]
正座    		&せいざ 			&seiza; kneeling with the tops of the feet flat on the floor\\ \hline \\[-1em]
あぐら      	&           	&sitting cross-legged {lit. foreign/barbarian sitting} \\ \hline \\[-1em]
座椅子      	&ざいす     		&tatami chair; legless chair \\ \hline \\[-1em]
布団        	&ふとん     		&futon (quilted Japanese-style mattress laid out on the floor) \\ \hline \\[-1em]
空調        	&くうちょう 			&air conditioning /AC \\ \hline \\[-1em]
エアコン    	&えあこん   		&air conditioning /AC \\ \hline \\[-1em]
クーラー    	&くうらあ   		&air conditioning /AC \\ \hline \\[-1em]
扇風機      	&せんぷうき 		&electric fan \\ \hline \\[-1em]
風鈴        	&ふうりん   		&wind chime; wind bell \\ \hline \\[-1em]
金魚        	&きんぎょ   		&goldfish \\ \hline \\[-1em]
こたつ      	&           	&kotatsu; table over an electric heater \\ \hline \\[-1em]
火鉢        	&ひばち     		&brazier; hibachi \\ \hline  \\[-1em]
環境問題     	&かんきょうもんだい   	&environmental issue \\ \hline \\[-1em]
エネルギー問題  &えねるぎいもんだい   	&energy problems \\ \hline \\[-1em]
発電所      	&はつでんしょ      	&power plant; power station \\ \hline \\[-1em]
原子力発電所  	&げんしりょくはつでんしょ	&nuclear power plant; nuclear power station \\ \hline \\[-1em]
原発        	&げんぱつ       	&nuclear power plant; nuclear power generation \\ \hline \\[-1em]
原子力発電   	&げんしりょくはつでん  	&nuclear power/electricity generation \\ \hline \\[-1em]
地熱        	&じねつ          	&geothermy; terrestrial heat \\ \hline \\[-1em]
太陽光       	&たいようこう       	&sunlight \\ \hline \\[-1em]
風力        	&ふうりょく         	&wind power \\ \hline \\[-1em]
潮力        	&ちょうりょく        	&tidal energy \\ \hline \\[-1em]
火力        	&かりょく          	&heating power; steam power; || military: firepower \\ \hline \\[-1em]
水力        	&すいりょく         	&hydropower; water power \\ \hline \\[-1em]
二酸化炭素   	&にさんかたんそ    	&carbon dioxide \\ \hline \\[-1em]
地球温暖化   	&ちきゅうおんだんか   	&global warming \\ \hline \\[-1em]
黄砂        	&こうさ          	&loess; bai; yellow river dust/sand; asian dust \\ \hline \\[-1em]
PM2.5      	&              	&particles smaller than 2.5 micrometers \\[-1em]
\end{tabular}
\section{\ruby{歴史}{れきし} \tabni history}
\begin{tabular}{ l | l | p{10.5cm} }
時代       	&じだい 			&period; epoch; era; age \\ \hline \\[-1em]
世紀       	&せいき 			&century	\\ \hline \\[-1em]
縄文時代		&じょうもんじだい		&Jōmon period (c. 14k BCE-300 BCE)\\ \hline \\[-1em]
弥生時代		&やよいじだい		&Yayoi period (c. 300 BCE-300 CE) \\ \hline \\[-1em]
大和時代		&やまとじだい		&Yamato period (300-550 CE) \\ \hline \\[-1em]
飛鳥時代		&あすかじだい		&Asuka period (550-710 CE) \\ \hline \\[-1em]
奈良時代		&ならじだい			&Nara period (710-794 CE) \\ \hline \\[-1em]
平安時代		&へいあんじだい		&Heian period (794-1185 CE) \\ \hline \\[-1em]
藤原時代		&ふじわらじだい		&Fujiwara period (894-1185 CE) \\ \hline \\[-1em]
院政時代		&いんせいじだい		&era of government by cloistered emperors \\ \hline \\[-1em]
鎌倉時代		&かまくらじだい		&Kamakura period (1185-1333 CE) \\ \hline \\[-1em]
南北朝時代	&なんぼくちょうじだい	&period of the northern and southern dynasties (1336-92 CE) \\ \hline \\[-1em]
室町時代		&むろまちじだい		&Muromachi period (1333-1573 CE) \\ \hline \\[-1em]
戦国時代		&せんごくじだい		&Warring States period (1467-1568 CE) \\ \hline \\[-1em]
安土桃山時代	&あづちももやまじだい	&Azuchi-Momoyama period (1558-1600 CE) \\ \hline \\[-1em]
江戸時代		&えどじだい			&Edo period (1603-1868 CE) \\ \hline \\[-1em]
明治時代		&めいじじだい		&Meiji period (1868-1912) \\ \hline \\[-1em]
大正時代		&たいしょうじだい		&aisho period (1912-1926 CE) \\ \hline \\[-1em]
昭和時代		&しょうわじだい		&Showa period (1926-1989 CE) \\ \hline \\[-1em]
平成時代		&へいせいじだい		&Heisei period (1989-) \\ \hline \\[-1em]
%今上天皇		&きんじょうてんのう		&the present Emperor \\ \hline \\[-1em]	%have this voc already 

\end{tabular} \\ \\ \\ \\
\begin{tabular}{ l | p{10.5cm} }
年表 Nenpyô		&chronological table \\ \hline \\[-1em]
西紀 Seiki、西暦 Seireki		&Christian Era; CE; anno domini; AD \\ \hline \\[-1em] %Use Seireki
干支 						&Eto (Chinese 10 Basic Elements and 12 Zodiac Signs)\\ \hline \\[-1em]
東北 						&Tôhoku (North East JAP, 2011 Earthquake)\\ \hline \\[-1em]
関東大震災  &Kantō daishinsai (1923 Great Kantō Earthquake) \\ \hline \\[-1em]
朝鮮 						&Chôsen	 (Korea)\\ \hline \\[-1em]
日紀 						&Nikki (1st Year of Jap Calendar (660 BC))\\ \hline \\[-1em]
前660年 						&Zen 660 nen (660 BC)\\ \hline \\[-1em]
紀元前660年 					&Kigen-zen 660 nen (660 BC)\\ \hline \\[-1em]
周  							&Shû (Name of CHI Dynasty)\\ \hline \\[-1em]            %Do not need to learn!
惠王 						&Kei-ô (Name of CHI Dynasty)\\ \hline \\[-1em]          %Do not need to learn!
神武(天皇)					&Jinmu-tennô  (First JAP Tennô)\\ \hline \\[-1em]
中華人民共和國 				&Chûka-jinmin-kyôwa-koku	(Peoples Republic of CHI) \\ \hline \\[-1em]
中華民國(台灣)					&Chûka-min-koku (Republic of CHI {Taiwan})\\ \hline \\[-1em]
朝鮮民主主義人民共和國(北朝鮮) 	&Chôsen-minshu-shugi-jinmin-kyôwa-koku (Democratic People's Republic of Korea), Kita chôsen (North Korea)\\ \hline \\[-1em]
大韓民國(南朝鮮) 				& Dai-kann-min-koku (South Korea), Minami chôsen	\\ \hline \\[-1em]
昭和時代 						&Shôwa-jidai \\ \hline \\[-1em]
昭和天皇 						&Shôwa-tennô	 \\ \hline \\[-1em]
平成時代 						&Heisei-jidai	 \\ \hline \\[-1em]
今上天皇 						&Kinjô-tennô	\\ \hline \\[-1em]
2017年 						&Ni-sen-jûnana-nen  \\ \hline \\[-1em]
平成29年 					&Heisei-nijû-kyû-nen			 \\[-1em]
\end{tabular} \\
\begin{tabular}{ l | l | p{10.5cm} }
暦		&こよみ			&calendar; almanac \\ \hline \\[-1em]
十干		&じっかん			&the 10 basic elements \\ \hline \\[-1em]
甲		&きのえ			&1st element (elder brother wood) \\ \hline \\[-1em]
乙		&きのと			&2nd element (younger brother wood)\\ \hline \\[-1em]
丙		&ひのえ (へい)		&3rd element (elder brother fire) \\ \hline \\[-1em]
丁		&ひのと (てい)		&4th element (younger brother fire) \\ \hline \\[-1em]
戊		&つちのえ			&5th element (elder brother earth) \\ \hline \\[-1em]
己		&つちのと (き)		&6th element (younger brother earth) \\ \hline \\[-1em]
庚		&かのえ			&7th element (elder brother metal) \\ \hline \\[-1em]
辛		&かのと			&8th element (younger brother metal) \\ \hline \\[-1em]
壬		&みずのえ (じん)	&9th element (elder brother water) \\ \hline \\[-1em]
癸		&みずのと (き)		&10th element (younger brother water) \\ \hline \\[-1em]
十二支	&じゅうにし 			&12 signs of the Chinese zodiac \\ \hline \\[-1em]
子		&ね				&1st sign CHI zodiac (The Rat, 11pm-1am, north, November) \\ \hline \\[-1em]
丑		&うし				&2nd sign CHI zodiac (The Ox, 1am-3am, north-northeast, December) \\ \hline \\[-1em]
寅		&とら				&3rd sign CHI zodiac (The Tiger, 3am-5am, east-northeast, January) \\ \hline \\[-1em]
卯		&う				&4th sign CHI zodiac (The Hare, 5am-7am, east, February) \\ \hline \\[-1em]
辰		&たつ				&5th sign CHI zodiac (The Dragon, 7am-9am, east-southeast, March) \\ \hline \\[-1em]
巳		&み				&6th sign CHI zodiac (The Serpent, 9am-11am, south-southeast, April) \\ \hline \\[-1em]
午		&うま				&7th sign CHI zodiac (The Horse, 11am-1pm, south, May) \\ \hline \\[-1em]
未		&ひつじ			&8th sign CHI zodiac (The Ram, 1pm-3pm, south-southwest, June) \\ \hline \\[-1em]
申		&さる				&9th sign CHI zodiac (The Monkey, 3pm-5pm, west-southwest, July) \\ \hline \\[-1em]
酉		&とり				&10th sign CHI zodiac (The Bird, 5pm-7pm, west, August) \\ \hline \\[-1em]
戌		&いぬ				&11th sign CHI zodiac (The Dog, 7pm-9pm, west-northwest, September) \\ \hline \\[-1em]
亥		&い				&12th sign CHI zodiac (The Boar, 9pm-11pm, north-northwest, October) \\ \hline \\[-1em]
丙午		&ひのえうま			&43rd year of the sexagenary cycle (year of the Fire Horse, 
							renowned for disasters and the birth of women destined to kill their husbands) \\[-1em] 
%1960年の干支は。		&		& \\[-1em]
%1996年の干支は。		&		& \\[-1em]
\end{tabular}
\section{\ruby{絵画}{かいが} \tabni picture /painting}
\begin{tabular}{ l | l | p{10.5cm} }
大和絵			&	やまとえ			& Yamato-e (classical Japan decorative paintings) \\ \hline \\[-1em]
土佐派			&	とさは				& Tosa school \\ \hline \\[-1em]
狩野派			&	かのは			& Kanō school \\ \hline \\[-1em]
山水画			&	さんすいが			& landscape picture \\ \hline \\[-1em]
人物画			&	じんぶつが			& portrait painting \\ \hline \\[-1em]
花鳥画			&	かちょうが			& painting of flowers and birds \\ \hline \\[-1em]
鳥獣人物戯画		&	ちょうじゅうじんぶつぎが	& picture scrolls, or emakimono, belonging to Kōzan-ji temple \\ \hline \\[-1em]
水墨画			&	すいぼくが			& India-ink painting \\ \hline \\[-1em]
洛中洛外図		&	らくちゅうらくがいず		& views in and around the city of Kyoto \\ \hline \\[-1em]
南蛮屏風			&	なんばんびょうぶ		& southern barbarian folding screen \\ \hline \\[-1em]
浮世絵			&	うきよえ			& ukiyoe (color print of everyday life in the Edo period) (colour) \\ \hline \\[-1em]
喜多川歌麿		&	きたがわうたまろ		& Utamaro (famous painter) \\ \hline \\[-1em]
東洲斎写楽		&	とうしゅうさいしゃらく	& Toshusai Sharaku (Edo period painter) \\ \hline \\[-1em]
葛飾北斎			&					& \\ \hline \\[-1em]
安藤広重(歌川広重)	&					& (Hiroshige) %\\ \hline \\[-1em]
\end{tabular}
\section{\ruby{音楽}{おんがく} \tabni music}
\begin{tabular}{ l | l | p{10.5cm} }
楽器	& がっき & musical instrument \\ \hline \\[-1em]
雅楽	& ががく & old Japanese court music \\ \hline \\[-1em]
琵琶 & びわ & biwa (Japanese lute) \\ \hline \\[-1em]
曲項、曲項琵琶 & きょくこう,(びわ-) & composition-paragraph, (biwa-) \\ \hline \\[-1em]
五絃、五絃琵琶 & ごげん, (びわ-) & five strings, (biwa-) \\ \hline \\[-1em]
阮咸 & げんかん & ruanxian (4 or 5-stringed Chinese lute); ruan  \\ \hline \\[-1em]
箜篌 & くご & konghou (ancient Chinese harp) \\ \hline \\[-1em]
鳳首箜篌 & ほうしゅくご &  \\ \hline \\[-1em]
臥箜篌 & がくご & \\ \hline \\[-1em]
琴 & こと & 13-stringed Japanese zither \\ \hline \\[-1em]
瑟 & しつ & ancient Chinese plucked zither, usu. with 25 or 23 strings \\ \hline \\[-1em]
箏 & ひさご & gourd \\ \hline \\[-1em]
筑 & ちく & an ancient musical instrument \\ \hline \\[-1em]
匏琴 & & \\ \hline \\[-1em]
篳篥 & ひちりき & small double-reed wind instrument used in gagaku  \\ \hline \\[-1em]
笙 & しょう & traditional Japanese wind instrument resembling panpipes \\ \hline \\[-1em]
簫、縦笛 & しょう, たてぶえ & xiao (chi bamboo flute), end-blown flute \\ \hline \\[-1em]
龍笛、横笛 & りゅうてき,  よこぶえ & transverse flute (e.g. a fife) \\ \hline \\[-1em]
銅鈸 & どうはし & copper cymbals \\ \hline \\[-1em]
拍 & はく & beat \\ \hline \\[-1em]
方響 & ほうきょう & sound direction \\ \hline \\[-1em]
鼓 & つづみ & hand drum \\ \hline \\[-1em]
太鼓 & たいこ & drum \\ \hline \\[-1em]
三味線 & しゃみせん & shamisen /three-stringed Japanese lute \\ \hline \\[-1em]
国歌 & こっか & national anthem \\ \hline \\[-1em]
君が代 & きみがよ & Imperial reign %\\ \hline \\[-1em]
\end{tabular}
\end{document}
