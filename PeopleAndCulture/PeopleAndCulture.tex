\documentclass{article}
\usepackage{xeCJK}
\usepackage{enumitem}
\usepackage{geometry}
\usepackage{titlesec}
\usepackage[dvipsnames]{xcolor}
\usepackage{lipsum}
\usepackage[CJK,overlap]{ruby}
 \geometry{
 a4paper,
 %total={170mm,257mm},
 left=10mm,
 top=12mm,
 right=10mm,
 bottom=12mm,
 }
 
\renewcommand{\rubysep}{0.1ex}%3.9ex}
\makeatletter
\renewcommand\paragraph{\@startsection{paragraph}{4}{\z@}%
            {-0.5ex\@plus -.1ex \@minus -.25ex}%
            {-1em}%
            {\normalfont\normalsize\bfseries}}
\renewcommand\subparagraph{\@startsection{subparagraph}{5}{\z@}%
            {-0.1ex\@plus -.1ex \@minus -.25ex}%
            {-1em}%
            {\normalfont\normalsize\bfseries}}
\makeatother

\setlist{nolistsep}
\titlespacing*{\section}
{0pt}{1.5ex plus 1ex minus .2ex}{1.3ex plus .2ex}
%{0pt}{5.5ex plus 1ex minus .2ex}{4.3ex plus .2ex}
\titlespacing*{\subsection}
{0pt}{0.5ex plus 1ex minus .2ex}{0.5ex plus .2ex}
\titlespacing*{\subsubsection}
{0pt}{0.1ex plus 1ex minus .2ex}{0.1ex plus .2ex}

\newcommand\tab[1][1cm]{\hspace*{#1}} 
\newcommand\tabi[1][0.05cm]{\hspace*{#1}}
\newcommand\tabni[1][0.2cm]{\hspace*{#1}}
\newcommand\tabyon[1][0.4cm]{\hspace*{#1}}
\newcommand\tabroku[1][0.6cm]{\hspace*{#1}}

\XeTeXlinebreaklocale "ja"
\XeTeXlinebreakskip=0em plus 0.1em minus 0.01em

\setlength{\parindent}{0pt}
\setCJKmainfont{ipaexm.ttf}
\setcounter{tocdepth}{5}

\begin{document}	%\large {1.5ex \@plus .2ex}%

\tableofcontents
\newpage

\part{\ruby{日本文化地理概論}{にほんぶんかちりがいろん}}
\section{\ruby{地理}{ちりし} \tabni geography | \ruby{自然}{しぜん} \tabni nature}
\begin{itemize}
\item Tidbits about Japanese geography:
\begin{itemize}
\item Japan has few natural resources. Only some oil.
\item Japan lies in the north pacific in the so called 'Sea of Japan'
\begin{itemize}
\item Koreans would prefer to call the 'Sea of Japan' the 'East Sea'
\item The straight between Japan and Korea is called the 'Straight of Korea'
\item The sea south of Korea is called the 'East Chinese Sea'
\end{itemize}
\item Japan has in general a subtropic clima
\item Japan has 5 neighbouring nations
\item The lake \ruby{琵琶湖}{びわこ} north east of Kyoto is Japan's largest lake
\end{itemize}
\end{itemize}
\subsection{The islands of Japan }
Japan consists of 4 main islands: 
\begin{itemize}
\item \ruby{本州}{ほんしゅう} the biggest and true main island.
\item A line on the island of 本州 from 京都 to 九州 contains the majority of Japanese industy and urbanization.
\item 福島 lies to the north of 東京.
\begin{itemize}
\item 本州 even literally means main island.
\end{itemize}
\item \ruby{北海道}{ほっかいど} the northern most island, home to the ainu.
\begin{itemize}
\item 北海道 means northern district.
\item 北海道 has colder clima than the rest of Japan.
\end{itemize}
\item \ruby{九州}{きゅうしゅう} the southernmost and right next to Korea.
\begin{itemize}
\item Originally 九州 meant '9 countries' they now make up 7 prefectures.
\end{itemize}
\item \ruby{四国}{しこく} the smallest islands, squished in between 本州 and 九州. 
\begin{itemize}
\item 四国 originally meant '4 countries', they now have become 4 prefectures instead.
\end{itemize}
\end{itemize} 
Further more there are the Ryukyu islands to the nations south, which since 1879 now form 沖縄. Which presumably means 'A Distant Land' \\
\begin{itemize}
\item Until 1879 Ryukyu was a tributary state to China.
\item China took Taiwan and Japan took Ryukyu in 1879.
\item Ryukyu served as intermediary port between China and Japan. The most shipped good was Seetang/Seeweed produced in Hokkaido and sold in China.
\item American troops are stationed on Okinawa.
\end{itemize}
\subsection{Prefectures and Cities}
Prefecture Division:
\begin{itemize}
\item 1 Metropolis: \ruby{東京・都}{とうきょう・と}
\item 1 Circuit: \ruby{北海・度}{ほっかい・ど}
\item 2 urban prefectures: \ruby{大阪・府}{おおさか・ふ} and \ruby{京都・府}{きょうと・ふ}
\item 43 prefectures \ruby{県}{けん} 
\end{itemize}
Places in Detail:
\begin{itemize}
\item Sapporo, host of the Winter Olympics 1972, is the political capital of Sapporo. Sapporo is a sistercity to Munich. Sapporo is famous for its beer. This beer is called Ebisu-Beer, it was founded by a German
\item Sendai us since the 17th century the political centre of north east Japan. Originally Fukushima used to have that role. 
\item Tokyo, which was originally called Edo, is the political capital of Japan since 1868.
\begin{itemize}
\item The Tenno now lives in Tokyo it thus can be regarded as capital.
\item Tokyo is a Metropolis, not a city.
\end{itemize}
\item Nagoya, is a city somewhat to the south of Tokyo and birthplace of Toyota. Nearby now lies the city of Toyota(-shi). Nagoya is political centre of the pacific coastal region.
\item Kyoto, it has been capital from 794 to 1868. 
\item Osaka lies to the south of Kyoto.  
\begin{itemize}
\item It is the financial center of western Japan since the 17th century.
\item It used to be Japan's most important harbour thanks to its proximity to Kyoto. 
\item Rice from the entire nation was collected as tax and brought to Osaka in the past.
\end{itemize}
\item Fukuoka, the largest city of Kyuushuu:
\begin{itemize}
\item Used to be very important until the 17th century because of trade with China and Korea
\item Lost its importance through the closure of the nation to the world from 1639-1854.
\end{itemize}
\item Hiroshima, lies at the south-west tip of Honshuu.
\begin{itemize}
\item Important for arms manufacture.
\item Destroyed by nuke in 1945
\item Selected as target as there was no war prisoner camp.
\end{itemize}
\item Nagasaki, lies on Kyuushuu to the south of Fukuoka.
\begin{itemize}
\item Destroyed 3 days after Hiroshima in 1945.
\item Used to be the only open port during the time of Sakoku(closed nation). Open only to the dutch, chinese and korean.
\item Nagasaki became unimportant in the 19th/20th century because of its distance to Kyoto and Tokyo.
\end{itemize}
\item Fukushima, not a city but rather a prefecture, it is home to a lot of elctric companies.
\end{itemize}
\subsection{People, and Nature}
\begin{itemize}
\item \ruby{人口}{じんこう} 2015: 127mil. \ruby{一億二千七百万人}{いちおくにせんななひゃくまん}
\item 6852 islands make uo Japan
\item 70% of Japan is mountainous
\item Fuji-san an active volcano is with 3776m the highest Mountain of Japan
\item There are 110 active volcanos in Japan, 47 of those are very active
\item Onsen:
\begin{itemize}
\item There are more than 2000
\item Kusatsu-onsen is very famous, it was introduced by German Doctor Erwin von Bälz, at the time doctor of the Imperial family. Thus started Japan's onsen culture. 
\end{itemize}
\item Earthquakes: There is a large one every century.
\item Tsunami: They happen regularly.
\item Temperatures:
\begin{itemize}
\item Between -40°C and 40°C.
\item Japan has a season of rain called: \ruby{梅雨}{つゆ}, the rain itself is called \ruby{梅雨}{ばいう}. 
\end{itemize}
\item Typhoon:
\begin{itemize}
\item Usually hit in August/September
\item Typhoon always 来る。
\item Storms are called typhoon in japan, hurricane in America and cyclone in India.
\end{itemize}
\item Animals:
\begin{itemize}
\item The largest animal Japan's is the Bear.
\item Japans most famous bear is the ツキノワグマ /月の輪熊, the 'Japanese Moon Bear', they have a white ring around there neck thus the name. (Officially called asian black bear).
\item The most destructive animal in Japan is the racoon, or アライグマ /洗い熊. (The racoons are seen as a plague)  
\end{itemize}
\item Birds:
\begin{itemize}
\item The national Bird is the Kiji キジ /雉 (Green Pheasant). It was choosen because it can't fly.
\item The crane 鶴(つる), or tanchoo-zuru.	Biological name: grus-japonensis
\item The ibis (crested ibis) or 鴇(トキ). Biological name: Nipponia nippon.
\item 鶯(うぐいす), is the most liked bird in Japan. (Japanese Bush Warbler /Grasmücke)
\end{itemize}
\item Shokubutsu
\begin{itemize}
\item Sugi: (conifer)(Japanese cedar) Cryptomeria japonica --> became less popular because of allergies.
\item Hinoki (cyprese) --> architecture material for important Buildings such as shinto-shrines.
\item Sakura (cherry) it is now the most liked tree since the 9th century in japan, --> there exists a tradition of 花見(はなみ) aka cherry-blossom-viewing
\item Ume, (plum) it was the most liked tree before the 9th century through the inluence of the chinese culture.
\item Kiku, (chrysanthemum) even though this flower originates in china it now represents the imperial family of japan. This causes it to be somewhat disliked.
\end{itemize}
\end{itemize}
\section{\ruby{政治}{せいじ} \tabni politics }
\subsection{Post war Japan and its constituion of peace}
\begin{itemize}
    \item 1946 -> Humanity Declaration -> Emperor not a God
    \item 1947 -> New Constitution (けんぽう)
    \begin{itemize}
            \item Old constitution -> てんのう as absolute ruler.
            \item New constitution -> てんのう as symbol of the state and the unity of the people
            \item Pacifist constitution that denounces further wars and even right out avoids the word war itself.
    \end{itemize}
    \item Prime Minister as of 2017: しんぞう あべ
\end{itemize}
\subsection{Japan's legislative (りっぽう) }
\begin{itemize}
    \item The Diet or こっかい
    \begin{itemize}
        \item It is made up of 2 Houses:
        \begin{itemize}
            \item しゅうぎいん (lower house, house of representatives)
            \begin{itemize}
                \item There are 475 members
                \item One has to be 18 to vote, and 25 to be voted for.
                \item Members are elected on 4 year terms.
                \item The current ruling coalition (2017) holds 326 seats.
            \end{itemize}
            \item さんぎいん (upper house, house of councillors)
            \begin{itemize}
                \item Has 242 members.
                \item One has to be 18 to vote, and 30 to be voted for.
                \item Members are elected on 5year terms.
                \item The coalition holds currently (2017) 116+20 seats.
            \end{itemize}
            \item The strongest party of the lower house rules. This is currently the LDP (JYMST)
        \end{itemize}
        \item Voting rights were established in 1950 (suffrage)
        \item To vote in a local election one has to be a 3month resident.
    \end{itemize}
    \item Japan's Parties:
    \begin{itemize}
        \item じゆうみんしゅうとう
        \begin{itemize}
            \item The Liberal Democratic Party.
            \item Abe is president of this party.
        \end{itemize}
        \item こうめいとう
        \begin{itemize}
            \item The party of public brightness.
            \item Highly religious party, affiliated with Buddhist organisation  そうかがっかい .
            \item Chief representative is Yamaguchi, Natsuo.
        \end{itemize}
        \item にほんきょうさんとう
        \begin{itemize}
            \item The communist party.
            \item Chairman: Shii, Kazuo
        \end{itemize}
        \item みんしんとう
        \begin{itemize}
            \item Democratic Party.
            \item Party of the new elites.
            \item (Murata) Renh$ \overline{o} $ is president. He is also a member of the さんぎいん.
        \end{itemize}
        \item いしんのとう
        \begin{itemize}
            \item Japan Innovation Party
            \item Party of Osaka
            \item President is the Mayor of Osaka: Hashimoto, T$\overline{o}$ru
        \end{itemize}
        \item  
    \end{itemize}
    \item Temporal Employment Introduced -> Increased the number of the poor.
    \item さいこうさいばんしょ
    \begin{itemize}
        \item さいこう = best/highest,   さいばん = court, しょ = place
    \end{itemize}
\end{itemize}
\subsection{Japan's Emperor}
\begin{itemize}
    \item 23.Dec.1933 てんのうAkihito born as son of しょうわてんのう.
    \item 1989 てんのうAkihito becomes ruling てんのうへいか <-> honorific <-> こうごうへいか
    \item $\overline{O}$kimi <- originally used as the imperial family name.
    \item The now used Name is -> (of) Mikado <- meaning: gate (of the imperial house)
    \item Current Empress: Michiko
    \begin{itemize}
        \item She is considered a normal person (commoner). As such she is very popular.
    \end{itemize}
    \item There is a law as of 1947 that the emperor has to be male.
    \item さんしゅのじんぎ
    \begin{itemize}
        \item The 3 Imperial Regalia of Japan.
        \item Mirror --> Claimed to be housed in the Ise Grand Shrine in the Mie Prefecture.
        \item Sword Kusanagi --> Claimed to be housed in the Atsuta Shrine in Nagoya Aichi Prefecture.
        \item Yasakani no Magatama Jewel --> Claimed to be housed in Kokyo (the Imperial Palace) in Tokyo, at the Imperial Household Agency.
    \end{itemize}
\end{itemize}
\section{\ruby{外交}{がいこう} \tabni diplomacy }
\subsection{Japanese Crests}
\begin{itemize}
    \item The kiku is to be found with a single round of pedals on the Japanese passport.
    \item A doublke layered version of the kiku-crest (Chrysanthemum flower) is used by the imperial Houselhold. 
    \item All the while the Japanese government employs a form of kiri-crest (Paulownia tree) as symbol.
\end{itemize}
\subsection {Japans 4 Diplomatic Issues}
The foreign ministry led by Kishida, Fumio currently deals with 4 main issues.
\begin{itemize}
    \item The kokko-mondai, or general diplomatic issues.
    \item The ryodo-modai, or territorial disputes.
    \item The boei-mondai, or defence issues.
    \item The boeki-mondai, or trade issues.
\end{itemize}
\subsection{Post-War Japan and it's issues:}
\begin{itemize}
    \item Sanfrancisco-Treaty*
         \begin{itemize}
               \item Japan's most important post WW2 treaty.
               \item Signed in 1951, effective since 1952. 
               \item The peace-treaty is a basis for diplomatic contact with other nations. 
               \item China was not invited to the peace-conference in San Francisco. 
               \item The Soviets (later Russians), never signed this treaty.
         \end{itemize}
    \item Nisso-Kyoudou-Sengen = JAP-SOV joint-declaration of 1956
         \begin{itemize}
               \item This joint-declaration is not exactly a peace-treaty, as such borders stay undefined.
         \end{itemize}
    \item United Nations
         \begin{itemize}
               \item Was founded in 1945.
               \item Thanks to the peace-treaty with RUS, JAP was able to join in 1956.
               \item A right they essentially bought for money.
         \end{itemize}
    \item Happo-ryoudo-mondai = Northern Territories issues
         \begin{itemize}
               \item 4 by SOV/RUS occupied small islands north of Hokkaido (Kurils)
         \end{itemize}
    \item Nisso-chuuritsu-jouyaku
         \begin{itemize}
               \item JAP-SOV non-agression-pact (1941)
         \end{itemize}
    \item Yaruta-kuidan = Yalta Conference
         \begin{itemize}
               \item Famous wartime meeting between Churchill, Roosevelt and Stalin.
               \item USA-SOV agreement: SOV would gain south Sakhalin and the Kuril islands
               \begin{itemize}
                    \item Japan recognises the Kurils beeing part of RUS but insissts that Iturup and Kunashir
                    as well as Shikotan and the Habomai islands are not part of the Kurils and thus rightfully theirs.
                    \item Japan furthermore recognizes its own loss of sovereignity over south Sakhalin but as SOV/RUS
                    never signed the SFR-treaty it considers this are now stateless and not part of RUS.
                    \item These views are also partly based on the Nichiro-washin-jouyaku, or treaty of friendship
                    between JAP and RUS(SOV). This is the treaty of Shimoda 1855.
               \end{itemize}
         \end{itemize}
     \item Taiwan:
         \begin{itemize}
               \item Taiwan had been a part of Japan from 1895 to 1945.
               \item Nikka-heiwa-jouyaku, or the Sino-Japanese peace treaty of 1952 in Taipei.
               \item しょう かいせき or Chiang Kai-shek, leader of the Kuomintang and the president of the ROC (Republic of China)
               renounced reparation claims against Japan.
         \end{itemize}
     \item China:
          \begin{itemize}
               \item Nichuu-kyoudou-seimei, or the joint-communique of communist china (PRC) and JAP in 1972.
               \item Japan thereafter still did not recognise de jure Taiwan as part of CHI but atleast de facto.
          \end{itemize}
     \item Senkaku Islands Issues
          \begin{itemize}
               \item The island desired by CHI(PRC), CHI(ROC/TAI) and JAP is currently uninhabited.
               \item It used to be inhabited by Japanese betweem 1880 and 1940.
          \end{itemize}
     \item South-Korea:
          \begin{itemize}
               \item Korea was part of Japan from 1910 to 1945.
               \item Nikkon-kihon-jouyaku, or the treaty of basic relations JAP-KOR 1965
               \begin{itemize}
                     \item This treaty was mainly possible thanks to large reparation payments made by Japan.
                     \item Korea was in dire need of that money in order to rebuild after the Korean-War.          
                     \item Ianfu-mondai (pleasure-women): The money was supposed to resolve this issue.
                     \item However the South-Korean government did not reveal this and used it instead otherwise.
               \end{itemize}
          \end{itemize}
     \item Takeshima-mondai: Liancourt-Rocks
     \item Kaku-Heiki-Mondai: Nuclear Weapons Issues
     \item Kita-chousen:
          \begin{itemize}
               \item Compensation claims from North-Korea about WW2 and others. (Japan consider them paid to all of KOR) 
               \item Rachi-mondai: Abduction problematic. (Japanese citicens are occassionally abducted by North-Korea)
               \item The Abductions happen to support North-Korean spies. (The JAP are supposed to teach the KOR)
          \end{itemize}
       \item Bouei-Mondai: Defence issues
           \begin{itemize}
               \item Jieitai: JAP Self-Defense-Forces
               \begin{itemize}
                    \item Could originally only be deployed in Japan.
                    \item Since 2015 they also can be send to foreign nations.
                    \item Japan currently mainly relies on the american military.
                    \begin{itemize}
                          \item Which leads to the Kichi-Mondai, or Issues on the (american) military bases.
                    \end{itemize}
               \end{itemize}
           \end{itemize}
        \item Boueki-Mondai: Trade issues
           \begin{itemize}
               \item Mainly with CHI and USA.
               \item Since 2011 Trade Deficit caused by jishin (earthquake) and the following tsunami.
           \end{itemize}
        \item Official Development Assistance (ODA)
        \item Oki-no-tori-shima: 'Islands' which are small rocks fortified against nature with concrete, in order to preserve
        them and continue to use them to spread sea-superiority-claims.
\end{itemize}
\section{\ruby{宗教}{しゅうきょう} \tabni religion | \ruby{社会思想}{しゃかいしそう} \tabni societal-thought }
\subsection{社会思想 \tabni societal-thought}
%\begin{enumerate}[itemsep=-6pt]\setlength\itemsep{-0.4em}
\subsubsection{The citizens primary concerns:}
\begin{itemize}%\setlength\itemsep{-0.4em}
\item 失業 (しつぎょう)--> unemployment
\begin{itemize}
\item 失業りつ --> unemploment rate
\begin{itemize}%\setlength\itemsep{-0.4em}
\item りつ = rate
\end{itemize}
\item In Japan only qualified job-seekers are counted as unemployed.
\end{itemize}
\item 自殺 じさつ suicide
\begin{itemize}%\setlength\itemsep{-0.4em}
\item Korea is 1st in suicides
\item Russian is 3rd in suicides
\item Japan is 4th in suicides \\
\end{itemize}
%\end{itemize}
\end{itemize}
%\begin{itemize}
%\subsubsection{\ruby{宗教}{しゅうきょう} \tabni religion} %IS ADDED BELOW!
%\begin{itemize}%\setlength\itemsep{-0.4em}
%\item Shinto has 51\% buddhism 42.7\%, and christians 1.5\% of the population as followers.
%\item People of Japan have the freedoms of religion and free speech. Guaranted by their constitution (けんぽう). \\
%\end{itemize}
\subsubsection{\ruby{恩}{おん} \tabni moral indebtedness || \ruby{義理}{ぎり} \tabni moral and/or social obligation}
\begin{itemize}%\setlength\itemsep{-0.4em}
\item auf die frage warum man was macht(私は恩があります。ですからそうです。) \\
\end{itemize}
%\subsubsection{}
\begin{itemize}%\setlength\itemsep{-0.4em}
\item 義理チョコ given on Feb.14(Valentines Day) to men by woman, social-obligation-chocolate. \\
\end{itemize}
\subsubsection{\ruby{人情}{にんじょう} \tabni human kindness}
\begin{itemize}%\setlength\itemsep{-0.4em}
\item ninjou ga atsui hito desu. --> a warm hearted person
\begin{itemize}%\setlength\itemsep{-0.4em}
\item japanese use thick and thin for describing a persons kindness.
\end{itemize}
\item \ruby{恥}{はじ} Gesichtslosigkeit (to loose face)\\ %TODO find better translation than gesichtlos
\end{itemize}
\subsubsection{\ruby{根回し}{ねまわし} \tabni arrangement in advance, (originally root-wrapping)}
\begin{itemize}%\setlength\itemsep{-0.4em}
\item ne-mawashi o shimasu = to prefere something the japanese people think preparing something properly allows it to be good. 
\item The japanese people do nemawashi especially before the discussion, they are so prepared that they have little to nothing left to discuss.
\end{itemize}
\begin{itemize}%\setlength\itemsep{-0.4em}
\item hon-ne literally: a real sound, meaning: a real feeling/opinion, private opinion
\item tatemae literally: a facade, meaning: public opinion \\
\end{itemize}
\subsubsection{\ruby{年功序}{ねんこうじょれつ} (ねんこうじょれつ) \tabni system of age(seniority)}
\begin{itemize}%\setlength\itemsep{-0.4em}
\item The japanese people think this system allows them to avoid fights/discussions 
\item This system comes from growing rice as the seniors where very important in the rice farming society as the experience in rice growing was very important. \\
\end{itemize}
\subsubsection{\ruby{村八分}{むらはちぶ}(むらはちぶ) \tabni societal ostrication}
\begin{itemize}%\setlength\itemsep{-0.4em}
\item originally: no cooperation except for funerals and fires (かじ)--> traditional social punishment \\
\end{itemize}
\subsubsection{\ruby{賄賂}{わいろ} \tabni bribe}
\begin{itemize}%\setlength\itemsep{-0.4em}
\item \ruby{袖の下}{そでのした} under-the-sleeves --> meaning: money-under-the-sleeves \\
\end{itemize}
\subsubsection{Gift-Giving}
\begin{itemize}
\item \ruby{心行け}{こころずけ} kokoro: heart, zuke: attachement	--> token(money) of special thanks (in advance),	dozoyoroshiku-onegai-shimasu is said when given this present
\item \ruby{(ご)祝儀}{しゅぎ} --> token(money) as thanks afterwards, presented with the words: どもありがとうございました
\begin{itemize}%\setlength\itemsep{-0.4em}
\item ごしゅうぎ are used for weddings, there a envelope is used which is hard to open
\item ごしゅうぎ are also used for birthdays, there a easy to open reusable envelope is used \\
\end{itemize}
\end{itemize}
\subsubsection{\ruby{お祝い}{おいわい} \tabni Oiwai --> Celebration}
\begin{itemize}%\setlength\itemsep{-0.4em}
\item oiwai-no-okurimono, oiwai-no-shina --> celebration gifts
\begin{itemize}%\setlength\itemsep{-0.4em}
\item \ruby{中元}{ちゅうげん} \tabni Chuugen --> Bon Festival gifts
\begin{itemize}
\item Halb-Jahres-Geschenk am 15.Juli. (The present could be beer for example.)
\end{itemize}
\item \ruby{歳暮}{せいぼ} \tabni Seibo
%\begin{itemize}
%\item  
 --> Year-end gift, the present could be soya sauce for example.
%\end{itemize}
\end{itemize}
\item \ruby{お返し}{おかえし} O-kaeshi --> return --> important to give return gifts \\
\item \ruby{年賀状}{ねんがじょう} nenga-jo
\begin{itemize}%\setlength\itemsep{-0.4em}
\item new years greetings card --> very important for japanese --> should be written by hand
\begin{itemize}
\item as preparation japanese people should learn caligraphie
\item often hundreds of cards are sent by one person \\
\end{itemize} 
\end{itemize}
\end{itemize}
\subsubsection{\ruby{名刺}{めいし} --> Business cards: 91mm x 55mm in size}
\begin{itemize}%\setlength\itemsep{-0.4em}
\item exchanging business cards is an important part of japanese business
\item the younger person should initialize giving the card. 
\item The card is given with both hands and held so that the receiver can read the card immediately. 
\item Even students often have such cards. \\
\end{itemize}
\subsubsection{\ruby{右翼}{うよく} \tabni U-Yoku --> Right-Wing-People}
\begin{itemize}
\item Still see tenno as shinto-god, and think they sacrifice themselves for emperor and country.
\item They love fuji-san and sakana. \\
\end{itemize}
\subsubsection{\ruby{右翼}{うよく} \tabni Kamikaze}
\begin{itemize}
\item Kami: god/spirit,\tabni Kaze: wind --> thus: divine wind
\item Originally used to describe the storm who threw the mongols in the 13th century back into the sea. \item Also eventually used for suicide attacks in the 2nd worldwar. \\
\end{itemize}
\subsubsection{\ruby{切腹}{せっぷく} \tabni Seppuku}
\begin{itemize}
\item Act to forgive ones sin through suicide, seppuku is the cuting of the belly (disembowlement).
\item The japanese where not allowed to talk their own opinions to their superiors if they did anyways they would afterwards commit seppuku.
\end{itemize}
\subsection{宗教 \tabni religion}
\begin{itemize}%\setlength\itemsep{-0.4em}
\item Shinto has 51\% buddhism 42.7\%, and christians 1.5\% of the population as followers.
\item People of Japan have the freedoms of religion and free speech. Guaranted by their constitution (けんぽう). \\
\end{itemize}
\begin{enumerate}
%\subsubsection{\ruby{神道}{しんとう} \tabni Shinto}
\item \ruby{神道}{しんとう} \tabni Shinto
\begin{itemize}%\setlength\itemsep{-0.4em}
\item しんとう is based on nature, there are supposed to be 8 million gods(kami)
\item tenno was believed to be great great great grandson of amatersu was first tenno
\begin{itemize}%\setlength\itemsep{-0.4em}
\item all further tennos where decendants of him
\end{itemize}
\item jinja --> shrine, to properly worship gods offerings are very important
\item torii --> shrine gate, entrance to a sacred area \\
\end{itemize}
%\subsubsection{\ruby{神道}{しんとう} \tabni Bukkyou}
\item \ruby{神道}{しんとう} \tabni Bukkyou
\begin{itemize}
\item introduced through china-->korea--> to japan in the 6th centurary
\item buddhism is the path to enlightement
\item tera, o-tera, ji-in, bukkaku are interchangeable --> use o-tera, all mean buddhist-shrine
\item sanmon --> temple gate with 3 doorways \\
\end{itemize}

%\subsubsection{The Shinto-Buddhistic relationship}
\item The Shinto-Buddhistic Relationship
\begin{itemize}%\setlength\itemsep{-0.4em}
\item \ruby{神仏習合}{しんぶつしゅうごう} \tabni Shinbutsu-Shūgō, is the unity of shintoism und buddhism.
\begin{itemize}%\setlength\itemsep{-0.4em}
\item hotoke/butsu = buddha
\item shin = shinto
\item They believed that shinto gods where reincarations of buddhistic gods or the other way around.
\item There were shinto shrines in buddhistic temples and the other way around.
\item The difference between the two religions where unimportant for the average japanese.
\item This was kept until the meiji-restoration in the 19th century.
\item Until this point the effective ruler of the country had been the \ruby{将軍}{しょうぐん}(Shogun).
\item Shogun means general/military leader.
\item The Shoguns took over in the 12th century and ruled until the meiji restoration.
\end{itemize}
\item \ruby{神仏分離}{しんぶつぶんり} \tabni Shin-Butsu-Bunri, is the separation of shintoism and buddhism.
\begin{itemize}
\item Ordered in 1868 by the government to distinguish the shinto gods from the buddhistic ones. \\
\end{itemize}
\end{itemize}
%\subsubsection{キリスト\ruby{教}{きょう} \tabni Kirisuto-Kyou --> christian religion}
\item キリスト\ruby{教}{きょう} \tabni Kirisuto-Kyou --> christian religion
\begin{itemize}
\item From 1612 to 1873 anti-christian-edicts
\item The religion was introduced to japan around the end of 16th century.
\item Most important christian church(kyoukai) in Japan is the(Oura church) in Nagasaki build in 1864.
\item 1865 hidden christians appeared surprisingly in the church, these christians were called hidden christians, the christian faith had been hidden for over 250years from the beginning of the 17th century until then.
\item The church was destroyed through an atomic bomb in 1945, but rebuild after the war. \\
\end{itemize}
\end{enumerate}

%\end{enumerate}
\section{\ruby{教育}{きょういく} \tabni education}
\subsection{\ruby{文部科学省}{もんぶかがくしょう} ministry of education}
\begin{itemize}
\item monbu = culture,	kagaku = science, sho = ministy
\item Also is concerned with matters of sport such as illegal betting on: soccer, baseball, sumo.
\end{itemize}
\subsection{Gakko-Seido}
\begin{itemize}
\item Education-level
\item Family financial Background
\end{itemize}
\subsubsection{\ruby{義務教育}{ぎむきょういく} Gimu-Kyoiku}
\begin{itemize}
\item compulsary-education for 9 years
\end{itemize}
\begin{enumerate}
\item Sho-gakku --> elementary school
\begin{itemize}
\item students are 6-years old when joining
\item no tuiton for public elementary schools, around 5000-euros for private schools
\end{itemize}
\item Chuu-gakko --> middle school
\begin{itemize}
\item students are 12-years old when joining
\item private schools are expensive but highly sought after, as they are a middle and highschool in one and thus do not require highschool entrance exams. Private schools also tend to offer better preparation for university entrance exams.
\end{itemize}
\end{enumerate}
\subsubsection{Higher Education}
\begin{enumerate}
\item \ruby{高等学校}{こうとうがっこう} --> high school
\begin{itemize}
\item short: 高校 (こうこう)
\item the average school day lasts 5-8 hours
\item students are enrolled between the ages of 15-17
\item students who are older than 17 have a problem fitting in as they will be talked to by everyone with a certain distance, as japanese students are often incapable of speaking plain style to someone their senior, thus excluding said person from the group.
\item The quality of the "senior"-high-school is very important for any future japanese career.
\end{itemize}
\item \ruby{大学}{だいがく} --> high school
\begin{itemize}
\item Students are aged 18+
\item Bachelor 4-years, master 2-years, doctor 3-years
\item There are 100.000 foreign students in Japanese universities of those are 90\% asian, 60\% even chinese.
\item They are differentiated between the \ruby{公立大学}{こうりつだいがく} general public universities, and special public universities so called \ruby{国立大学}{こくりつだいがく} national universities.
\end{itemize}
\end{enumerate}
\subsubsection{\ruby{帝国大学}{ていこくだいがく} The 7 national universities or (former) imperial universities }
\begin{itemize}
\item Also formerly know as \ruby{旧帝大}{きゅうていだい} (The 9 Imperial Universities.)
\item They were founded under the imperial government between 1886 and 1939. \\%<--Wiki %Sotos numbers: 1877 and 1936.
\begin{tabular}{ l l p{14cm} }
Hokkaido &in Sapporo & 18k(20k)students || Vice president  (and founder of predecessor Sapporo Agricultural College) was William Smith Clark, as such the university caused christian scholars to appear (students became christians)\\
Tohoku &in Sendai& 18k students || first to accept female students in 1913\\
Tokyo &in Tokyo& 30k students || oldest and most prestigious university of Japan || called 'University of Tokyo' || many japanese politicians studied here \\
Nagoya &Nagoya& 17k students\\
Kyoto &in Kyoto& 23k students || 2nd university of japan founded 1897 and build with compensation money from the sino-japanese war 1894-1895 || organized as german-styled university which meant students had to learn german\\
\end{tabular}
\begin{tabular}{ l l p{14cm} }
Osaka &in Osaka(Suita City)& 23k students || started as dutch learning school\\
Kyuushuu &in Fukuoka& originally established 1867 as medical school\\
Keijo &in Seoul& \\
Taipei &in Taiwan & \\
\end{tabular}
\item Other Universities
\begin{itemize}
\item Waseda University: 54k students, and one of the most prestigious universities in japan
\item Keio University: 34k students, originally for oranda-gaku, now also one of the most prestigious universities of japan
\item \ruby{上智大学}{じょうち} Jōchi Daigaku --> Sophia University, 12k students 
\item Würzburg University, 32k students
\end{itemize}
\item \ruby{進学率}{しんがくりつ} \tabi university-enrollment-rate \tabroku Japan:50.4\% \tab Deutschland: 35\%
\item \ruby{塾}{じゅく} \tabi --> cram school
\item \ruby{家庭教師}{かていきょうし} \tabi --> private tutor
\item \ruby{浪人}{ろうにん} \tabi originally: wandering samurai without a master to serve, now: waiting for another chance to enter a university
\item \ruby{予備校}{よびこう} prep school for students in der ronin year
\end{itemize}
\subsubsection{\ruby{教育費}{きょういくひ} \tabi educational expenses}
\begin{itemize}
\item 5.280.000 yen to educate a child in japan (117yen = 1euro) thus around 45.000 euros
\item \ruby{授業料}{じゅぎょうりょう}: tuition fees: national: 535.800yen \tabi public:537.857yen \tabi private:742.478yen
\item national--> 4579euro semester, 170euro monthly personal expenses  
\end{itemize}
\ruby {同窓生}{どうそうせい} \tabi ALUMNI
\begin{itemize}
\item important for japanese worklife, a good one is needed, they are depending on ones highschool
\end{itemize}
\section{\ruby{経済}{けいざい} \tabni economy}
\subsection{The City of おおつ}
\begin{itemize}
\item The City of Otsu is found at the southern coast of the Biwako.
\item It contains an in Japan very famous restaurant, build in medieval franconian Fachwerkhaus Style.
\begin{itemize}
\item Said restaurant is aptly named 'Würzburg'.
\end{itemize}
\item 1911 the Würzburgian Poet Max Dauthenday wrote the poem "Die Acht Gesichter vom Biwasee" about the Biwako.
\item As consequence Würzburg and Otsu became sister cities and the previously mentioned restaurant came about. 
\end{itemize}
\subsection{Japan's international corporate faces}
\begin{itemize}
\item きぎょう is the more professional term for かいしゃあ。
\end{itemize}
\subsubsection{\ruby{豊田自動車}{とよだじどうしゃ} \tabni Toyota}
\begin{itemize}
\item Toyota is the largest Business Enterprise in Japan. It was founded in 1937
\item Toyota started in 1997 as first of the World to sell Hybrid cars.
\item Toyota is called so and not Toyoda because more people read the families Kanji that way.
\item ToyotaCity near Nagoya contains the real Business Headquarter, though they also have a Headquarter in Tokyo like most Japanese companies as Tokyo is the economical center of Japan.
\item The company operates 'Nagoya' style, meaning they do never borrow money or run a deficit.
\begin{itemize}
\item No debts lead to a very stable and trustworthy company. Though also very little profit margins.
\end{itemize}
\item The company remains to this day in the hands of the Toyoda family.
\end{itemize}
\subsubsection{\ruby{任天堂}{にんてんどう} \tabni Nintendo}
\begin{itemize}
\item Unlike most corporations in the country, Nintendo has its headquarters in the city of Kyoto. Nintendo is classified as \ruby{中小企業}{ちゅうしょうきぎょう}
\item Nintendo was founded in 1889 and started out selling はなふだ, or in english flower-cards, for traditional japanese card-playing games.
\item Nintendo had its first big hit in 1980 with the 'Game \& Watch' handheld portable electronic games (Japanese: ゲーム\&ウオッチ Gēmu \& Uotchi; called Tricotronic in West Germany and Austria). 
\item These electronic games made them a fortune and kickstarted Nintendo into becoming a games giant.
\item Nintendo runs a policy of few employes, this allows for effective management, high employe salaries and very large profit margins.
\item Nintendo has however started to struggle as of late, the cause beeing the ever growing dominance of smartphones, which eradicate the need for traditional gameconsoles.
\end{itemize}
\subsubsection{ソニー  \tabni Sony}
\begin{itemize}
\item Sony is one of the largest international media conglomerates in the world.
\item It was founded at the dawn of the new age after the war, in 1946.
\item Starting as Tokyo Tsushin Kogyo(Tokyo Telecommunications Engineering Corporation) the company was renamed in 1958 to the foreign sounding Sony. (Thus カタカナ)
\item Sony too hit it large at the onset of the 1980's with the release of the Walkman in 1979.
\item In 1983 Sony followed Philips to the Compact Disc or CD and thus became one of the first and leading corporations in all things CD.
\end{itemize}
\subsection{Japan's everlasting financial super conglomerates}
\subsubsection{The original \ruby{財閥}{ざいばつ} \tabi (finacial groups) }
\begin{itemize}
\item Mitsubishi:
\begin{itemize}
\item Founded in 1870 as a shipping firm, renamed to Mitsubishi in 1873.
\item The founder Yataro Iwasaki had plans for this his company for more almost half a century by the time he was finally able to realize it. Fulfilling his dream had unto then been impossible because of the nations closed country policy. 
\item With the Meiji Restoration the nation however opened, Mitsubishi came into beeing and soon also found a close connection to the new government and its meiji-tenno as well as the military.
\item These connection proved to be ingenious as they made Mitsubishi within a few years to one of the 3 biggest corporations in Japan. 
\item The corporate logo features 3 water caltrops, or often also simply described as 3 diamonds. 
\end{itemize}
\item Mitsui:
\begin{itemize}
\item Starting of as きものや in 1876(Shop for traditional Japanese Kimono's ), the Mitsui-Family exploited a close relationship to the still reigning defacto ruler of the nation, the shogun. Soon one of the wealthiest families known they famously supported the leader of the Xinhai Revolution in China, which marked the end of the Qing Dynasty and the Emperors of China.
\item The corporate logo contains water wells.
\end{itemize}
\item Sumitomo:
\begin{itemize}
\item Founded as いずみや, a water well shop and smelting business in 1590, aswell as a bookshop from 1615's Kyoto this is by far the oldest Zaibatsu.
\item The corporation grew rich and powerful through the business of copper production which had started in 1691.
\item After the opening of the nations borders, the rapid import of western machines and techniques made the corporation rich and famous.
\item The logo shows the Igeta mark, or in other words a water well, as those were what started it all. 
\end{itemize}
\end{itemize}
\subsubsection{The dissolution of the \ruby{財閥}{ざいばつ} \tabi 1945-1951}
\begin{itemize}
\item In order to decentralise Japan's economy it was decided that all Zaibatsu were to be forcefully dissolved.
\item This happened under the occupation of the allied GHQ(General Headquarters).
\item The Supreme Commander for the Allied Powers (SCAP) and thus leader of the GHQ was General Douglas MacArthur.
\item The GHQ occupation ended with the San-Francisco-Peace-Treaty.
\item This act of dissolution though was mostly ineffective.
\end{itemize}
\subsubsection{The dissolution of the \ruby{六大グループ}{ろくだい-ぐるうぷ} }
\begin{itemize}
\item Initially however the former zaibatsu held on, 6 of them developed into the 'Group of the 6 largest', they organised business within them as to not compete but support and complete each others business. "A bank provides financial means. A maunfacturing company produces a good. A trading company ships and sells this product."
\item Soon known as Keiretsu(conglomeration of businesses linked by cross-shareholdings), they succeded and dominated Japans economy until 1991.
\end{itemize}
\subsubsection{The Bubble-Economy 1986-1991 }
\begin{itemize}
\item Because of excessive speculation the Japanese Economy finally collapsed in 1991.
\item This put an abrupt end to the big 6, or roku-dai-group.
\end{itemize}
\subsubsection{ The 3 Mega Banks \ruby{三大メガバンク}{さんだいめがばんく} }
\begin{itemize}
\item The Zaibatsu returned however soon thereafter as the 3 mega-banks:
\begin{itemize}
\item Mitsubishi UFJ Financial (Group)
\item Sumitomo Mitsui Financial (Group)
\item Mizuho Financial (Group)
\end{itemize}
\end{itemize}
\subsubsection{ The ソフトバンク group: }
\begin{itemize}
\item It's Central office is in Tokyo, and it is the largest of all Japanese internet and telecommunications corporations. 
\item It's president is amongst the 2richest persons in Japan.
\item Unexpectedly this corporation has no connection whatsoever to the former Zaibatsu, Keiretsu or Mega-Banku.
\item As such it is hailed as a symbol of the new economy. 
\end{itemize}
\subsection{ Smaller Issues (GDP, Seniors, Unemployement and Trade)}
\begin{enumerate}
\item The GDP:
\begin{itemize}
\item Japan has a highly fluctuating and unstable GDP.
\item The GDP per Capita is falling since 2011, the immediate cause beeing the 9.0 earthquake and its following tsunami that hit Japan.
\end{itemize}
\item Rodo Jikan (労働時間) or Working Hours:
\begin{itemize}
\item To combat growing unemploment, Japan adopted the 5 day working week around 1980.
\end{itemize}
\item Age of Retirement:
\begin{itemize}
\item Japanese citizens retire at the age of 60, yet their rent only kicks in at 65, as such many elderly are forced to work as cheap part time workforce. Which again is proplematic for the younger generations as they thus find it harder to get any job.
\end{itemize}
\item The \ruby{為替レート}{かわせれえと} \tabi or exchange rate:
\begin{itemize}
\item Primeminister Shinzo caused an intentionally weaker yen to boost exports and thus the economy.
\item While Japanese economists originally agreed that this plan would work, by now so many Japanese firms are no longer in Japan that the continuing effect of this decision is entirely unknown.
\end{itemize}
\item Foreign Money Reserves:
\begin{itemize}
\item The Chinese and Japanese decided large reserves of foreign currency would be to their benefit as such they hold unto the most and second most of them in the world.
\end{itemize}
\item Boeki-Aite-Koku or foreign trade partners
\begin{itemize}
\item Amongst Japans 10 most important import and export partners, only one european nation is listed, Germany.
\end{itemize}
\end{enumerate}
\section{\ruby{文化}{ぶんか} \tabni culture }
\subsection{\ruby{人口}{じんこう}}
\begin{itemize}
    \item \ruby{体格}{たいかく} physique; constitution
    \begin{itemize}
        \item \ruby{身長}{しんちょう} height(of body) /stature: avg japanese male: 171cm || avg japanese female: 158cm
        \item \ruby{体重}{たいじゅう} body weight: avg japanese male: 66kg || avg japanese female: 50kg
    \end{itemize}
    \item \ruby{平均寿命}{へいきんじゅみょう} average life expectancy
    \begin{itemize}
        \item \ruby{平均}{へいきん} is average, and \ruby{寿命}{じゅみょう} is life expectancy
        \item Japanese have the longest life expectancy in the world. With 86.8 years for women and 80.5 years for men.
    \end{itemize}
    \item \ruby{死亡率}{しぼうりつ} death rate /mortality
    \begin{itemize}
        \item Japanese go to the cities to work and life during their prime years, once old they return to the countryside to life out their live and die.
    \end{itemize}
    \item \ruby{世紀}{せいき} century,  \ruby{紀元前}{きげんぜん} BC /BCE
\end{itemize}
\subsection{Beginning of Japanese Clothes}
\begin{itemize}
    \item Clothes are seen as a \ruby{衣食住}{いしょくじゅう} necessity of life (food, clothing \& shelter).
    \item They are divided into \ruby{和服}{} (traditional Japanese clothes) and \ruby{}{} (western clothes).
    \begin{itemize}
        \item The わ in \ruby{和服}{わふく} was the original name of Japan, before it became にっぽん in the 8th century.
        \item Traditional Japanese clothes are seen as troublesome and expensivea.
        \item However their great cost also makes them to a status symbol.
        \item Japanese clothes were originally worn left-side in and right-side out.
        \item Though by the time わ became にっぽん Chinese culture had become a major influence and was considered stylish. As the Chinese thought of someone wearing clothes the traditional Japanese way as a barbarian, the Japanese soon started to switch over to the Chinese system of right-side in and left-side out.
        \item Around the year 790 there was even an official edict commanding the citizens to wear their clothes in Chinese tradition.
    \end{itemize}
\end{itemize}
\subsection{History of Japanese Clothes}
\subsubsection*{Men's Clothes}
\begin{itemize}
    \item From the 8th to the 12th century young aristocrats started wearing \ruby{狩衣 }{かりぎぬ}. 
    \item In the 16th century Samurai began wearing \ruby{垂領}{ひたたれ}. When wearing it they would put there upper clothes into there lower-clothing.
    \begin{itemize}
        \item In order to represent their different and especially higher rank, aristocrats kept wearing the かりぎぬ with traditional upper clothes out.
    \end{itemize}
    \item The \ruby{裃}{かみしも} then became in the 17th Century the main samurai clothing.
    \item 1864 adopted the then ruling shogun western clothes as standard clothing equipment for his soldiers as they were more practical then their traditional Japanese counterparts.
    \item 1871 ruled the tenno-meiji that all government officials and parts of the military had to wear western clothing. As traditional Japanese clothes had become an image as the clothes of barbarians.
    \item 1871-1873 was the Iwakura Mission
    \begin{itemize}
        \item Though the leader of the mission initially wore the traditional Japanese clothes, he too soon accepted that	
        they were seen as barbaric and switched to western clothing.
        \item The Iwakura mission passed by Würzburg on the 5th May of 1873, there they were guided by Alexander Siebold,	
        son of the famous german physician, botanist, and traveler Philipp Franz Balthasar von Siebold the who achieved 	prominence by his studies of Japanese flora and fauna and the introduction of Western medicine in Japan.
    \end{itemize}
\end{itemize}
\subsubsection*{Women's Clothes}
\begin{itemize}
\item 8-12C
\begin{itemize}
\item \ruby{十二単}{じゅうにひとえ}, 単 means -> one piece, thus this is a twelve layered one piece ceremonial kimono.
\item \ruby{小袖}{こそで} are clothes with small sleeves (thus the name ko=small sode=sleeve) they were originally underwear.
\end{itemize}
\item 13-16C: The 小袖 became initially casual wear and finally towards the end of the 16th century even formal wear. Thereafter it still grew in popularity and soon was to be found generally everywhere.
\item 17-19C \ruby{振り袖}{ふりそで} or long-sleeved (swinging sleeves)kimono was developed out of the 小袖
\item 19-20C The \ruby{袴}{はかま} was adopted as school uniform for women.
\item First half of the 20th C, セーラー服(serafuku) or sailor-uniform became popular school uniform for women.
\item 1923 The great Kanto earthquake --> A turning point for clothing as many women died in this earthquake because they could not run away wearing the traditional clothing. As consequence more and more women adopted western clothes.
\end{itemize}
\subsection*{\ruby{冠婚葬祭}{かんこんそうさい}}
\begin{itemize}
\item かん --> coming of age at 20 years old
\item こん --> marriage
\item そう --> funeral
\item さい --> rites for the ancestors
\end{itemize}
\subsubsection{七五三 \tabi 7-5-3 }
\begin{itemize}
    \item しちごさん is a rite of passage celebrated on the 15th November each year for boys 3 or 5 and girls 3 or 7.
    \item The children are brought to a shinto shrine to celebrate and pray for their growth and well-being.
    \item Originally children up until the age of 3 had their heads shaven. Though this is no longer a tradition.
    \item At 5 years old boys wear a \ruby{袴}{はかま} for the first time.
    \item At 7 years old girls wear a obi bound kimono for the first time (Until then their kimono is held together by a simple strap.) One waits until the age of 7 as the obi-kimono is to complicated and heavy to wear for a younger child.
\end{itemize}
\subsubsection{\ruby{成人式}{せいじんしき} \tabi coming of age}
\begin{itemize}
    \item 成人 --> adult, 式 --> person, thus meaning coming of age
    \item It is celebrated at the 2nd Monday of January and marks the now 20year old's to be officially allowed to drink and smoke, as well as their welcome into the world of adults.
    \item There are many voices against this tradition with the ever smaller number of young Japanese and their increasing lack of 'good manners'.
    \item Though the government supports the tradition as the kimono manufacturers and shops lobby them for its existence. They of course earn a significant amount of money from the tradition as \ruby{和服}{わふく} remains very expensive.
\end{itemize}
\subsubsection{\ruby{結婚式}{けっこんしき} \tabi marriage}
\begin{itemize}
    \item Japanese weddings were originally held at home, though these days they usually take place at shrines or churches.
    \item \ruby{披露宴}{ひろうえん}, or wedding reception are usually held following a wedding. Often in hotels or as of recently in newly emerging reception halls at shrines.
    \item At 披露宴 couples receive wedding gifts お\ruby{祝}{いわ}い or \ruby{祝儀}{しゅうぎ}.
    \item One customary gives 30k 円 (or any other uneven amount of 10k円notes), important hereby is that the given amount is 'hard' to divide amongst the 2 new spouses.
\end{itemize}
\subsection{More clothes and clothing related items.}
\begin{enumerate}
    \item Montsuki-Haori-Hakama, a ceremonial kimono for men with hakama skirt and bearing the family sigil (montsuki).
    \item Tabi, Japanese socks which group the large toe together with its neighbouring toe, and the 3 remaining toes as a group seperately. This is to accomodate traditional japanese sandals which have there holding straps in the middle of the front.
    \item Sandals, Setta and Zori(made of ricestraw or other plant fiber) and Geta(made of wood) are the 3 types of traditional Japanese sandals.
    \item An obi is considered less formal then a hakama, Obi means belt, they used to be 10cm wide up until the end of the 17th century but thereafter considerably wider obis became very popular.
    \item Furi-Sode the already mentioned furi-sode has sleeves with a length of ca 100cm and is worn by unmarried women.
    \item Tome-Sode, worn by married women this shoulder short sleeved kimono is especially expensive and very formal.
    \item Nihon-Gami, is the traditional Japanese hairdo.
    \item Yukata, originally a bathrobe it is now a summer kimono often worn at summer festivals. Cheap yukata also serve as sleepwear.
    \item Tenugui, the Japanese towel used to be made from silk which in the past had been cheaper then cotton, the towel often doubles as personal ornament.
    \item \ruby{団扇}{うちわ} \tabi, a fan to create a breeze, highly decorated variations often served as status symbol.
    \item \ruby{扇子}{せんす} or folding fan is a japanese invention, it is often used to draw a borderline for bowing, it furthermore is always to be held in the right hand, anything else is considered barbaric. It is also used as accessoire while dancing.
    \item \ruby{懐紙}{かいし}, the pocket-paper, or better chest-paper as kimono's do not have a pocket is a handy utensil, not just part of tea ceremonies it is also more generally often used to wish up something, as coaster or to take notes.
\end{enumerate}
\end{document}
