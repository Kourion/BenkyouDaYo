\documentclass{article}
\usepackage{xeCJK}
\usepackage{enumitem}
\usepackage{titlesec}
\usepackage[dvipsnames]{xcolor}
\usepackage{geometry}
\usepackage[CJK,overlap]{ruby}
 \geometry{
 a4paper,
 %total={170mm,257mm},
 left=10mm,
 top=12mm,
 right=10mm,
 bottom=12mm,
 }
 
\renewcommand{\rubysep}{0.1ex}%3.9ex}
\makeatletter
\renewcommand\paragraph{\@startsection{paragraph}{4}{\z@}%
            {-0.5ex\@plus -.1ex \@minus -.25ex}%
            {-1em}%
            {\normalfont\normalsize\bfseries}}
\renewcommand\subparagraph{\@startsection{subparagraph}{5}{\z@}%
            {-0.1ex\@plus -.1ex \@minus -.25ex}%
            {-1em}%
            {\normalfont\normalsize\bfseries}}
\makeatother

\newcommand\tab[1][1cm]{\hspace*{#1}} 
\newcommand\tabi[1][0.05cm]{\hspace*{#1}}
\newcommand\tabni[1][0.2cm]{\hspace*{#1}}
\newcommand\tabyon[1][0.4cm]{\hspace*{#1}}
\newcommand\tabroku[1][0.6cm]{\hspace*{#1}}

\XeTeXlinebreaklocale "ja"
\XeTeXlinebreakskip=0em plus 0.1em minus 0.01em

\setlist{nolistsep}
\titlespacing*{\section}
{0pt}{1.5ex plus 1ex minus .2ex}{1.5ex plus .2ex}
\titlespacing*{\subsection}
{0pt}{0.5ex plus 1ex minus .2ex}{0.5ex plus .2ex}

\setlength{\parindent}{0pt}
\setCJKmainfont{ipaexm.ttf}
\setcounter{tocdepth}{5}

\begin{document}	%\large {1.5ex \@plus .2ex}%

\tableofcontents
\newpage

\section{Verb Conjugation}
\subsection{The 3 Verb Groups}
\begin{itemize}
\item Group 3: \ruby{来}{く}る and する make up the group 3 verbs.
\item Group 2: If there is a え or い ending syllable before the
 る-ending of the verb then it is a group 2 verb.
\item Exceptions: There are some exceptions to group 2: \ruby{帰}{かえ}る,\ruby{切}{き}る,
\ruby{知}{し}る,\ruby{入}{はい}る,\ruby{走}{はし}る these are group 1 verbs.
%More such excpetions for future reference: しる (to know), はいる (to enter), いる (to need), すべる (to slide), へる (to decrease), ねる (to knead), ける (to kick), かじる (to bite), まじる (to be mixed), しゃべる (to talk), せびる (to extort), あせる (to be in a hurry), はしる (to run), ちる (to scatter), てる (to shine)
\item Group 1: Everything else. Including る ending verbs which are not in group 2.
\end{itemize}
\subsection{The Polite-Style, ます-Form}
\begin{itemize}
\item Group 1: Change the う-ending to a い-ending and add ます.
\item Group 2: Remove the る and add ます instead.
\item Group 3: する --> します, \tabyon \ruby{来}{く}る --> \ruby{来}{き}ます
\end{itemize}
\subsubsection{The Forms ません, ました, ませんでした, ましょう}
\begin{itemize} \item Remove the ます and add ません for the negative-form, ました for the past-form,\\
ませんでした for the past-negative-form and finally ましょう for the 'lets-do'-form.
\end{itemize}
\subsection{The Plain-Style, ない-Form }
\begin{itemize}
\item Group 1: Change the う-ending to an あ-ending, if the verb ends in う itself, 
change it to わ instead.
Add ない. \\ \tab ある is an exception, it becomes simply ない itself.
\item Group 2: Remove the る and add ない. 
\item Group 3: する --> しない, \tabyon \ruby{来}{く}る --> \ruby{来}{こ}ない
\end{itemize}
\subsubsection{The negative Plain-Style, なかった-Form}
\begin{itemize}
\item In order to negate any verb in plain style simply remove the ない and add なかった instead.
\end{itemize}
\subsection{The て-Form}
\begin{itemize} \item Group 1:
\begin{itemize} \item Ending with: る, つ, う: Remove the ending and replace it with って.
\item Ending with: む, ぶ, ぬ Remove the ending and replace it with んで.
\item Ending with: く, ぐ: The く becomes いて and the ぐ becomes いで. Exception: いく becomes いって.
\item Ending with: す: Remove the ending and replace it with 「して.」
\end{itemize}
\item Group 2: Replace the ending る with て.
\item Group 3: する --> して, \tabyon \ruby{来}{く}る --> きて
\end{itemize}
\subsubsection{The た-Form}
\begin{itemize}
\item For the た-Form simply replace the て with た and で with だ.
\end{itemize}
\subsection{The Passive-Form }
\begin{itemize} 
\item Group 1: Change the last character as you would for negative verbs (う-ending to an あ-ending, if the verb ends in う itself, change it to わ instead.) and attach 「れる」
\item Group 2: Replace the last 「る」 with 「られる」
\item Group 3: 「する」 becomes 「される」 and 「くる」 becomes 「こられる」
\item Note: All passive verbs become る-verbs, as they then must end with れる.
\end{itemize}
\subsection{The Casual Volitional-Form ¥}
\begin{itemize}
\item This form expresses ones desire to do something akin to 'Let's...',
 it is the plain-style of ましょう.
\item Group 1: Change the ending from its う-form to the お-form and add う.
This then forms essentially a long o.
\item Group 2: Take out the ending る and replace it with よう.
\item Group 3: する --> しよう, \tabyon \ruby{来}{く}る --> \ruby{来}{こ}よう.
\end{itemize}
\subsection{The Potential-Form ¥}
\begin{itemize}
\item This form expresses ones ability to know how to, to be able to, do something.
\item Group 1: Change the ending from its う-form to the え-form and add る.
\item Group 2: Take out the ending る and replace it with られる.
\item Group 3: する --> できる, \tabyon \ruby{来}{く}る --> \ruby{来}{こ}られる.
\item Warning: All Verbs in potential-form are group 2 verbs when conjugating further.
\end{itemize}
\subsection{The Causative-Form }	
\begin{itemize}
\item This form expresses one to let (with あげる,くれる,もらう) or make someone else do something, though it usually is interpreted as forcing somebody else to do something.
\item Group 1: Change the ending from its う-form to the あ-form (わ for う) and add せる.
\item Group 2: Take out the ending る and replace it with させる.
\item Group 3: する --> させる, \tabyon \ruby{来}{く}る --> \ruby{来}{こ}させる.
\item Warning: All Verbs in causative-form are group 2 verbs when conjugating further.
\end{itemize}
%\section{Vocabulary}
%\begin{itemize}
%\item おぼえています --> to remember
%\item おぼえる --> to memorize
%\item わすれました --> to have forgotten
%\item わすれる --> to forget
%\item よとう --> ruling party
%\item やとう --> opposition party
%\end{itemize}
\section{To want something...}
\subsection{Desiring things: ほしい}
\begin{itemize}
\item ほしい is used when the thing you desire is an object and not an action.
\item First Person:
\begin{itemize}
\item Because ほしい is an adjective and not a verb, the object of the sentence is marked by が and not を.
\item Example: 私はお金がほしいです。 --> I want money.
\end{itemize}
\item Third Person:
\begin{itemize}
\item: When expressing other’s desires using ほしがっています you should use the particle を instead of が.
\item Example: 彼女はワインをほしがっています。	 --> She seems to want wine.	(Wordstem: ほしがる)
\item An equivalent would be: 彼女はワインがほしいと言っています。 --> She said she wants wine.
\end{itemize}
\end{itemize}
\subsection{Desiring actions: たい}
\begin{itemize}
\item たい works again like an adjective thus が is used and the negative can be formed with (た)くない。
\item Positive first person: 私はパンが食べたいです。 --> I want to eat bread.
\item Negative first person: 私はパンが食べたくないです。 --> I do not want to eat bread. \\
\item Again third person is formed with がって thus we are back to using を.
\item Third Person: かれはコーヒーを飲みたがっています。 --> He wants to drink coffee.
\item Equivalent: コーヒーがのみたいといっています。
\end{itemize}
\subsection{Comparing たい and  ほしい}
\begin{itemize}
\item あなたはなにがほしいですか。 --> What do you want?
\item 彼女はなにをほしいがっていますか。 --> What does she want?
\item あなた\ruby{何}{なに}がしたいですか。 --> What do you want to do?
\item 彼は\ruby{何}{なに}をしたがっています。 --> What does she want to do?
\end{itemize}
\section{Expressions of Desire or Hope}
\subsection{\ruby{怖}{こわ}い --> こわい --> scary/awful }
\begin{itemize}
\item 彼女はこわいです。 --> She is awful.
\item 彼女は\ruby{犬}{いぬ}をこわがっています。 --> It seems she is afraid of dogs.
\item 彼女は\ruby{犬}{いぬ}をこわいと言ってます。 --> She said she is afraid of dogs.
\item All statements are to be taken as facts, the uncertainty the middle one contains comes from politeness.
\end{itemize}
\subsection{\ruby{嬉}{うれ}しい / \ruby{喜}{よろこ}んで --> happy}
うれしい and よろこんで both mean happy. But うれしい is used in first person and よろこんで in third person.
よろこんで comes from よろこぶ.
\begin{itemize}
\item 私はうれしいです。 --> I am happy.
\item 彼はよろこんでいます。 --> He is happy.
\end{itemize}
\subsection{\ruby{悲}{かな}しい --> かなしい --> sad }
\begin{itemize}
\item The first person uses \ruby{悲}{かな}しい as い-adjective the third person however derives \ruby{悲}{かな}しんで from \ruby{悲}{かな}しむ.
\item 私はかなしいです。 --> I am sad.
\item 彼女は\ruby{悲}{かな}しんでいます。 --> She is sad.
\end{itemize}
\subsection{\ruby{寂}{さび}しい --> さびしい --> lonely }
\begin{itemize}
\item The first person uses \ruby{寂}{さび}しい as い-adjective the third person however again uses it in connection with がって forming さびしがっています.
\item 私は\ruby{寂}{さび}しいです。 --> I am lonely.
\item 彼はさびしがっています。 --> He is lonely.
\end{itemize}
\subsection{\ruby{恥}{は}ずかしい --> はずかしい --> ashamed / shy }
\begin{itemize}
\item The first person uses \ruby{恥}{は}ずかしい as い-adjective the third person however again uses it in connection with がって forming \ruby{恥}{は}ずかしがっています.
\item 私は\ruby{恥}{は}ずかしいです。 --> I am ashamed.
\item 彼女は\ruby{恥}{は}ずかしがっています。 --> She is ashamed.
\end{itemize}
\subsection{\ruby{羨}{うらや}ましい --> うらやましい --> envy / jealous }
\begin{itemize}
\item The first person uses \ruby{羨}{うらや}ましい as い-adjective the third person however again uses it in connection with がって forming \ruby{羨}{うらや}ましがっています.
\item 私は、彼女が\ruby{羨}{うらや}ましいです。 --> I am envious of her. / I envy her.
\item 彼は、彼女を\ruby{羨}{うらや}ましがっています。 --> He is envious of her. / He envies her.
\end{itemize}
\section{The difference between となり and よこ ¥} 
\begin{itemize}
\item Things that are \ruby{隣}{となり} \tabi must be of the same type, caliber, or quality. They must be perceived in some regard as equal. \ruby{隣}{となり} \tabi is also used to refer to the nearest object of the same type, regardless of whether the actual distance is near or far.
\item Things that are \ruby{横}{よこ} \tabi are right next to each other, size and quality do not matter.
\end{itemize}
\section{Expressions of Like or Dislike}
\begin{itemize}
\item \ruby{好}{す}き --> to like:
\begin{itemize}
\item すき is a な-adj, thus the negative is formed as: 好きではありません.
\item 私はビールが好きです。
\end{itemize}
\item \ruby{嫌}{きら}い --> to hate
\begin{itemize}
\item きらい is also a な-adj, thus again the negative is formed as: 嫌いではありませn.
\end{itemize}
\item Both すき and きらい are not to be modified with things like たいへん for emphasis, but instead だい. Also 大好き for example.
\item Example sentence structures:\\
%\begin{itemize}
%\item Question:				\tab あなたはコーラがすきですか。 --> Do you like cola?
%\item Positive answer:		\tab はい、_(私)_は_(コーラ)_が_(好き)_。
%\item Negative answer:		\tab いいえ、(私)  は_(コーラ)_は_(嫌い)_。
%\item Question: 			\tab あなたはなにがすきですか。 --> What do you like?
%\item Answer: 				\tab 私はビールがすきです。
%\end{itemize}
\begin{tabular}{ l l l l l l l l l l }
Question:&&あなた&は&コーラ&\color{orange}が&すき&です&か。& --> Do you like cola? \\
Positive answer:&はい、&私&は&コーラ&\color{orange}が&好き&です&。& --> Yes i like cola. \\
Negative answer:&いいえ、&私&は&コーラ&\color{blue}は&嫌い&です&。& --> No i hate cola. \\ \\
Question:&&あなた&は&なに&\color{orange}が&すき&です&か。& --> What do you like? \\
Answer:&&私&は&ビール&\color{orange}が&すき&です&。& --> I like beer. \\
\end{tabular}
\end{itemize}
\section{Normalising a verb through the use of こと}
\begin{itemize}
\item Changing the verb to dic-style and adding こと results in the verb becoming a noun.
\begin{itemize}
\item 本を読みます。 それは楽しいです。 --> thus becomes --> 本を読むことは楽しいです。
\end{itemize}
\item Other Examples are: 
\begin{itemize}
\item あなたがすきなことはなんですか。 	--> What do you like?
\item 私が好きなことは、音楽をきくことです --> I like to listen to music.
\end{itemize}
\end{itemize}
\section{Describing Ability}
\tab \begin{tabular}{ l l l l l l l l l l }
私&\color{red}は&日本語&&&&\color{orange}が&\ruby{出来}{でき}ます&。& --> I am capable of Japanese.\\
彼&\color{red}は&日本語&\color{blue}を&話す&\color{ForestGreen}事&\color{orange}が&\ruby{出来}{でき}る&。& --> He can speak Japanese.\\
彼女&\color{red}は&日本語&\color{blue}を&読む&\color{ForestGreen}事&\color{orange}が&\ruby{上手}{じょうず}です&。& --> She is good at reading Japanese.\\
彼達&\color{red}は&日本語&\color{blue}を&書く&\color{ForestGreen}事&\color{orange}が&\ruby{下手}{へた}です&。& --> They are bad at writing Japanese.\\
彼&\color{red}は&&&食べる&\color{ForestGreen}事&\color{orange}が&\ruby{得意}{とくい}です&。& --> He is good at eating and likes it.\\
彼女&\color{red}は&&&走る&\color{ForestGreen}事&\color{orange}が&\ruby{苦手}{にがて}です&。& --> She is bad at running and dislikes it.\\
彼&\color{red}は&ピアノ&\color{blue}を&ひく&\color{ForestGreen}事&\color{orange}が&\ruby{得意}{とくい}です&。& --> He likes to play the piano and is good at it.\\
彼女&\color{red}は&フルート&\color{blue}を&ふく&\color{ForestGreen}事&\color{orange}が&\ruby{苦手}{にがて}です&。& --> She likes to play the flute and is good at it.\\
\end{tabular}
\subsection{Other sentence structures: \\ \ruby{必要}{ひつよう} (to need), \ruby{分}{わ}かる(to understand/know), \ruby{苦手}{にがて}(to dislike) }
\vspace{1mm}
\tab \begin{tabular}{ l l l l l l l l l l }
私&は&&&ミルク&が&苦手&です。& --> I dislike milk.\\
あなた&は&&&日本語&が&わかります&か。& --> Do you understand Japanese?\\
私&は&&&時間&が&必要&です。& --> I need time.\\
ダンス&が&じょうず&な&ひと&は&だれ&ですか。& --> The person who is a good dancer is who? %The person who plays the piano is who?\\
\end{tabular}
\section{Expressions of Possession}
\vspace{1mm}
\tab \begin{tabular}{ l l l l l l l l }
N1&には&N2&が&&あります。& --> N1 has N2. & \\
N1&には&N2&が&おおい&です。& --> N1 has many/much N2. & (This effectively means たくさん) \\
N1&には&N2&が&すこない&です。& --> N1 has few/little N2. & (This effectively means すこし) \\
\end{tabular}
\vspace{2mm} \\ \\
%\tab \begin{tabular}{ l l l l l l l l l l }
The い-adjective can also be modified to become a noun as usual:
\begin{itemize}
\item おおくのがくせいがいます。 --> There are many students. (There is no opposite for few persons.) \\
\end{itemize}
Furthermore this structure allows to use あります on inanimate objects: \\ 
(Note the consent amongst the Japanese is that this is complete nonsense, so maybe don't use it in real life.)
\begin{itemize}
\item N1にはN2があります。 /Using あります on inanimate objects however is limited to friends and family.
\item 彼女には、車の\ruby{運転}{うんてん}が\ruby{得意}{とくい}な友達が、\ruby{全部}{ぜんぶ}で\ruby{十人}{じゅうにん}あります。 --> She has 10 friends who can drive very well.
\end{itemize}
\vspace{3mm}
Possession of inanimate objects however can be expressed through the verb もつ:
\begin{itemize}
\item NsはNoをもっています。 --> To have/own something.
\end{itemize}
\vspace{3mm}
\subsection*{How many siblings does one have?}
\begin{itemize}
\item あなたは\ruby{何人}{なんにん}\ruby{兄弟}{きょうだい}ですか。 --> How many siblings are you(altogether). --> Effectively asking how many brothers and sisters one has.
\item 私はふたりきょうだいです. In regards to me there are 2 siblings. (your sibling + you) --> Meaning that one has \color{red}one \color{black}sibling.
\item 私は\ruby{一人っ子}{ひとりっこ}です。 --> I am a lone child.
\item 私はふたごです。 --> I am a twin.
\end{itemize}
\section{Expressions of Experience}
\begin{itemize}
    \item N1 は...V-た  こと が あります. = N1 has the experience of V-ing.
    \item N1 は...V-Dic こと が あります. = It sometimes happens that N1 does V
                                     = There are cases in which N1 does V
\end{itemize}
\subsection{Counting Experience}
\begin{itemize}
\item Asking how often did one experience something: \\
    \begin{tabular}{ l l l l l l l l }
    あなたは&なんかい&それを&した&ことがありますか。\\
            &なんど  &      &    &                  \\
    \end{tabular}
\item Modifying counting words: \\    
    \begin{tabular}{ l l l l l l l }
    how often   & once      & twice     & thrice    & several times & many times \\
    なんかい    & いっかい  & にかい    & さんかい  & なんかいか    & なんかいも \\
    なんど      & いちど    & にど      & さんど    & なんどか      & なんども 
    \end{tabular}
\item Further question words which can be modified:

\tabni \begin{tabular}{ l l l }
\ruby{何時}{いつ} = when, & --> \ruby{何時}{いつ}か = someday, & --> \ruby{何時}{いつ}も = always/usually/every time |never (with neg. verb) \\

\ruby{何処}{どこ} = where, & --> \ruby{何処}{どこ}か = somewhere, & --> \ruby{何処}{どこ}も = everywhere/wherever |nowhere (with neg. verb) \\

\ruby{何}{なに} = what, & --> \ruby{何}{なに}か = something, & --> \ruby{何}{なに}も = all |nothing (with neg. verb) \\

\ruby{誰}{だれ} = who, & --> \ruby{誰}{だれ}か = someone, & --> \ruby{誰}{だれ}も = everyone /anyone |no-one (with neg. verb)
\end{tabular} \\

も can here also always be replaced with にも for further modification of the question words. \\
\end{itemize}
\section{The Whole and it's Parts}
\begin{itemize}
\item An elegant structure to define parts of a whole is via usage of the は-が-structure. Where the whole is initially named through the particle は and then the part is indicated by standing before the が. After the が an adjective or a verb can be used to describe the part or what it is doing. The conjugation of a verb can express the time as usual here, with た-form being past, て-form present and Dic-form future in an informal setting. \\ (ました /て-form /ます for a formal setting)\\
\item \begin{tabular}{ l l l l l l l }
きょう&は&てんき&が&いいです。 & The weather today is good.\\
かれ&は&あたま&が&わるいです。 & He is unintelligent.\\
かのじょ&は&かみ&が&ながいです。 & Her hair is long.\\
かれ&は&め&が&わるいです。 & His eyesight is bad.\\
今日&は&あめ&が&ふっています。 & It rains.\\
あした&は&ゆき&が&ふります。 & It will snow tomorrow.
\end{tabular}
\end{itemize}
\section{Having Interest in Something}
\begin{itemize}
\item The object interested in is marked with に. Both かんしん and きょうみ can be used to express this interest then.
\item \begin{tabular}{ l l l l l l l }
私は&テニス&に&かんしん&があります。 &I am interested in tennis. \\
   &Noun&&きょうみ&&
\end{tabular}
\end{itemize}
\section{Passive construction, causes and death.}
\begin{itemize}
\item See Verb conjugation for the passive form. For conjugation rules.
\item This changes verb meanings into "to be ..". \\
    Examples: to kill --> to be killed, to say --> to be said/ to be told \\
              to make --> to be made, to write --> to be written \\
\end{itemize}
\begin{tabular}{ l | l | l | l | l | l }
Ns & は & N & によって & かかれました。 &--> Ns was written by N. \\
&&&に&つくりました。&--> Ns was created by N. \\
&&&から&& 
\end{tabular} \\
\vspace{2 mm} \\
Example sentences: \\
\begin{tabular}{ l l l l l l | l }
active:&Aは&B&を&ころしました。&--> A killed B. \\ %\hline
passive:&Bは&A&によって&ころされました。&--> B was killed by A. \\ %\hline
active:&Bは&じつ&(を)&しました。&--> B killed himself. /B committed suicide. \\ %\hline
cause:&Bは&びょうき&で&しにました。&--> B died of sickness. \\ %\hline
cause:&Bは&とし&で&しにました。&--> B died of age. \\ %\hline
cause:&Bは&せんそう&で&しにました。&--> B died because of the war. \\ %\hline
cause:&Bは&じこ&で&しにました。&--> B died through an accident. 
\end{tabular}    
\section{The causative construction}
\begin{itemize}
\item The causative can either be used to force somebody to do something, or in connection with あげる,くれる,もらう to express that somebody does let somebody else do something, or asks for said permission. \\
\begin{tabular}{l l l l l|l}
Noun1&は&Noun2&を&V(intra)-causative& (Only intransitive Verbs can be used with を) \\
私&は&弟&を&買い物に行かせます。 &-> I order my little brother to go shopping. \\
\end{tabular} \\
\begin{tabular}{l l l l l l l|l}
\hline \\
Noun1&は&Noun2&に&NounObject&を&V-causastive.& (All Verbs can be used with に)\\
私&は&妹&に&歌&を&歌わせます。 &-> I order my little sister to sing a song.
\end{tabular}
\item The passive and causative forms can be combined. They then usually express that the subject is forced to do something against his will.
\begin{itemize}
\item Example:
\item 弟は父に東京へ行かせられました。 = My brother was made to go to Tokyo by my father.
\end{itemize}
\end{itemize}
\section{Verbs of Giving and Receiving}
\begin{itemize}
\item Giving and receiving verbs are depending on who gives to whom. あげる expresses giving away from 家 towards 外 (outwards movement)or within the same level. くれる is the opposite and describes giving from 外 towards 家 (inwards movement). もらう means 'is given' or received and describes always a inward movement or one on the same level.\\
\vspace{2 mm} \\
\begin{tabular}{ l l l l l l l | l }
A&が&B&に&N&を&あげます。&-> A gives N to B. \\
G&が&R&に&N&を&くれます。&-> G gives N to R. \\
\end{tabular} \\
\vspace{2 mm} \\
\begin{tabular}{ l l l l l l l | l }
X&が&Y&に&N&を&もらいます。&-> X is given N by Y.\\
&&&から&&&&-> X receives N from Y. \\ 
\end{tabular}
\end{itemize}
\section{The Giving and Receiving of Acts}
\begin{itemize}
\item Doing something for somebody else or having something done for oneself is expressed through the giving and receiving of acts in Japanese. \\ \vspace{2 mm} \\
\begin{tabular}{ l l l l l l l | l }
X&が&Y&に&Noun&を&Vte-あげます。&-> X does NをVte for Y. \\ 
私&は&タロウ&に&英語&を&教えてあげました&-> I taught Taro English. \\
\\
%\end{tabular} \\ \vspace{2 mm} \\
%\begin{tabular}{ l l l l l l l | l }
Y&が&X&に&Noun&を&Vte-くれます。&-> Y does NをVte for X. \\ 
タロウ&が&私&に&日本語&を&教えてくれました。&-> Taro taught me Japanese. \\
\\ 
%\end{tabular} \\ \vspace{2 mm} \\
%\begin{tabular}{ l l l l l l l | l}
X&が&Y&に&Noun&を&Vte-もらいます。&-> X has Y do NをVte (for X) \\ 
私&は&タロウ&に&日本語&を&教えてもらいました。&-> I asked Taro to teach me Japanese.
\end{tabular}
\end{itemize}
\end{document}