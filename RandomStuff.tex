\documentclass{article}
\usepackage{xeCJK}
\usepackage{enumitem}
\usepackage{geometry}
\usepackage{titlesec}
\usepackage{lipsum}
\usepackage[CJK,overlap]{ruby}
 \geometry{
 a4paper,
 %total={170mm,257mm},
 left=10mm,
 top=12mm,
 right=10mm,
 bottom=12mm,
 }
 
\renewcommand{\rubysep}{0.1ex}%3.9ex}
\makeatletter
\renewcommand\paragraph{\@startsection{paragraph}{4}{\z@}%
            {-0.5ex\@plus -.1ex \@minus -.25ex}%
            {-1em}%
            {\normalfont\normalsize\bfseries}}
\renewcommand\subparagraph{\@startsection{subparagraph}{5}{\z@}%
            {-0.1ex\@plus -.1ex \@minus -.25ex}%
            {-1em}%
            {\normalfont\normalsize\bfseries}}
\makeatother

\setlist{nolistsep}
\titlespacing*{\section}
{0pt}{1.5ex plus 1ex minus .2ex}{1.0ex plus .2ex}
\titlespacing*{\subsection}
{0pt}{0.5ex plus 1ex minus .2ex}{0.5ex plus .2ex}

\newcommand\tab[1][1cm]{\hspace*{#1}} 
\newcommand\tabni[1][0.2cm]{\hspace*{#1}}
\newcommand\tabyon[1][0.4cm]{\hspace*{#1}}
\newcommand\tabroku[1][0.6cm]{\hspace*{#1}}

\XeTeXlinebreaklocale "ja"
\XeTeXlinebreakskip=0em plus 0.1em minus 0.01em

\setlength{\parindent}{0pt}
\setCJKmainfont{ipaexm.ttf}
\setcounter{tocdepth}{5}

\begin{document}	%\large {1.5ex \@plus .2ex}%
\section{How to count Millions}
\begin{itemize}%\setlength\itemsep{-0.4em}	%\vspace{-\topsep} 
\item Population of Japan: 127million: ichi-oku-ni-sen-nana-hyaku-man nin
\item Population of Russia: 143million: ichi-oku-yon-sen-san-hyaku nin
\item Population of China: 1376million: juu-san-oku-nana-sen-roku-hyaku-man nin
\item 100 million: ichi-oku-man
\item 10 million: sen-man
\item 1 million: hyaku-man
\end{itemize}
\section{How to say something is more of something (here: bigger)}
\begin{itemize}
\item どちらのほうがおおいですか。
\item フランスのほうがおおいです。
\end{itemize}
\section{Ausflug}
12.10.2016 (10.DEC) um 10:00 treff am Bahnhof dann 10:37 zug, entweder um 15:30 oder 17:30 zurück.
\section{Random Vocabulary}
\begin{itemize}
\item ほとんど倒れしている。 \ruby{倒}{たお}れる		
\begin{itemize}
\item this is akin to benkyou shimasu, benkyou-shite-iru
\begin{itemize}
\item the を is apparently not needed for this form (though this is informal)
\end{itemize}
\end{itemize}
\item \ruby{発音}{はつおん} = pronunciation
\item \ruby{惜}{お}しい = regretable
\item \ruby{予定}{よてい} = plan / schedule
\end{itemize}

\end{document}