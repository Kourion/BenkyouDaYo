\documentclass{article}
\usepackage{xeCJK}
\usepackage{enumitem}
\usepackage{titlesec}
\usepackage[dvipsnames]{xcolor}
\usepackage{fancyhdr}
\usepackage{geometry}
\usepackage[CJK,overlap]{ruby}
 \geometry{
 a4paper,
 %total={170mm,257mm},
 left=5mm,
 top=8mm,
 right=5mm,
 bottom=12mm,
 }
 
\renewcommand{\rubysep}{0.1ex}%3.9ex}
\makeatletter
\renewcommand\paragraph{\@startsection{paragraph}{4}{\z@}%
            {-0.5ex\@plus -.1ex \@minus -.25ex}%
            {-1em}%
            {\normalfont\normalsize\bfseries}}
\renewcommand\subparagraph{\@startsection{subparagraph}{5}{\z@}%
            {-0.1ex\@plus -.1ex \@minus -.25ex}%
            {-1em}%
            {\normalfont\normalsize\bfseries}}
\makeatother

\pagestyle{fancy}
\lhead{}
\chead{}
\rhead{}
\lfoot{}
\cfoot{\thepage}
\rfoot{v0.7}

\newcommand\tab[1][1cm]{\hspace*{#1}} 
\newcommand\tabi[1][0.05cm]{\hspace*{#1}}
\newcommand\tabni[1][0.2cm]{\hspace*{#1}}
\newcommand\tabyon[1][0.4cm]{\hspace*{#1}}
\newcommand\tabroku[1][0.6cm]{\hspace*{#1}}

\XeTeXlinebreaklocale "ja"
\XeTeXlinebreakskip=0em plus 0.1em minus 0.01em

\setlist{nolistsep}
\titlespacing*{\section}
{0pt}{1.5ex plus 1ex minus .2ex}{1.5ex plus .2ex}
\titlespacing*{\subsection}
{0pt}{0.5ex plus 1ex minus .2ex}{0.5ex plus .2ex}

\setlength{\parindent}{0pt}
\setCJKmainfont{ipaexm.ttf}
\setcounter{tocdepth}{5}

\begin{document}	%\large {1.5ex \@plus .2ex}%

%\tableofcontents
%\newpage
\section{新しい単語}
%\section{Days and Months}
%\begin{tabular}{ l l l l l l l l l }
\begin{tabular}{ l | p{4cm} p{12cm} }
海 & うみ & sea \\ \hline\\[-1em]
晴れ & はれ & clear weather; fine weather \\ \hline\\[-1em]
長い & ながい & long \\ \hline\\[-1em]
午前 & ごぜん & morning (A.M.) \\ \hline\\[-1em]
午後 & ごご & afternoon (P.M.) \\ \hline\\[-1em]
叩く & たたく & to strike /to hit /to clap /to beat /to tap \\ \hline\\[-1em]
洗う & あらう & to wash \\ \hline\\[-1em]
後で & あとで & afterwards \\ \hline\\[-1em]
先に & さきに & before; earlier than; first; ahead; previously; \\ \hline\\[-1em]
払う & はらう & to pay \\ \hline\\[-1em]
先ず & まず & first (of all); to start with; \\ \hline\\[-1em]
服 & ふく & clothes  \\ \hline\\[-1em]
着る & きる & to put on (or wear) upper-body clothing \\ \hline\\[-1em]
靴 & くつ & shoes \\ \hline\\[-1em]
履く & はく & to put on (or wear) lower-body clothing or footwear \\ \hline\\[-1em]
脱ぐ & ぬぐ & to take off (clothes) /to undress \\ \hline\\[-1em]
被る & かぶる & to put on (one's head) \\ \hline\\[-1em]
帽子 & ぼうし & hat	 \\ \hline\\[-1em]
取る & とる & to remove (one's glasses, etc.) \\ \hline\\[-1em]
下着 & したぎ & underwear \\ \hline\\[-1em]
掛かる & かかる & to call /to put on (glasses) \\ \hline\\[-1em]
外す & はずす & to unfasten; to undo​ (glasses /rings) \\ \hline\\[-1em]
立つ & たつ & to stand; to rise; to stand up​  \\ \hline\\[-1em]
座る & すわる & to sit; to squat​ \\ \hline\\[-1em]
寝る & ねる & to lie down​ /to go to bed; to lie in bed​ \\ \hline\\[-1em]
---- & いろいろ & various \\ \hline\\[-1em]
臭い & くさい & stinking; smelly​ \\ \hline\\[-1em]
危険 & きけん & danger; peril; hazard​; risk \\ \hline\\[-1em]
危ない & あぶない & dangerous; risky; hazardous; perilous; precarious​ \\ \hline\\[-1em]
---- & うるさい & noisy; loud \\ \hline\\[-1em]
賑やか & にぎやか & bustling; busy; lively​ \\ \hline\\[-1em]
重い & おもい & heavy; weighty; massive​ \\ \hline\\[-1em]
軽い & かるい & light (i.e. not heavy); feeling light \\ \hline\\[-1em]
---- & おかしい & peculiar; strange; odd; funny; laughable; ridiculous; weird; unusual; eccentric​ \\ \hline\\[-1em]
汚い & きたない & dirty; filthy; foul; unclean​ \\ \hline\\[-1em]
広い & ひろい & wide​; spacious; vast \\ \hline\\[-1em]
特別 & とくべつ & special /particular \\ \hline\\[-1em]
退屈 & たいくつ & tedium; boredom \\ \hline\\[-1em]
狭い & せまい & narrow; confined \\ \hline\\[-1em]
珍しい & めずらしい & unusual; rare; curious​; new; novel​ \\ \hline\\[-1em]
考える & かんがえる & to think about; to take into consideration; to ponder; to devise; to scheme​ \\ \hline\\[-1em]
文法 & ぶんぽう & grammar; syntax \\ \hline\\[-1em]
説明 & せつめい & explanation; exposition​ \\ \hline\\[-1em]
---- & ただいまう & Here I am; I'm home!​ \\ \hline\\[-1em]
---- & いただきます & expression of gratitude before meals​ \\ \hline\\[-1em]
---- & どうぞ & by all means; feel free; you are welcome​; here you are \\ \hline\\[-1em]
お帰りなさい & おかえりなさい & welcome back /welcome home %\\ \hline\\[-1em]
\end{tabular}

\section{もっと新しい単語}
\begin{tabular}{ l | p{4cm} p{12cm} }
---- & このなんで & among these \\ \hline\\[-1em]
世界 & せかい & the world; society; the universe​ \\ \hline\\[-1em]
世界で & せかいで & in the world \\ \hline\\[-1em]
全部 & ぜんぶ & all; entire; whole; altogether​ \\ \hline\\[-1em]
全部で & ぜんぶで & in the whole lot \\ \hline\\[-1em]
---- & ごちそうさまでした & that was a delicious meal \\ \hline\\[-1em]
お粗末さまでした & おそまつさまでした & expression of humility said by the person who provided a meal after it is eaten \\ \hline\\[-1em]
違う & ちがう & to differ (from); to vary \\ \hline\\[-1em]
間違う & まちがう & to make a mistake; to be incorrect; to be mistaken​ \\ \hline\\[-1em]
間違える & まちがえる & to make a mistake (in); to commit an error (e.g. in calculation)​; to confuse; to mistake something for something else​ %\\ \hline\\[-1em]
\end{tabular}

\section{新しいカタカナの単語}
\begin{tabular}{ l | l }
コップ & glass /tumbler \\ \hline\\[-1em]
カップ & cup /mug \\ \hline\\[-1em]
コーヒーカップ & coffee cup \\ \hline\\[-1em]
フルート & flute \\ \hline\\[-1em]
ハーモニカ & harmonica \\ \hline\\[-1em]
サンダル & sandals \\ \hline\\[-1em]
ジャケット & jacket \\ \hline\\[-1em]
コート & coat \\ \hline\\[-1em]
ジーンズ & jeans \\ \hline\\[-1em]
セーター & sweater	 \\ \hline\\[-1em]
スカート & skirt \\ \hline\\[-1em]
Tシャツ & T-shirt \\ \hline\\[-1em]
イヤリング & earring \\ \hline\\[-1em]
リング & ring \\ \hline\\[-1em]
ピアッシング & piercing \\ \hline\\[-1em]
ベルト	 & belt \\ \hline\\[-1em]
コンタクトレンズ & contact lens​ \\ \hline\\[-1em]
ネクタイ & necktie %\\ \hline\\[-1em]
\end{tabular}

\end{document}
